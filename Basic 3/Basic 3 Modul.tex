\section{Aljabar}
    \subsection{Persamaan Eksponen dan Logaritma}
    \begin{enumerate}
        \item $a^0=1$ untuk $a \neq 0$.
        \item $a^n =  \underbrace{a \cdot a \cdot \ldots \cdot a}_{n \text{ kali}}$ untuk $n \in \NN$.
        \item $a^b\cdot a^c=a^{b+c}$.
        \item $a^b\cdot c^b = (ac)^b$.
        \item $\dfrac{a^b}{a^c}=a^{b-c}$ untuk $a\neq 0$.
        \item $(a^b)^c=a^{bc}$.
        \item $a^{-m} = \dfrac{1}{a^m}$ untuk $a \neq 0$.
        \item $\sqrt[n]{a^m}=a^{\dfrac{m}{n}}$
    \end{enumerate}
    
    Definisi: $a^b =c \iff b = ^a \log c = \log_a c$ untuk $a> 0$, $a \neq 1$, dan $c >0$.
    \begin{enumerate}
        \item $^a \log b = \dfrac{^p \log b}{^p \log a}$.
        \item $^a \log a = 1$.
        \item $^a \log b = \dfrac{1}{^b \log a}$.
        \item $^a \log b + ^a \log c = ^a \log (bc)$.
        \item $^a \log b - ^a \log c = ^a \log \left(\dfrac{b}{c}\right)$.
        \item $^a \log b^n = n \cdot ^a \log b$.
        \item $^(a^n) \log b^m = ^a \log b^{\dfrac{m}{n}} = \dfrac{m}{n}\cdot ^a \log b$.
        \item $^a \log b \cdot ^b\log c = ^a \log c$.
    \end{enumerate}
    \subsection{Ketaksamaan}
    \subsubsection{QM-AM-GM-HM}
    Untuk bilangan real positif $a_1,a_2,\dots,a_n$ dengan $n\ge 2$,definisikan
    \begin{align*}
        QM \text{ (Quadratic Mean) } &= \sqrt{\dfrac{a_1^2+a_2^2+\dots+\a_n^2}{n}}\\
        AM \text{ (Arithmetic Mean) } &= \dfrac{a_1+a_2+\dots+a_n}{n}\\
        GM \text{ (Geometric Mean) } &=
        \sqrt[n]{a_1a_2\dots a_n}\\
        HM \text{ (Harmonic Mean) } &=
        \dfrac{n}{\dfrac{1}{a_1}+\dfrac{1}{a_2}+\dots+\dfrac{1}{a_n}}
    \end{align*}
    
    Maka berlaku $QM \ge AM \ge GM \ge HM$ atau 
    $$\sqrt{\dfrac{a_1^2+a_2^2+\dots+a_n^2}{n}} \ge  \dfrac{a_1+a_2+\dots+a_n}{n}\ge
        \sqrt[n]{a_1a_2\dots a_n} \ge
        \dfrac{n}{\dfrac{1}{a_1}+\dfrac{1}{a_2}+\dots+\dfrac{1}{a_n}}$$
    dengan kesamaan terjadi jika dan hanya jika $a_1=a_2=\dots =a_n$.
    
    \subsubsection{Cauchy-Schwarz}
    Untuk bilangan real $a_1,a_2,\dots,a_n$ dan $b_1,b_2,\dots,b_n$ berlaku
    $$(a_1^2+a_2^2+\dots+a_n^2)(b_1^2+b_2^2+\dots+b_n^2) \ge (a_1b_1+a_2b_2+\dots+a_nb_n)^2$$
    
    dengan kesamaan terjadi jika dan hanya jika $\dfrac{a_1}{b_1}=\dfrac{a_2}{b_2}=\dots =\dfrac{a_n}{b_n}$.
    
    \subsubsection{Ketaksamaan Bernoulli}
    Untuk $x > -1$ berlaku $(1+x)^n \ge 1+nx$.
    
    \subsection{Ketaksamaan Kuadrat}
    Untuk $x \in \RR$, berlaku kuadrat sempurnanya selalu nonnegatif atau $x^2 \ge 0$. Hal ini menyebabkan fungsi kuadrat $f(x)=ax^2+bx+c$ untuk $a \neq 0$ selalu mempunyai nilai minimum atau maksimum saat $x = -\dfrac{b}{2a}$. (why?)
     \subsection{Floor and Ceiling}
    Definisikan $\floor{x}$ (floor x) sebagai bilangan bulat terbesar yang kurang dari sama dengan $x$. Simpelnya, $\floor{x}$ dapat dikatakan sebagai "pembulatan ke bawah". Contoh: $\floor{\pi}=3$, $\floor{2}=2$, $\floor{10,51}=10$, $\floor{-1,5}=-2$.
    
    Definisikan $\ceiling{x}$ (ceiling x) sebagai bilangan bulat terkecil yang lebih dari sama dengan $x$. Simpelnya, $\ceiling{x}$ dapat dikatakan sebagai "pembulatan ke atas". Contoh: $\ceiling{\pi}=4$, $\ceiling{2}=2$, $\ceiling{10,51}=11$, $\ceiling{-1,5}=-1$.
    
    Beberapa properti:
    \begin{enumerate}
        \item $\floor{x}=\ceiling{x}$ untuk $x \in \ZZ$.
        \item $\floor{x}=\ceiling{x}-1$ untuk $x \not \in \ZZ$.
        \item $\floor{x} \le  x < \floor{x}+1$ untuk $x \in \RR$.
        \item $\ceiling{x}-1 < x \le \ceiling{x}$ untuk $x \in \RR$.
        \item $\floor{a+x}=a+\floor{x}$ dan $\ceiling{a+x}=a+\ceiling{x}$ untuk $a\in \ZZ$ dan $x \in \RR$.
    \end{enumerate}
    
    \subsubsection{Hermite's Identity}
    Untuk sembarang bilangan real $x$ dan bilangan bulat positif $n$,
    $$\floor{nx}=\floor{x}+\floor{x+\dfrac{a}{n}}+\floor{x+\dfrac{2}{n}}+\dots+\floor{x+\dfrac{n-1}{n}}.$$

    
    \section{Teori Bilangan}
    \subsection{Inverse Modulo}
    Misalkan bilangan bulat $a$, $x$ dan bilangan bulat positif $m$. Kita sebut $x$ adalah inverse dari $a \mod m$ jika dan hanya jika $gcd(a,m)=1$ dan $ax \equiv 1 \mod m$.
    
    \subsection{Bilangan Prima Serta Trik-trik Modulo Umum}
    Bilangan prima adalah bilangan asli yang hanya dapat dibagi dirinya sendiri dan angka 1. 
    
    Bilangan bukan prima dan bukan 1 disebut bilangan komposit.
    
    1 bukan bilangan prima dan bukan pula bilangan komposit. 
    
    Untuk bilangan prima $p$ dan bilangan bulat $n$.
    \begin{enumerate}
        \item Bilangan prima genap hanya ada satu buah, yaitu 2.
        \item Dari definisi bilangan prima $p$, karena $p$ tak terbagi oleh $2$ dan $5$, maka tak ada bilangan prima yang berakhiran $0$.
        \item Untuk sembarang bilangan prima $p$ berlaku $p \mid n$ atau $gcd(p,n)=1$.
        \item $p \mid n^2$ jika dan hanya jika $p \mid n$.
        \item $p \mid ab \iff p \mid a \text{ atau } p \mid b$.
        \item Untuk $p > 3$, kita punya bentuk $p = 6k \pm 1$ untuk suatu bilangan asli $k$.
        \item (Sieve of Erastosthenes) 
        Faktor prima terkecil $t$ dari bilangan komposit $n$ selalu $t \le \sqrt{n}$.
        \end{enumerate}
        
    Lalu, beberapa trik-trik modulo umum:
    \begin{enumerate}
        \item Pada sistem persamaan bulat, tinjau modulo 3, 4, 5, 7, atau modulo 11 nya.
        \item Untuk bilangan bulat $n$ selalu terjadi $n^2 \equiv 1 \mod 4$, $n^2 \equiv 1 \mod 3$. Peninjauan terhadap modulo lain juga bisa, namun tidak terlalu umum.
    \end{enumerate}
    
    \subsection{Basis Bilangan}
    Basis bilangan adalah sistem bilangan yang menyatakan banyaknya digit atau kombinasi dari digit-digit yang menyatakan sebuah bilangan.
    
    Secara umum, bilangan $a$ dalam basis $n > 0$ yaitu $(a)_n$ mempunyai bentuk (yang setara dengan nilai basis 10):
    $$(c_kc_{k-1}\dotsc_1c_0)_n = c_{k}n^k + c_{k-1}n^{k-1}+\dots+c_1n^{1}+c_0n^{0}$$
    
    Secara umum bahkan kita telah memakai sistem basis tersebut untuk basis 10. Misalkan 123 dapat dinyatakan sebagai $123 = 1\cdot 10^2 + 2\cdot 10^1 + 1\cdot 10^0$
    
    Lalu, berikut merupakan contoh untuk bilangan basis selain 10 misalnya: 
    \begin{itemize}
        \item Bilangan basis 2 atau bilangan biner yang digit-digitnya terdiri dari $\{0,1\}$. Misalkan $1001_2$ dalam biner yang setara dengan $9$ atau $1001_2 = 9$ karena $1001_2 = 1\cdot 2^3+0\cdot 2^2+0\cdot 2^1+1\cdot 2^0 = 9$. 
        \item Bilangan basis 3 yang digit-digitnya terdiri dari $\{0,1,2\}$. Misalkan $211_3 = 22$ karena $211_3 = 2\cdot 3^2+ 1\cdot 3^1+ 1\cdot 3^0 = 22$.
        \item Bilangan basis 16 atau heksadesimal yang digit-digitnya terdiri dari $\{0,1,2,\dots,9,A,B,\dots,F\}$. Misalkan $5F_{16} = 95$ karena $5F_{16} = 5 \cdot 16^1 + (15)\cdot 16^0 = 95$.
    \end{itemize}
   
    \section{Kombinatorika}
    Seluruh bagian kombinatorika ini disadur dari buku Diktat Pembinaan Olimpiade Matematika Dasar versi 5.2 karya Pak Eddy Hermanto.
    
    \subsection{Percobaan}
Misalkan kita melempar sekeping uang logam, maka kegiatan ini disebut dengan percobaan. Hasil 
percobaan yang didapat biasanya adalah munculnya sisi gambar, $G$, atau munculnya sisi tulisan, $T$. 
Sedangkan jika kita melempar sebuah dadu, maka hasil percobaan yang didapat adalah mata dadu 1, 
2, 3, 4, 5 atau 6. 

\subsection{Ruang Contoh atau Ruang Sampel} 
Ruang contoh atau ruang sampel adalah himpunan dari semua hasil percobaan yang mungkin. Ruang 
contoh atau ruang sampel biasanya dilambangkan dengan $S$ yang dalam teori himpunan disebut 
dengan himpunan semesta. 
Pada percobaan melempar uang logam, ruang sampelnya adalah $\{G, T\}$ sedangkan pada percobaan
melempar satu buah dadu, ruang sampelnya adalah $\{1, 2, 3, 4, 5, 6\}$. 
Jika $\{G, T\}$ adalah ruang sampel, maka anggota-anggota dari ruang sampel tersebut disebut titik
contoh. Titik contoh dari $\{G, T\}$ adalah $G$ dan $T$. Pada percobaan melempar satu buah dadu, titik 
sampel yang didapat ada 6 yaitu 1, 2, 3, 4, 5, 6 sedangkan jika melempar dua buah dadu akan didapat
36 buah titik contoh, yaitu $(1, 1), (1, 2), (1, 3), \dots , (6, 6)$. 

\subsection{Kejadian}
Kejadian atau peristiwa (event) adalah himpunan bagian dari ruang contoh yang dapat berupa
kejadian sederhana maupun kejadian majemuk. Kejadian sederhana adalah suatu kejadian yang
hanya mempunyai sebuah titik contoh. Jika suatu kejadian memiliki lebih dari satu titik contoh 
disebut dengan kejadian majemuk. 
Kejadian munculnya mata dadu satu $\{1\}$ pada percobaan melempar sebuah dadu adalah contoh 
kejadian sederhana. Contoh dari kejadian majemuk adalah munculnya mata dadu genap pada 
percobaan melempar sebuah dadu. 
    
\subsection{Peluang Suatu Kejadian}
Menghitung peluang dengan pendekatan frekuensi 
Dari suatu percobaan yang dilakukan sebanyak $n$ kali, ternyata kejadian $A$ munculnya sebanyak 
$k$ kali, maka frekuensi nisbi munculnya kejadian $A$ sama dengan 
$$p(A)=\dfrac{k}{n}$$
Kalau $n$ semakin besar dan menuju tak terhingga maka nilai $p(A)$ akan cenderung konstan 
mendekati suatu nilai tertentu yang disebut dengan peluang munculnya kejadian $A$.

    \subsection{Prinsip Inklusi Eksklusi}
    Pada dasarnya adalah konsep dari mengurangi "kelebihan hitung". Contohnya adalah soal himpunan yang dinyatakan dalam rumus berikut
    $$|A \cup B|=|A|+|B|-|A \cap B|.$$
    
    Untuk tiga himpunan $A,B,C$ adalah
    $$|A \cup B \cup C|=|A|+|B|+|C|-|A \cap B|-|A \cap C|-|B \cap C|+|A \cap B \cap C|.$$
    
    dan seterusnya. Lebih lengkapnya boleh mengacu ke \href{https://brilliant.org/wiki/principle-of-inclusion-and-exclusion-pie/}{link ini}
    
    \subsubsection{Derangement}
    Teorema ini juga bisa disebut "teorema kado silang". Bunyi teorema ini:
    
    Misalkan $n$ adalah bilangan bulat non-negatif. Kita sebut $!n$ atau $D_n$ sebagai derangement dari $n$ yaitu banyaknya permutasi $n$ elemen berbeda sedemikian sehingga tidak ada elemen yang menempati tempatnya semula.
    
    \textbf{Versi yang tidak terlalu abstrak:} $!n$ adalah derangement dari $n$, dimana misalkan pada sebuah pesta ulang tahun, $n$ orang saling bertukar kado (awalnya semua orang mempunyai tepat satu kado) dimana setelah bertukar kado tidak ada orang yang mendapat kado dari dirinya sendiri. Banyak kemungkinan pertukaran kado ini adalah $!n$.
    
    Rumus umum untuk menghitung derangement adalah
    $$!n = n! \left(\dfrac{1}{0!}-\dfrac{1}{1!}+\dfrac{1}{2!}-\dfrac{1}{3!}+\dfrac{1}{4!}-\dfrac{1}{5!}+\dots+(-1)^n\dfrac{1}{n!}\right).$$
    
    
    
    \section{Geometri}
    
    \subsection{Power of a point}
    Diberikan lingkaran $\Gamma$ dan titik $P$ yang terletak di dalam atau di luar lingkaran $\Gamma$. Maka definisikan kuasa atau power dari $P$ terhadap lingkaran $\Gamma$ sebagai
    $$Pow_\Gamma (P) = |OP^2-r^2|$$
    dimana $O$ adalah pusat dari $\Gamma$ dan $r$ adalah jari-jari lingkaran $\Gamma$.
    
    Jika $A,B,C,D$ berada di $\Gamma$, serta $AB$ dan $CD$ berpotongan di $P$, maka $$Pow_\Gamma(P)=PA \cdot PB = PC \cdot PD.$$
    Jika $P$ berada di luar $\Gamma$ dan $E$ berada di $\Gamma$ sehingga $PE$ bersinggungan dengan $\Gamma$ di $E$, maka $$Pow_\Gamma (P) = PE^2 =  PA \cdot PB = PC \cdot PD.$$
    
    \subsection{Dalil Ptolemy}
        Diketahui sebuah segiempat siklis $ABCD$ maka berlaku
        $$AB \cdot CD + BC \cdot DA = AC \cdot BD.$$
        
    \subsection{Dalil Stewart}
        Pada segitiga $ABC$ dengan titik $D$ pada segmen $BC$, dimana $AB=c, BC=a, CA=b, AD=d, BD=m, CD=n$, maka berlaku
        $$BC \cdot AD^2 + BC \cdot BD \cdot CD = CA^2 \cdot BD + AB^2 \cdot CD$$
        atau $$ad^2+amn = b^2m+c^2n.$$
        
    \subsection{Dalil Sinus dan Dalil Cosinus}
        Misalkan $ABC$ adalah suatu segitiga dengan $R$ adalah panjang jari-jari lingkaran luarnya. Maka
        \subsubsection{Dalil Sinus}
        $$\dfrac{BC}{\sin \angle A} = \dfrac{CA}{\sin \angle B}= \dfrac{AB}{\sin \angle C} = 2R$$
        
        \subsubsection{Dalil Cosinus}
        \begin{align*}
            AB^2 &= BC^2 + CA^2 - 2\cdot BC \cdot CA \cdot \sin \angle C\\
            BC^2 &= CA^2 + AB^2 - 2\cdot CA \cdot AB \cdot \sin \angle A\\
            CA^2 &= AB^2 + BC^2 - 2\cdot AB \cdot BC \cdot \sin \angle B
        \end{align*}
        
        
    \subsection{Teorema Miquel}
        Pada segitiga $ABC$, titik $D,E,F$ berturut-turut berada di garis $BC,CA,AB$ maka lingkaran luar dari $\triangle AEF$, $\triangle  BDF$, $\triangle CDE$ akan bertemu atau berpotongan di satu titik, sebut sebagai titik $M$. Titik $M$ ini biasa disebut sebagai \textit{Miquel Point}.
        
    \section{Latihan Soal}
    \subsection{Aljabar}
    \begin{enumerate}
        \item Carilah jumlah semua bilangan bulat positif $a$ yang memenuhi $a^{(a-1)^{(a-2)}}=a^{a^2-3a+2}$.
        
        \item Carilah jumlah seluruh solusi real $x$ yang memenuhi $(x^2+5x+5)^{x^2-10x+21}=1.$
        
        \item Jika $5^x=6^y=30^7$, berapakah nilai $\dfrac{xy}{x+y}$?
        
        \item (OSK 2012) Jumlah semua bilangan bulat $x$ sehingga $^2 \log (x^2-4x-1)$ merupakan bilangan bulat adalah \dots
        
        \item (OSK 2014) Misalkan $x,y,z>1$ dan $w>0$. Jika $\log_x w = 4$, $\log_y w = 5$, dan $\log_{xyz} w = 2$, maka nilai $\log_z w$ adalah \dots 
        
        \item (Modifikasi JBMO 2021) Carilah seluruh penyelesaian dari persamaan $2\cdot \lfloor{\frac{1}{2x}}\rfloor - 7 = 9(1 - 8x)$.
        
        \item Banyaknya bilangan asli $n \in \{1,2,3,\dots,1000\}$ sehingga terdapat bilangan real positif $x$ yang memenuhi $x^2+\floor{x}^2=n$ adalah \dots
        
        \item (OSK 2016) Misalkan $x,y,z$ adalah bilangan real positif yang memenuhi $$3 \log_x (3y) = 3 \log_{3x} (27z) = \log_{3x^4} (81yz) \neq 0.$$ Nilai dari $x^5y^4z$ adalah \dots
        
        \item (Nesbitt's Inequality) Untuk bilangan real positif $a,b,c$ tentukan nilai minimum dari $$\dfrac{a}{b+c}+\dfrac{b}{c+a}+\dfrac{c}{a+b}.$$
        
        \item Tentukan nilai minimum dari $8x^4+y^2$ untuk bilangan real positif $x$ dan $y$ yang memenuhi $x^4y=\dfrac{1}{2}$.
        
        \item Nilai minimum dari $$\dfrac{1}{w}+\dfrac{1}{x}+\dfrac{1}{y}+\dfrac{1}{z}$$
        untuk bilangan real positif $w,x,y,z$ yang memenuhi $w+x+y+z=3$ adalah \dots
        
        \item Jika $x^2+y^2+z^2=1$, nilai maksimum dari $x+2y+3z$ adalah \dots
        
        \item Diberikan $a+b+c=1$ dan $a,b,c>0$, carilah nilai minimum dari $a^2+2b^2+c^2$.
        
        \item (OSK 2014) Untuk $0 < x < \pi$, nilai minimum dari $\dfrac{16 \sin^2 x + 9}{\sin x}$ adalah \dots
        
        \item (OSK 2017) Misalkan $a,b,c$ bilangan real positif yang memenuhi $a+b+c=1$. Nilai minimum dari $\dfrac{a+b}{abc}$ adalah \dots
        
        \item (OSK 2017) Pada segitiga $ABC$ titik $K$ dan $L$ berturut-turut adalah titik tengah $AB$ dan $AC$. Jika $CK$ dan $BL$ saling tegak lurus, maka nilai minimum dari $\cot B + \cot C$ adalah \dots
    \end{enumerate}
    \subsection{Teori Bilangan}
        \begin{enumerate}
            \item (OSK 2012) Banyaknya bilangan bulat $n$ yang memenuhi $$(n-1)(n-3)(n-5)\dots(n-2013)=n(n+2)(n+4)\dots (n+2012)$$ adalah \dots
            
            \item (OSK 2013) Diketahui $x_1,x_2$ adalah dua bilangan bulat berbeda yang merupakan akar-akar dari persamaan kuadrat $x^2+px+q+1=0$. Jika $p$ dan $p^2+q^2$ adalah bilangan-bilangan prima, maka nilai terbesar yang mungkin dari $x_1^{2013}+x_2^{2013}$ adalah \dots
            
            \item (OSK 2014) Diberikan tiga bilangan bulat positif berurutan. Jika bilangan pertama tetap, bilangan kedua ditambah 10 dan bilangan ketiga ditambah bilangan prima, maka ketiga bilangan ini membentuk deret ukur. Bilangan ketiga dari bilangan bulat berurutan adalah \dots
            
            \item (OSK 2014) Semua pasangan bilangan prima $(p,q)$ yang memenuhi persamaan
            $$(7p-q)^2=2(p-1)q^2$$
            adalah \dots
            
            \item (OSK 2014) Semua bilangan bulat $n$ sehingga $n^4-51n^2+225$ merupakan bilangan prima adalah \dots
        \end{enumerate}
        
    \subsection{Kombinatorika}
        \begin{enumerate}
            \item (OSK 2012) Suatu set soal terdiri dari 10 soal pilihan B atau S dan 15 soal pilihan ganda dengan 4 pilihan. Seorang siswa menjawab semua soal dengan menebak jawaban secara acak. Tentukan probabilitas ia menjawab dengan benar hanya 2 soal.
            
            \item (OSK 2012) Misalkan terdapat 5 kartu dimana setiap kartu diberi nomor yang berbeda yaitu 2, 3, 4, 5, 6. Kartu-kartu tersebut kemudian dijajarkan dari kiri ke kanan secara acak sehingga berbentuk barisan. Berapa probabilitas bahwa banyaknya kartu yang dijajarkan dari kiri ke kanan dan ditempatkan pada tempat ke- $i$ akan lebih besar atau sama dengan $i$ untuk setiap $i$ dengan $1 \le i \le 5$ ?
            
            \item (OSK 2013) Suatu dadu ditos enam kali. Banyak cara memperoleh jumlah mata yang muncul 28 dengan tepat satu dadu muncul angka 6 adalah \dots
            
            \item (OSK 2013) Sepuluh kartu ditulis dengan angka satu sampai sepuluh (setiap kartu hanya terdapat satu angka dan tidak ada dua kartu yang memiliki angka yang sama). Kartu - kartu tersebut dimasukkan kedalam kotak dan diambil satu secara acak. Kemudian sebuah dadu dilempar. Probabilitas dari hasil kali angka pada kartu dan angka pada dadu menghasilkan bilangan kuadrat adalah \dots
            
            \item (OSK 2018) Diberikan satu koin yang tidak seimbang. Bila koin tersebut ditos satu kali, peluang muncul angka adalah $\frac{1}{4}$. Jika ditos $n$ kali, peluang muncul tepat dua angka sama dengan peluang muncul tepat tiga angka. Nilai $n$ adalah \dots
            
            \item (OSK 2017) Pada suatu kotak ada sekumpulan bola berwarna merah dan hitam yang secara keseluruhannya kurang dari 1000 bola. Misalkan diambil dua bola. Peluang terambilnya dua bola merah adalah $p$ dan peluang terambilnya dua bola hitam adalah $q$ dengan $p-q =\frac{23}{37}$. Selisih terbesar yang mungkin dari banyaknya bola merah dan hitam adalah \dots
            
            \item (OSK 2017) Terdapat enam anak, $A, B, C, D, E$ dan $F$, akan saling bertukar kado. Tidak ada yang menerima kadonya sendiri, dan kado dari $A$ diberikan kepada $B$. Banyaknya cara membagikan kado dengan cara demikian adalah \dots
        \end{enumerate}
    
    \subsection{Geometri}
        \begin{enumerate}
            \item (\textbf{Soal Legend: OSK 2011,2012,2013,2018}) Diberikan segitiga $ABC$ dan lingkaran $\Gamma$ yang berdiameter $AB$ . Lingkaran $\Gamma$ memotong sisi $AC$ dan $BC$ berturut-turut di titik $D$ dan $E$. Jika $AD = \frac13 AC, BE =\frac14 BC$ dan $AB = 30$, maka luas segitiga $ABC$ adalah \dots
            
            \item (OSK 2016) Diberikan empat titik pada satu lingkaran $\Gamma$ dalam urutan $A,B,C,D$. Sinar garis $AB$ dan $DC$ berpotongan di $E$, dan sinar garis $AD$ dan $BC$ berpotongan di $F$. Misalkan $EP$ dan $FQ$ menyinggung lingkaran $\Gamma$ berturut-turut di $P$ dan $Q$. Misalkan pula bahwa $EP=60$ dan $FQ=63$, maka panjang $EF$ adalah \dots
            
            \item (OSK 2013) Misalkan $P$ adalah titik interior dalam daerah segitiga $ABC$ sehingga besar $\angle PAB = 10^\circ, \angle PBA = 20^\circ, \angle PCA = 30^\circ, \angle PAC=40^\circ$. Besar $\angle ABC = \dots$
            
            \item (OSK 2012) Diberikan segitiga $ABC$ dengan keliling 3, dan jumlah kuadrat sisi-sisinya sama dengan 5. Jika jari-jari lingkaran luarnya sama dengan 1, maka jumlah ketiga garis tinggi dari segitiga $ABC$ tersebut adalah \dots
            
            \item (OSK 2014) Diberikan segitiga $ABC$ yang sisi-sisinya tidak sama panjang sehingga panjang garis berat $AN$ dan $BP$ berturut-turut 3 dan 6. Jika luas segitiga $ABC$ adalah $3\sqrt{15}$, maka panjang garis berat ketiga $CM$ adalah \dots
            
            \item (OSK 2016) Pada segitiga $ABC$, titik $M$ terletak pada $BC$ sehingga $AB=7, AM=3, BM=5$, dan $MC=6$. Panjang $AC$ adalah \dots
            
            \item Diberikan sebuah segiempat siklis $ABCD$ dengan $ABC$ adalah segitiga sama sisi. Jika $AD=2$ dan $CD=3$, panjang $BD=\dots$
            
            \item (OSK 2017) Pada sebuah lingkaran dengan pusat $O$, talibusur $AB$ berjarak 5 dari titik $O$ dan talibusur $AC$ berjarak $5\sqrt{2}$ dari titik $O$. Jika panjang jari-jari lingkaran 10, maka $BC^2=\dots$
        \end{enumerate}
            \section{Referensi}
        \begin{enumerate}
            \item Hermanto, Eddy. 2011. Diktat Pembinaan Olimpiade Matematika Dasar.
        \end{enumerate}