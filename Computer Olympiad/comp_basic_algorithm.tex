
	\title{Handout Basic Programming and Algorithm} % Beginner
	\date{\today}
	\author{Azzam L. H.}
	\maketitle

 \renewcommand*\contentsname{Daftar Isi}
\tableofcontents
	\noindent
	\section{Referensi dan Situs Utama untuk Latihan}
	\begin{enumerate}
	\item Belajar programming dasar:           \url{https://tlx.toki.id/courses/basic} dan \url{https://usaco.guide/}
	\item Latihan pemrograman kompetitif: \url{https://tlx.toki.id/courses/competitive} dan \url{https://usaco.guide/problems/}
	\item Ebook materi pemrograman kompetitif dasar: \url{https://osn.toki.id/data/pemrograman-kompetitif-dasar.pdf}
	\end{enumerate}
	
	
	
	\section{Bilangan Prima}
	Bilangan prima adalah bilangan asli yang hanya dapat dibagi dirinya sendiri dan angka 1. 
	    
	 Bilangan bukan prima dan bukan 1 disebut bilangan komposit.
	 
	 1 bukan bilangan prima dan bukan pula bilangan komposit. 
	 
	 Untuk bilangan prima $p$ dan bilangan bulat $n$.
	 \begin{enumerate}
	     \item Bilangan prima genap hanya ada satu buah, yaitu 2.
	     \item Dari definisi bilangan prima $p$, karena $p$ tak terbagi oleh $2$ dan $5$, maka tak ada bilangan prima yang berakhiran $0$.
	     \item Untuk sembarang bilangan prima $p$ selalu berlaku $p \mid n$ atau $gcd(p,n)=1$ untuk suatu bilangan bulat $n$.
	     \item $p \mid n^2$ jika dan hanya jika $p \mid n$.
	     \item $p \mid ab \iff p \mid a \text{ atau } p \mid b$.
	     \item Untuk $p > 3$, kita punya bentuk $p = 6k \pm 1$ untuk suatu bilangan asli $k$.
	     \item Untuk seluruh faktor prima $t$ dari bilangan komposit $n$ selalu berlaku $t \le \sqrt{n}$.
	  \end{enumerate}
	  
	\subsection{Algoritma Untuk Mengecek Bilangan Prima}
	\subsubsection{Naive}
	\begin{lstlisting}
	bool isPrime(int n)
	{
		// https://www.geeksforgeeks.org/prime-numbers/
	    // Corner case
	    if (n <= 1)
	        return false;
	 
	    // Check from 2 to n-1
	    for (int i = 2; i < n); i++)
	        if (n % i == 0)
	            return false;
	    return true;
	}
	\end{lstlisting}
	
	\subsubsection{Better Naive}
	\begin{lstlisting}
		bool isPrime(int n)
		{
			// https://www.geeksforgeeks.org/prime-numbers/
		    // Corner case
		    if (n <= 1)
		        return false;
		 
		    // Check from 2 to square root of n
		    for (int i = 2; i <= sqrt(n); i++)
		        if (n % i == 0)
		            return false;
		    return true;
		}
		\end{lstlisting}
	
	\subsection{Prime Generating Algorithm}
	\subsubsection{Sieve of Eratosthenes}
	\begin{lstlisting}
	// modified from https://www.geeksforgeeks.org/sieve-of-eratosthenes/
	int[] SieveOfEratosthenes(int n)
	{
	    // Create a boolean array
	    // "prime[0..n]" and initialize
	    // all entries it as true.
	    // A value in prime[i] will
	    // finally be false if i is
	    // Not a prime, else true.
	    bool prime[n + 1];
	    memset(prime, true, sizeof(prime));
	 
	    for (int p = 2; p * p <= n; p++)
	    {
	        // If prime[p] is not changed,
	        // then it is a prime
	        if (prime[p] == true)
	        {
	            // Update all multiples
	            // of p greater than or
	            // equal to the square of it
	            // numbers which are multiple
	            // of p and are less than p^2
	            // are already been marked.
	            for (int i = p * p; i <= n; i += p)
	                prime[i] = false;
	        }
	    }
	}
	\end{lstlisting}
	
	
  	\section{FPB dan KPK dua bilangan}
	 Secara matematis FPB (atau gcd - greatest common divisors) dan KPK (atau lcm - least common multiples) dari dua bilangan bulat positif $a$ dan $b$ 
	 didefinisikan sebagai:
	 $$FPB(a,b) = gcd(a,b) = p_1^{\min\{a_1,b_1\}}\cdot p_2^{\min\{a_2,b_2\}} \cdot \ldots \cdot p_n^{\min\{a_n,b_n\}}$$
	 $$KPK(a,b) = lcm(a,b) =p_1^{\max\{a_1,b_1\}}\cdot p_2^{\max\{a_2,b_2\}} \cdot \ldots \cdot p_n^{\max\{a_n,b_n\}}$$
	 dimana kedua bilangan tersebut dapat difaktorisasi prima menjadi
	 $a=p_1^{a_1}\cdot p_2^{a_2}\cdot \ldots \cdot p_n^{a_n}$ dan $b=p_1^{b_1}\cdot p_2^{b_2} \cdot \ldots \cdot p_n^{b_n}$, dengan $p_1,p_2,\dots,p_n$ adalah bilangan prima berbeda, serta $a_1,a_2,\dots,a_n,b_1,b_2,\dots,b_n$ adalah bilangan bulat non-negatif.
	 
	 Dari definisi tersebut mudah dibuktikan bahwa
	 $$FPB(a,b) \cdot KPK(a,b) = ab.$$
	 
	 Perlu dicatat, bahwa $a$ dan $b$ tidak boleh bernilai nol. Untuk $a,b$ yang bernilai negatif, didefinisikan $FPB(a,b) = FPB(|a|,|b|)$.
	 
	 \subsection{Algoritma Naive}
	 Mencari seluruh faktor i yang mungkin secara mundur dari $m = min(a,b)$ sampai 1, lalu return $FPB$ nya jika ternyata $i \mid a$ dan $i \mid b$ nya. Kompleksitasnya adalah $O(m)$
	 
	 \begin{lstlisting}
	 int naiveGCD(int a, int b){
	   int m = min(a,b);
	   int gcd = 1;
	   for(int i = m; i > 0; --i){
	     if((a % i == 0) && (b % i == 0)){
	       gcd = i;
	       return gcd;
	     }
	   }
	 \end{lstlisting}
	 
	 
	 \subsection{Algoritma Euclidean}
	 Pada dasarnya algoritma ini bertumpu pada sebuah teorema:
	 $$FPB(a,b) = FPB(a,b-a)$$
	 
	 Jika teorema di atas dilakukan berkali-kali, maka secara umum akan didapat
	 $$FPB(a,b) = FPB(b\mod a , a)$$
	 
	 \begin{lstlisting}
	 // algoritma secara iteratif
	 int euclideanGCD(int a, int b) {
	   	while(a > 0) {
	   		int temp = a;
	   		a = b % a;
	   		b = temp;
	 	}
	 	return b;
	 }
	 \end{lstlisting}
	
	\begin{lstlisting}
	// algoritma secara rekursif
	int euclideanGCD(int a, int b) {
	    if (a == 0) {
	        return b;
	    }
	    return euclideanGCD(b % a, a);
	}
	\end{lstlisting}
	
	\subsection{Algoritma Mencari KPK}
	Dari sifat $FPB(a,b) \cdot KPK(a,b) = ab$, maka
	 \begin{lstlisting}
	 // fungsi yang menghitung KPK dari dua bilangan
	 int lcm(int a, int b) {
		 	// fungsi gcd  dapat memakai naive atau euclidean
		 	return (a * b) / gcd(a, b); 
	 }
	 \end{lstlisting}
	 
	  \section{Fungsi yang Melibatkan Faktor Bilangan}
	     Misalkan $a$ dapat difaktorisasi prima seperti sebelumnya yaitu $a=p_1^{a_1}\cdot p_2^{a_2}\cdot \ldots \cdot p_n^{a_n}$.
	     \subsection{Banyaknya Faktor Positif}
	     Fungsi $d(a)$ didefinisikan sebagai banyaknya faktor atau pembagi positif dari $a$ dengan
	     $$d(a) = (a_1+1)(a_2+1)\cdot \ldots \cdot (a_n+1).$$
	     
	     Contoh: Banyaknya faktor positif dari $12= 2^2 \cdot 3^1$ adalah $d(12)=(2+1)(1+1)=6$ dengan pembagi positifnya adalah $1,2,3,4,6,12$ (ada 6 faktor positif.)
	     
	     \subsection{Jumlah Faktor Positif}
	     Fungsi $\sigma (a)$ didefinisikan sebagai banyaknya faktor atau pembagi positif dari $a$ dengan
	     $$\sigma (a) = (p_1^0+p_1^1+p_1^2+\dots+p_1^{a_1})(p_2^0+p_2^1+p_2^2+\dots+p_2^{a_2})\cdot \ldots \cdot (p_n^0+p_n^1+p_n^2+\dots+p_n^{a_n}).$$
	     
	     Contoh: Jumlah faktor atau pembagi positif dari $12= 2^2 \cdot 3^1$ adalah $d(12)=(2^0+2^1+2^2)(3^0+3^1)=28$ yang setara dengan penjumlahan secara manualnya, yaitu $2^03^0+2^03^1+2^13^0+2^13^1+2^23^0+2^23^1=28.$
	     
	     \section{Uji habis dibagi}
	         Trik yang suatu saat dapat membuat hidup anda bahagia wkwkwk. Semua rumus ini dapat dibuktikan dengan aritmatika modular.
	         \begin{enumerate}
	             \item Bilangan $x$ genap jika dan hanya jika digit terakhir $x$ genap.
	             \item $3 \mid x$ jika dan hanya jika jumlah digit-digitnya habis dibagi $3$. Contohnya 2931 habis dibagi 3 karena $2+9+3+1=15$ habis dibagi 3.
	             \item $9 \mid x$ jika dan hanya jika jumlah digit-digitnya habis dibagi $9$.
	             \item $x$ habis dibagi 5 jika dan hanya jika digit terakhir $x$ adalah $0$ atau $5$.
	             \item $x$ habis dibagi 11 jika dan hanya jika jumlah selang-seling (alternate sums) dari digit-digitnya habis dibagi 11. Contoh: 945351 habis dibagi 11 karena $9-4+5-3+5-1=11$ habis dibagi 11. 121 habis dibagi 11 karena $1-2+1=0$ habis dibagi 11.
	         \end{enumerate}
	         
\section{Barisan dan Deret Aritmatika}
    Barisan aritmatika in a nutshell: $a,a+b,a+2b,\dots$ dimana setiap \textbf{suku ke-$n$} dari barisan tersebut adalah $$a_n=a+(n-1)b$$ untuk suatu bilangan asli $n$ dan bilangan \textbf{real} $a,b$.
    
    Lalu, \textbf{rumus deret}nya atau jumlah $n$ suku pertama barisan tersebut adalah (coba buktikan rumus ini) $$S_n = a_1+a_2+\dots+a_n=a+(a+b)+(a+2b)+\dots+(a+(n-1)b)=\dfrac{n}{2}(2a+(n-1)b).$$
    
    \subsection{Barisan dan Deret Geometri}
     Barisan geometri in a nutshell: $a,ar,ar^2,\dots$ dimana setiap \textbf{suku ke-$n$} dari barisan tersebut adalah $$a_n=ar^{n-1}$$ untuk suatu bilangan asli $n$ dan bilangan \textbf{real} $a$ dan $r \neq 0$.
    
    Lalu, \textbf{rumus deret}nya atau jumlah $n$ suku pertama barisan tersebut adalah (coba buktikan rumus ini) $$S_n = a_1+a_2+\dots+a_n=a+ar+ar^2+\dots+ar^{n-1}=\dfrac{a(1-r^n}{1-r}.$$
    Perhatikan jika $-1 < r < 1$ maka saat $n \rightarrow \infty$ rumus deret tak hingganya menjadi (why?) $$S_\infty =  \dfrac{a}{1-r}.$$
    
    \subsection{Rumus Barisan dan Deret Lainnya}
    Sangat dianjurkan untuk mencoba membuktikan rumus-rumus berikut. Untuk bilangan asli $n$, kita punya
    \begin{enumerate}
        \item $1+2+\dots+n = \dfrac{n(n+1)}{2}.$
        \item $1^2+2^2+\dots+n^2 = \dfrac{n(n+1)(2n+1)}{6}.$
        \item $1^3+2^3+\dots+n^3 = \left(1+2+\dots+n\right)^2= \left(\dfrac{n(n+1)}{2}\right)^2.$
    \end{enumerate}
    
    \section{Basis Bilangan}
        Basis bilangan adalah sistem bilangan yang menyatakan banyaknya digit atau kombinasi dari digit-digit yang menyatakan sebuah bilangan.
        
        Secara umum, bilangan $a$ dalam basis $n > 0$ yaitu $(a)_n$ mempunyai bentuk (yang setara dengan nilai basis 10):
        $$(c_kc_{k-1}\dotsc_1c_0)_n = c_{k}n^k + c_{k-1}n^{k-1}+\dots+c_1n^{1}+c_0n^{0}$$
        
        Secara umum bahkan kita telah memakai sistem basis tersebut untuk basis 10. Misalkan 123 dapat dinyatakan sebagai $123 = 1\cdot 10^2 + 2\cdot 10^1 + 1\cdot 10^0$
        
        Lalu, berikut merupakan contoh untuk bilangan basis selain 10 misalnya: 
        \begin{itemize}
            \item Bilangan basis 2 atau bilangan biner yang digit-digitnya terdiri dari $\{0,1\}$. Misalkan $1001_2$ dalam biner yang setara dengan $9$ atau $1001_2 = 9$ karena $1001_2 = 1\cdot 2^3+0\cdot 2^2+0\cdot 2^1+1\cdot 2^0 = 9$. 
            \item Bilangan basis 3 yang digit-digitnya terdiri dari $\{0,1,2\}$. Misalkan $211_3 = 22$ karena $211_3 = 2\cdot 3^2+ 1\cdot 3^1+ 1\cdot 3^0 = 22$.
            \item Bilangan basis 16 atau heksadesimal yang digit-digitnya terdiri dari $\{0,1,2,\dots,9,A,B,\dots,F\}$. Misalkan $5F_{16} = 95$ karena $5F_{16} = 5 \cdot 16^1 + (15)\cdot 16^0 = 95$.
        \end{itemize}
        Berikut disajikan contoh program yang dapat mengkonversi bilangan bulat positif basis 10 ke biner
        \begin{lstlisting}
        int convertToBinary(int n){ // for positive integer only
        	int result = 0;
        	while(n > 0){
        		result = 10 * result + (n % 2);
        		n = n / 2;
        	}
        	return result;
        }
        \end{lstlisting}
        Dan berikut program yang dapat mengkonversi bilangan biner bulat positif ke bilangan basis 10
        \begin{lstlisting}
        int convertFromBinary(int bin){ // for positive integer only
                	int result = 0;
                	int exponent = 1;
                	while(bin > 0){
                		result += (bin % 10)*exponent;
                		bin = bin / 10;
                		exponent = exponent * 2;
                	}
                	return result;
        }
        \end{lstlisting}
    
    \section{Latihan Soal}
	\begin{enumerate}	
	\item Berapa banyak bilangan bulat antara 1 sampai dengan 100 yang habis dibagi 3 atau 5?
	\item Berapa banyak bilangan bulat antara 1 sampai dengan 100 yang tidak habis dibagi 3 atau tidak 
	habis dibagi 5?
	\item (OSK 2018) Jika $FPB(a,2008)=251$. Jika $a < 4036$, maka nilai terbesar untuk $a$ adalah
	\item (OSK 2019)  Bilangan ajaib adalah bilangan yang memiliki jumlah faktor yang menyisakan 1 apabila dibagi 4, sebagai 
	contoh adalah angka 1, 1 memiliki 1 buah faktor (yaitu 1). Untuk kesekian kalinya, Pak Dengklek ingin 
	meminta tolong kalian untuk menghitung ada berapa banyak bilangan ajaib yang berada diantara 1 dan 
	300 inklusif. Ada berapakah bilangan ajaib yang ingin diketahui pak Dengklek? 
	\item (OSK 2019) Bilangan Harshad didefinisikan sebagai bilangan yang habis dibagi oleh hasil penjumlahan setiap digit 
	dari bilangan itu sendiri. Contohnya bilangan 18, karena 18 habis dibagi oleh 9. Ada berapa banyak 
	bilangan Harshad dari 1 sampai 50? 
	\item (OSK Matematika 2010) Nilai $n$ terkecil sehingga $\underbrace{20102010\dots2010}_\text{$n$ buah 2010}$ habis dibagi 99 adalah \dots
	
	\textbf{(Modifikasi OSK 2017) Perhatikan potongan kode berikut untuk 2 soal selanjutnya:}
	\begin{lstlisting}
	long n, temp;
	cin >> n;
	while(n >= 10) {
		temp = 1;
		while(n > 0) {
			temp = temp * (n % 10);
			n = n / 10;
		}
		n = temp;
	}
	\end{lstlisting}
	\item Berapakah nilai akhir $n$, jika nilai $n$ pada awalnya adalah 62792912?
	\item Berapakah nilai akhir $n$, jika nilai $n$ pada awalnya adalah 14934976?
	
	\item Perhatikan potongan kode berikut!
		 \begin{lstlisting}
		 int batu(int x, int y) {
		 	if(x == 0) {
		 		return y;
		 	} else {
		 		return batu(y % x, x);
		 	}
		 }
		 int kertas(int n) {
		 	int ans = 0;
		 	for(int i = 1; i <= n; i++) {
		 		for(int j = 1; j <= i; j++) {
		 			if(batu(i,j) == 1) {
		 				ans += n / i;
		 			}
		 		}
		 	}
		 	return ans;
		 }
		 \end{lstlisting}
		 Nilai  kembalian dari pemanggilan fungsi \verb*|kertas(100)| adalah \dots
		 
	\item (Modifikasi OSP 2008) Perhatikan potongan kode berikut
	\begin{lstlisting}
	int a;
	cin >> a ;
	int b = 4;
	while(a > 0) {
		b = b + (a % 10);
		a = a / 10;
	}
	if(((b % 3) > 0) || ((b % 9) > 0)){
		cout << "Angin bertiup\n";
	} 
	else {
		cout << "Angin semilir\n";
	} 
	\end{lstlisting}
	Berapakah nilai \verb*|a| yang akan menghasilkan keluaran "Angin semilir" ?
	
	\item (Modifikasi OSP 2006) Diberikan potongan program berikut
	\begin{lstlisting}
	void z(int t) {
		int k, j;
		k = 1;
		while(k < t) {
			j = 1;
			while(j < 100) {
				cout << "*";
				j = j + 10;
			}
			k = k + 1;
		}
	}
	\end{lstlisting}
	Dengan suatu harga pada variabel N dan 
	memanggil \verb*|z(N)| maka jumlah karakter \verb*|*|
	yang akan dicetak sebagai fungsi dari $N$:
	(dengan $c$ adalah suatu bilangan konstan 
	positif)
	\begin{enumerate}[a)]
	\item $\floor{N/10}$
	\item $(\floor{c \log N})^2$
	\item $cN$
	\item $\floor{c \log N}$
	\item $N \floor{c \log N}$
	\end{enumerate}
	
	\textbf{(Modifikasi OSP 2008) Perhatikan potongan kode program berikut  untuk menjawab dua pertanyaan berikutnya}
	\begin{lstlisting}
	int i,j,k,x,y;
	cin >> x ;
	i = 0;
	y = 0;
	while (i < x) {
		for(j = 0; j <= i; j++) {
			y = y + 2 * i;
		}
		i = i + 1;
	}
	for(k = 0; k <= y; i++){
		cout << 'a';
	}
	\end{lstlisting}
	\item Berapakah $y$ yang dihasilkan di akhir program jika $x$ adalah 3?
	\begin{enumerate}[a)]
	\item 13
	\item 14
	\item 15
	\item 16
	\item 17
	\end{enumerate}
	
	\item Berapakah nilai $x$ minimum yang diperlukan 
	supaya di layar tertulis yang lebih dari 80 huruf 
	$'a'$ di akhir program?
	\begin{enumerate}[a)]
		\item 2
		\item 4
		\item 5
		\item 10
		\item 15
	\end{enumerate}
	
	\textbf{(Modifikasi OSP 2008)Untuk dua pertanyaan berikutnya, perhatikanlah potongan kode berikut}
	\begin{lstlisting}
	int k;
	k = 1;
	for(int i=1; i < 5; i++){
		k = k * i;
		for(int j =  i + 1; j <= 2*i; j++) {
			k += j;
		}
	}
	cout << abs(k) << endl;
	\end{lstlisting}
	\item Keluaran dari program di atas adalah \dots
	\begin{enumerate}[a)]
		\item 242
		\item 1250
		\item 54
		\item 1
		\item 7557
	\end{enumerate}
	
	\item Agar program menghasilkan keluaran minimum, 
	nilai $k$ harus diinisialisasi dengan \dots
		\begin{enumerate}[a)]
			\item 0
			\item -1
			\item -9
			\item -10
			\item -11
		\end{enumerate}
		
	\item (Modifikasi OSK 2018) Diberikan potongan program sebagai berikut:
	\begin{lstlisting}
	int kancil, panda, j;
	for(int i = 2; i <= 100; i++) {
		j = 1;
		kancil = 0;
		while(j <= i) {
			if(i % j == 0) kancil++;
			j++;
		}
		if(kancil == 2) panda++;
	}
	cout << panda;
	\end{lstlisting}
\begin{enumerate}[a)]
			\item 10
			\item 15
			\item 25
			\item 30
			\item 40
	\end{enumerate}
	
\item (Modifikasi OSP 2008) Perhatikan kode berikut:
	\begin{lstlisting}
	bool tes1(int n) {
		bool ok;
		ok = true;
		for(int i = 2; i <= trunc(n)+1; i++){
			if(n % i == 0) ok = false;
		}
		return ok;
	}
	\end{lstlisting}
	Manakah pemanggilan yang menghasilkan false?
	\begin{enumerate}[a)]
	\item tes1(43)
	\item tes1(51)
	\item tes1(53)
	\item tes1(59)
	\item tes1(67)
	\end{enumerate}
	
\item (Modifikasi OSP 2008) Perhatikan kode berikut:
	\begin{lstlisting}
	bool tes2(int n) {
		bool ok = true;
		int j = 0;
		for(int i = 1; i <= n/2; i++){
			if(n % i == 0) j += i * 2;
		}
		return (j == 2*n);
	}
	\end{lstlisting}
	Manakah pemanggilan yang menghasilkan true?
	\begin{enumerate}[a)]
	\item tes2(12)
	\item tes2(24)
	\item tes2(28)
	\item tes2(29)
	\item tes2(36)
	\end{enumerate}
	
	\item (Modifikasi OSP 2008) Perhatikan potongan fungsi berikut:
	\begin{lstlisting}
	bool tes3(int n, int m){
		int i, j, tmp;
		tmp = n;
		i = 0;
		while(tmp > 0){
			if(tmp % 2 == 1) i++;
			tmp = tmp / 2;
		}
		tmp = m;
		j = 0;
		while(tmp > 0){
			if(tmp % 2 == 1) j++;
			tmp = tmp / 2;
		}
		return (i == j);
	}
	\end{lstlisting}
	Manakah yang menghasilkan true?
	\begin{enumerate}[a)]
	\item tes3(99,100)
	\item tes3(100,120)
	\item tes3(56,121)
	\item tes3(89,156)
	\item tes3(90,173)
	\end{enumerate}
	
	\textbf{(Modifikasi OSP 2008)Untuk dua pertanyaan berikutnya, perhatikanlah potongan kode berikut}
		\begin{lstlisting}
		int sum(int n){
			if(n>0) return n + sum(n-1);
			else {
				sum = n;
			}
			
		}
		\end{lstlisting}
		\item Hasil dari \verb*|sum(11)| adalah \dots
		\begin{enumerate}[a)]
			\item 55
			\item 66
			\item 11
			\item 78
			\item 91
		\end{enumerate}
		
		\item Agar keluaran fungsi \verb*|sum(n)>100|, 
		berapakah harga \verb*|n| terkecil?
			\begin{enumerate}[a)]
				\item 11
				\item 12
				\item 13
				\item 14
				\item 15
			\end{enumerate}
			
		\textbf{(Modifikasi OSP 2008) Untuk tiga pertanyaan berikutnya perhatikanlah kode program lengkap sebagai berikut:}
		\begin{lstlisting}
		#include <iostream>
		using namespace std;
		
		int hitung(int i){
			int j = 1, hasil = 0;
			while(j <= i) {
				hasil += j;
				j++;
			}
			return hasil;
		}
		
		int main (){
			int x = -1;
			int y = 4;
			cin >> z;
			cout << hitung(x) << endl;
			int t = hitung(y);
			for(int i = 15; i >= t; i--) {
				cout << "y";
			}	
			cout << "\n";
			if(hitung(z) > 0){
				for(int i = 1; i <= z; i++) {
					for(int k = 1; k <= i; i++) {
						cout << "*";
					}
				}
			}
		}
		\end{lstlisting}
		\item Angka berapakah yang pertama kali tertulis di 
		layar, jika program tersebut berhasil dijalankan?
		\item Berapa banyakah huruf "y" yang akan tertulis di 
		layar?
		\item Berapakah z yang harus dimasukkan pengguna 
		untuk menuliskan 15 karakter "*" di layar? 
	\end{enumerate}
	     
	\section{Relasi Rekurensi atau Rekursif}
		    Sering disebut dengan rekursif. Intinya adalah sebuah persamaan yang melibatkan barisan $a_1, a_2, \dots , a_n$ dimana untuk mendapatkan nilai $a_k$ membutuhkan suku-suku sebelumnya $a_{k-1}, a_{k-2}, \dots,$ atau $a_1$. 
		    
		    \subsection{Barisan Fibonacci}
		    Contoh paling terkenal dari persamaan rekursif adalah bilangan Fibonacci $0,1,1,2,3,5,8,13,21,\dots$ yang secara matematis bilangan Fibonacci ke-$n$ atau $F_n$ didefinisikan sebagai berikut.
		    \begin{align*}
		        F_0 &= 0, \hspace{20pt} F_1 = 1\\
		        F_n &= F_{n-1}+F_{n-2} \text{ untuk } n \ge 2
		    \end{align*}
		    
		    Berikut disajikan kode untuk mencari bilangan Fibonacci ke-$n$.
		    \begin{lstlisting}
		    int fibonacci(int n){
		    	if(n == 0 || n == 1) return n;
		    	return fibonacci(n-1) + fibonacci(n-2);
		    }
		    \end{lstlisting}
		    
		    Contoh yang lebih terkenal di ranah \textit{Computer Science} adalah permasalahan \textit{Tower of Hanoi} atau permasalahn menara Hanoi.
		    \begin{remark*} 
		    Disadur dari \url{https://id.wikipedia.org/wiki/Menara_Hanoi}, 
		    menara Hanoi adalah sebuah permainan matematis atau teka-teki. Permainan ini terdiri dari tiga tiang dan sejumlah cakram dengan ukuran berbeda-beda yang bisa dimasukkan ke tiang mana saja. Permainan dimulai dengan cakram-cakram yang tertumpuk rapi berurutan berdasarkan ukurannya dalam salah satu tiang, cakram terkecil diletakkan teratas, sehingga membentuk kerucut.
		    
		    Tujuan dari teka-teki ini adalah untuk memindahkan seluruh tumpukan ke tiang yang lain, mengikuti aturan berikut:
		    
		    \begin{enumerate}
		    \item Hanya satu cakram yang boleh dipindahkan dalam satu waktu.
		    	    \item Setiap perpindahan berupa pengambilan cakram teratas dari satu tiang dan memasukkannya ke tiang lain, di atas cakram lain yang mungkin sudah ada di tiang tersebut.
		    	    \item Tidak boleh meletakkan cakram di atas cakram lain yang lebih kecil.
		    \end{enumerate}
		    \includegraphics[scale=0.5]{Towers-Of-Hanoi.png}\\
		    Sumber gambar: \url{https://www.javainterviewpoint.com/tower-hanoi-java-recursion/}
		    \end{remark*}
		    Persamaan rekursifnya didefinisikan sebagai berikut dengan $T_n$ menotasikan banyaknya langkah yang diperlukan untuk memindahkan $n$ cakram.
		    \begin{align*}
		        T_1 &= 1 \\
		        T_n &= 2T_{n-1}+1
		    \end{align*}
		    Berikut merupakan kode secara rekursif dari permasalahan Tower of Hanoi.
		    \begin{lstlisting}
		    int TowerOfHanoi(int n) {
		    	if(n == 1){ 
		    		return 1;
		    	} else {
		    		return 2*TowerOfHanoi(n-1) + 1;
		    	}
		    	
		    }
		    \end{lstlisting}
		    
		    
		    Untuk menyelesaikan soal relasi rekurensi, butuh manipulasi aljabar yang mumpuni sehingga tidak ada pendekatan selain menggunakan persamaan karakteristik atau fungsi pembangkit (tidak dibahas disini) yang dijamin berhasil.
		    
		    
		    
		    \section{Dynamic Programming}
		    Pada dasarnya Dynamic Programming atau \textbf{DP} adalah Relasi Rekursif , tetapi lebih efisien karena ditambah dengan konsep memoisasi (memoization).
		    
		    Konsep Dynamic Programming sendiri dibagi dua yaitu \textbf{Top-Down} dan \textbf{Bottom-Up}. Untuk materi lebih lanjut, silakan membaca Buku \href{https://ksn.toki.id/data/pemrograman-kompetitif-dasar.pdf}{Pemrograman Kompetitif Dasar} yang disediakan oleh TOKI.
		    
		    \subsection{Contoh Kasus Bilangan Fibonacci ke-n}
		    Sebelumnya kita telah mendapatkan program yang mengembalikan bilangan Fibonacci secara rekursif 
		    \begin{lstlisting}
		    int fibonacci(int n){
		    	if(n == 0 || n == 1) return n;
		    	return fibonacci(n-1) + fibonacci(n-2);
		    }
		    \end{lstlisting}
		    Ternyata solusi tersebut masih kurang bagus, karena masih ada beberapa fungsi yang dipanggil lebih dari sekali. Sebagai contoh misalkan kita ingin menghitung $F_4$. Maka urutan pemanggilannya adalah sebagai berikut:
		    
		    $F_4$:
		    \begin{enumerate}
		    \item $F_3$
			    \begin{enumerate}[a)]
			    \item $F_2$
			    \begin{enumerate}[i)]
			    \item $F_1$
			    \item $F_0$
			    \end{enumerate}
			    \item $F_1$
			    \end{enumerate}
		    \item $F_2$
		    		\begin{enumerate}[a)]
		    	    \item $F_1$
		    	    \item $F_0$
		    	    \end{enumerate}
		    \end{enumerate}	
		    Perhatikan bahwa $F_2, F_1, F_0$ terpanggil lebih dari sekali. Tentu saja hal tersebut boros secara waktu dan resources komputasi karena seharusnya kalau kita sudah mendapatkan suatu bilangan Fibonacci, maka bilangan itu disimpan saja. 
		    
		    Proses penyimpanan tersebut yang dinamakan Memoization atau memoisasi yang merupakan konsep dasar dari DP.
		    
		    Berikut program yang mengembalikan Bilangan Fibonacci ke-$n$ menggunakan DP (Top-Down)
		    \begin{lstlisting}
		    // disadur dari https://www.geeksforgeeks.org/program-for-nth-fibonacci-number/
		    int fib(int n)
		    {
		        // Declare an array to store
		        // Fibonacci numbers.
		        // 1 extra to handle
		        // case, n = 0
		        int f[n + 2];
		        int i;
		     
		        // 0th and 1st number of the
		        // series are 0 and 1
		        f[0] = 0;
		        f[1] = 1;
		     
		        for(i = 2; i <= n; i++)
		        {
		             
		           //Add the previous 2 numbers
		           // in the series and store it
		           f[i] = f[i - 1] + f[i - 2];
		        }
		        return f[n];
		        }
		    };
		    \end{lstlisting}
		    
		    \newpage
		    \section{Latihan Soal}
		    \begin{enumerate}
		    \item Akan digunakan beberapa buah domino ukuran 2 $\times$ 1 atau 1 $\times$ 2 untuk dipasang di lantai ukuran 2 $\times$ n.
		    Untuk mengisi lantai ukuran 2 $\times$ 1 ada 1 cara.
		    Untuk mengisi lantai ukuran 2 $\times$ 2 ada 2 cara. Dan seterusnya.
		    Berapa banyak cara untuk mengisi lantai ukuran 2 $\times$ 8 ?
		    
		    \item (KSN-K 2021) Seekor semut berada di bidang dimensi dua. Ketika berada di koordinat $(x,y)$, dia bisa 
	 	    bergerak ke kanan $(x+1,y)$, kiri $(x-1,y)$, atas $(x,y+1)$, atau bawah $(x,y-1)$ dengan aturan 
	 	    tertentu sesuai dengan tabel berikut:\\
	 	    	\includegraphics[scale=0.5]{latsol1.png}\\
	 	    Tabel tersebut menunjukkan aturan pergerakan semut tersebut. Baris menunjukkan 
	 	    gerakan yang terakhir dilakukan sementara kolom menunjukkan ganjil/genapnya
	 	    koordinat semut sekarang. Misalnya, jika sekarang berada di koordinat $(10,12)$ yaitu 
	 	    (genap,genap) dan sebelumnya bergerak ke kiri dari $(11,12)$, maka gerakan selanjutnya 
	 	    adalah ke bawah yaitu ke $(10,11)$. Jika semut tersebut ingin bergerak dari koordinat $(0,0)$
	 	    ke koordinat $(1,-2)$, arah gerakan awal yang mungkin adalah ...
	 	    
	 	    \item (KSN-K 2021) Seekor semut berada di bidang dimensi dua. Ketika berada di koordinat $(x,y)$, dia bisa 
	 	     	    bergerak ke kanan $(x+1,y)$, kiri $(x-1,y)$, atas $(x,y+1)$, atau bawah $(x,y-1)$ dengan aturan 
	 	     	    tertentu sesuai dengan tabel berikut:\\
	 	     	    \includegraphics[scale=0.5]{latsol2.png}\\
	 	    Tabel tersebut menunjukkan aturan pergerakan semut tersebut. Baris menunjukkan 
	 	    gerakan yang terakhir dilakukan sementara kolom menunjukkan ganjil/genapnya
	 	    koordinat semut sekarang. Misalnya, jika sekarang berada di koordinat $(10,12)$ yaitu 
	 	    (genap,genap) dan sebelumnya bergerak ke kiri dari $(11,12)$, maka gerakan selanjutnya 
	 	    adalah ke bawah yaitu ke $(10,11)$. Jika semut tersebut ingin bergerak dari koordinat $(0,0)$
	 	    ke koordinat $(-1,-100)$, gerakan awal apa saja yang cocok?
	 	    
	 	    \item (KSN-K 2021) Pak Dengklek mempunyai 3 jenis ubin dengan bentuk dan ukuran sebagai berikut:\\
	 	    	\includegraphics[scale=0.8]{latsol3-4.png}\\
	 	    Sementara Pak Dengklek ingin menutup lantai dengan bentuk berikut :\\
	 	    	\includegraphics[scale=0.8]{latsol3.png}\\
	 	    Tentu saja, ubin yang dipasang tidak boleh tumpang tindih dan keseluruhan ubin harus 
	 	    masuk ke dalam lantai. Ubin dapat dirotasi maupun direfleksi. Apabila banyak stok ubin 
	 	    tak terhingga untuk setiap jenisnya, berapa banyak cara menutup lantai tersebut?
	 	    
	 	    \item (KSN-K 2021)Pak Dengklek mempunyai 3 jenis ubin dengan bentuk dan ukuran sebagai berikut:\\
	 	     	 \includegraphics[scale=0.8]{latsol3-4.png}\\
	 	    Sementara Pak Dengklek ingin menutup lantai dengan bentuk berikut :\\
	 	    	\includegraphics[scale=0.8]{latsol4.png}\\
	 	    Tentu saja, ubin yang dipasang tidak boleh tumpang tindih dan keseluruhan ubin harus 
	 	    masuk ke dalam lantai. Ubin dapat dirotasi maupun direfleksi. Apabila banyak stok ubin 
	 	    tak terhingga untuk setiap jenisnya, berapa banyak cara menutup lantai tersebut?
	 	    
	 	    \item (KSN-K 2021) Tiga ekor bebek Pak Dengklek bernama Kwik, Kwak, dan Kwek sedang bermain. 
	 	    Pertama-tama, mereka menentukan urutan giliran secara adil dengan peluang yang 
	 	    sama. Pada setiap giliran, seekor bebek dapat memilih bebek yang lain untuk diserang,
	 	    atau diam. Setiap serangan mempunyai peluang untuk berhasil tergantung kepada
	 	    penyerangnya. Diketahui semua serangan Kwik berhasil, $1/2$ serangan Kwak berhasil, dan 
	 	    $3/4$ serangan Kwek berhasil. Apabila sebuah serangan berhasil maka bebek yang diserang 
	 	    keluar dari permainan. Namun apabila serangan gagal, maka tidak terjadi apa-apa.
	 	    Permainan akan dilanjutkan sampai tersisa seekor bebek sebagai pemenang. Apabila 
	 	    diasumsikan semua bebek bermain secara optimal, maka siapakah yang mempunyai 
	 	    peluang menang lebih dari $1/2$?
	 	    
	 	    \item (KSN-K 2021) Diberikan denah ruangan seperti pada gambar di bawah.\\
	 	    	\includegraphics[scale=0.8]{latsol6.png}\\
	 	    Mula-mula Pak Dengklek berada pada ruangan S. Setiap saat, dengan peluang yang 
	 	    sama Pak Dengklek berpindah ke ruangan lain yang bersebelahan dengan ruangan saat 
	 	    ini. Jika Pak Dengklek sampai pada ruangan W, Pak Dengklek langsung menang. 
	 	    Sebaliknya, jika sampai pada ruangan L, Pak Dengklek langsung kalah. Selain itu, Pak 
	 	    Dengklek akan tetap berpindah-pindah selama belum menang maupun kalah. Berapa 
	 	    peluang Pak Dengklek untuk menang?
	 	    
	 	    \item (KSN-K 2021) Diberikan denah ruangan seperti pada gambar di bawah.\\
	 	     	    	\includegraphics[scale=0.8]{latsol7.png}\\
	 	     	    Mula-mula Pak Dengklek berada pada ruangan S. Setiap saat, dengan peluang yang 
	 	     	    sama Pak Dengklek berpindah ke ruangan lain yang bersebelahan dengan ruangan saat 
	 	     	    ini. Jika Pak Dengklek sampai pada ruangan W, Pak Dengklek langsung menang. 
	 	     	    Sebaliknya, jika sampai pada ruangan L, Pak Dengklek langsung kalah. Selain itu, Pak 
	 	     	    Dengklek akan tetap berpindah-pindah selama belum menang maupun kalah. Berapa 
	 	     	    peluang Pak Dengklek untuk menang?
	 	     	    
	 	    \item (KSN-K 2021)  Berapa banyak cara memasang 8 ubin berukuran 1 $\times$ 3 pada lantai berukuran 5 $\times$ 5 jika 
	 	    tidak boleh ada ubin yang saling tumpang tindih dan semua ubin masuk di area lantai?
	 	    
	 	    \item (KSN-K 2021)  Di atas papan berukuran 3 $\times$ 3 petak, ingin diletakkan beberapa buah batu sehingga setiap 
	 	    sub-petak berukuran 2 $\times$ 2 terdapat tepat 1 batu di atasnya. Berapa banyak cara yang 
	 	    mungkin?
	 	    
	 	    \item (KSN-K 2021) Diberikan 4 buah pewarna, yaitu: merah, biru, kuning, dan hijau serta petak berukuran 2 
	 	    $\times$ 4. Dua buah petak dikatakan bersebelahan apabila keduanya saling berbagi sisi. Apabila 
	 	    setiap petak bersebelahan tidak boleh mempunyai warna yang sama, berapa banyak cara 
	 	    mewarnai semua petak tersebut?
	 	    
		    \item (KSN-K 2021) Diberikan 3 buah pewarna, yaitu: merah, biru, dan kuning serta petak berukuran 2 $\times$ 4.
		    Dua buah petak dikatakan bersebelahan apabila keduanya saling berbagi sisi. Selain itu, 
		    ujung-ujung petak saling dikaitkan, artinya petak paling kiri dengan petak paling kanan di 
		    suatu baris dianggap bersebelahan juga. Berapa banyak cara mewarnai petak tersebut?
		    
		    \textbf{(KSN-K 2021) Perhatikan potongan program berikut!}
		    \begin{lstlisting}
		    int f(int x, int y) {
		    	if(x == 0 || y == 0) {
		    		return 1;
		    	}
		    	else {
		    		return f(x-1, y) + f(x, y-1);
		    	}
		    }
		    int g(int x, int y) {
	   	    	if(x == 0) {
	   	    		return 1;
	   	    	}
	   	    	else {
	   	    		return g(x-1, y) + f(x, y);
	   	    	}
	 	    }
		    \end{lstlisting}
		    \item  Nilai kembalian dari pemanggilan fungsi \verb*|f(9,5)| adalah ...
		    
		    \item  Nilai kembalian dari pemanggilan fungsi \verb*|g(10,5)| adalah ...
		    
		    \textbf{(KSN-K 2020) Perhatikan potongan program berikut:}
		    \begin{lstlisting}
		    int ayam(int a, int b) {
		    	if(b == 0) {
		    		return 0;
		    	} else if(b == 1) {
		    		return a;
		    	} else {
		    		return ayam(a, b/2) + ayam(a, b % 2);
		    	}
		    }
		    \end{lstlisting}
		    \item Berapakah hasil dari pemanggilan fungsi \verb*|ayam(19,39)|?
		    
		    \end{enumerate}
