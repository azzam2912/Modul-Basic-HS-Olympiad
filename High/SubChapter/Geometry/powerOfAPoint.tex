\subsection{Power of a point}
\begin{figure}[h]
  \begin{asy}
    unitsize(1.5cm);
    pair O, A, B, C, D, E, P;
    O = (0,0);
    P = (4,1);
    E = (0,1);
    A = dir(50);
    C = dir(-30);
    circle o = circumcircle(A,E,C);
    draw(o);
    pair B1[] = intersectionpoints(line(A,P), o);
    pair D1[] = intersectionpoints(line(C,P), o);
    B = B1[0];
    D = D1[0];
    draw(A--B);
    draw(C--D);
    draw(P--A);
    draw(P--D);
    draw(P--E, red);
    draw(P--O, blue+dotted);
    label("$O$",O,SW);
    label("$A$",A,E);
    label("$B$",B,N);
    label("$C$",C,SE);
    label("$D$",D,SW);
    label("$P$",P,NE);
    label("$E$",E,N);
    \end{asy}
    \begin{asy}
    unitsize(1.5cm);
    pair O, A, B, C, D, P;
    O = (0,0);
    A = dir(50);
    C = dir(130);
    D = dir(-30);
    B = dir(-110);
    P = extension(A,B,C,D);
    draw(circumcircle(A,B,C));
    draw(A--B);
    draw(C--D);
    draw(A--C--B--D--cycle);
    draw(O--P, dotted);
    label("$O$",O,SW);
    label("$A$",A,E);
    label("$B$",B,SW);
    label("$C$",C,NW);
    label("$D$",D,SE);
    label("$P$",P,E);
    \end{asy}
\end{figure}
Diberikan lingkaran $\Gamma$ dan titik $P$ yang terletak di dalam atau di luar lingkaran $\Gamma$. Maka definisikan kuasa atau power dari $P$ terhadap lingkaran $\Gamma$ sebagai
$$Pow_\Gamma (P) = |OP^2-r^2|$$
dimana $O$ adalah pusat dari $\Gamma$ dan $r$ adalah jari-jari lingkaran $\Gamma$.

Jika $A,B,C,D$ berada di $\Gamma$, serta $AB$ dan $CD$ berpotongan di $P$, maka $$Pow_\Gamma(P)=PA \cdot PB = PC \cdot PD.$$
Jika $P$ berada di luar $\Gamma$ dan $E$ berada di $\Gamma$ sehingga $PE$ bersinggungan dengan $\Gamma$ di $E$, maka $$Pow_\Gamma (P) = PE^2 =  PA \cdot PB = PC \cdot PD.$$