\subsection{Pigeon Hole Principle (PHP)}
Teorema yang dalam Bahasa Indonesia ini disebut dengan Teorema Sangkar Burung Merpati secara matematis berbunyi:
Jika ada $kn+1$ merpati dan $n$ sangkar, maka setidaknya ada satu sangkar yang berisi $k+1$ burung merpati.

Versi lebih simpelnya adalah: jika ada $n+1$ objek yang akan dibagi ke dalam $n$ buah kotak, maka setidaknya ada 1 kotak yang berisi 2 objek.

Contoh: \begin{itemize}
    \item Di dalam ruangan berisi 3 orang, pasti terdapat setidaknya 2 orang berjenis kelamin sama.
    \item Jika ada 367 orang di suatu sekolah, maka setidaknya ada dua orang diantara mereka yang tanggal lahirnya persis sama.
\end{itemize}