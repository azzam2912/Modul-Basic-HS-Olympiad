\subsection{Prinsip Inklusi Eksklusi}
Prinsip Inklusi (memasukkan, dari kata inklusif) Eksklusi (mengeluarkan, khusus, dari kata eksklusif) atau yang sering disingkat PIE, pada dasarnya adalah konsep dari mengurangi "kelebihan hitung" atau menambahkan "kekurangan hitung". Contohnya adalah soal himpunan yang dinyatakan dalam rumus berikut
$$|A \cup B|=|A|+|B|-|A \cap B|.$$

Untuk tiga himpunan $A,B,C$ adalah
$$|A \cup B \cup C|=|A|+|B|+|C|-|A \cap B|-|A \cap C|-|B \cap C|+|A \cap B \cap C|.$$

dan seterusnya. Lebih lengkapnya boleh mengacu ke \href{https://brilliant.org/wiki/principle-of-inclusion-and-exclusion-pie/}{https://brilliant.org/wiki/principle-of-inclusion-and-exclusion-pie/}

\subsubsection{Derangement}
Teorema ini juga bisa disebut "teorema kado silang". Bunyi teorema ini:

Misalkan $n$ adalah bilangan bulat non-negatif. Kita sebut $!n$ atau $D_n$ sebagai derangement dari $n$ yaitu banyaknya permutasi $n$ elemen berbeda sedemikian sehingga tidak ada elemen yang menempati tempatnya semula.

\textbf{Versi yang tidak terlalu abstrak:} $!n$ adalah derangement dari $n$, dimana misalkan pada sebuah pesta ulang tahun, $n$ orang saling bertukar kado (awalnya semua orang mempunyai tepat satu kado) dimana setelah bertukar kado tidak ada orang yang mendapat kado dari dirinya sendiri. Banyak kemungkinan pertukaran kado ini adalah $!n$.

Rumus umum untuk menghitung derangement adalah
$$!n = n! \left(\dfrac{1}{0!}-\dfrac{1}{1!}+\dfrac{1}{2!}-\dfrac{1}{3!}+\dfrac{1}{4!}-\dfrac{1}{5!}+\dots+(-1)^n\dfrac{1}{n!}\right).$$