\subsection{Peluang}
Misalkan kita melempar sekeping koin, maka kegiatan ini disebut dengan percobaan. Hasil percobaan yang didapat biasanya adalah munculnya sisi gambar, $G$, atau munculnya sisi tulisan, $T$. Ruang contoh atau ruang sampel adalah himpunan dari \textbf{semua hasil percobaan yang mungkin} biasanya dilambangkan dengan $S$, yang dalam teori himpunan disebut dengan himpunan semesta. Pada percobaan melempar koin, ruang sampelnya adalah $\{G, T\}$ sedangkan pada percobaan melempar satu buah dadu, ruang sampelnya adalah $\{1, 2, 3, 4, 5, 6\}$. Jika $\{G, T\}$ adalah ruang sampel, maka anggota-anggota dari ruang sampel tersebut disebut titik contoh. Titik contoh dari $\{G, T\}$ adalah $G$ dan $T$. Pada percobaan melempar satu buah dadu seimbang, titik sampel yang didapat ada 6 yaitu 1, 2, 3, 4, 5, 6 sedangkan jika melempar dua buah dadu akan didapat 36 buah titik contoh, yaitu $(1, 1), (1, 2), (1, 3), \dots , (6, 6)$. 

\subsubsection{Formula Penghitungan Peluang}
Secara mudahnya, enghitung peluang bisa dengan pendekatan frekuensi, yaitu suatu percobaan yang dilakukan sebanyak $n$ kali, ternyata kejadian $A$ munculnya sebanyak $k$ kali, maka frekuensi nisbi/relatif kejadian $A$ sama dengan 
$$p(A)=\dfrac{k}{n}$$
Kalau $n$ semakin besar dan menuju tak terhingga maka nilai $p(A)$ akan cenderung konstan mendekati suatu nilai tertentu yang disebut dengan peluang munculnya kejadian $A$.
