\subsubsection{Barisan dan Deret Aritmatika}
Barisan aritmatika in a nutshell: $a,a+b,a+2b,\dots$ dimana setiap \textbf{suku ke-$n$} dari barisan tersebut adalah $$a_n=a+(n-1)b$$ untuk suatu bilangan asli $n$ dan bilangan \textbf{real} $a,b$.

Lalu, \textbf{rumus deret}nya atau jumlah $n$ suku pertama barisan tersebut adalah (coba buktikan rumus ini) $$S_n = a_1+a_2+\dots+a_n=a+(a+b)+(a+2b)+\dots+(a+(n-1)b)=\dfrac{n}{2}(2a+(n-1)b).$$