\subsection{Latihan Soal Pencacahan: Aturan Penjumlahan dan Perkalian}
\begin{enumerate}
    \item Misalkan Michie mempunyai 3 buah celana dan 4 buah baju. Berapa banyak cara Michie memilih celana dan baju yang akan dipakai ?

    \item Berapa banyak cara menyusun huruf-huruf R, A, J, I, N jika 
    \begin{enumerate}
        \item huruf pertama dimulai dari huruf hidup (vokal) 
        \item huruf pertama dimulai dari huruf mati (konsonan) 
    \end{enumerate}

    \item Sembilan orang siswa akan duduk pada 5 kursi sejajar. Ada berapa cara susunan mereka ? 
    
    \item Denny akan membentuk bilangan genap 3 angka yang angka-angkanya diambil dari 2, 3, 4, 5, 6, 7, 8. Berapa banyak bilangan yang dapat dibentuk jika : 
    \begin{enumerate}
        \item angka-angkanya boleh berulang 
        \item angka-angkanya tidak boleh berulang
    \end{enumerate}

    \item (OSK 2003) Ada berapa banyak bilangan 4-angka (digit) yang semua angkanya genap dan bukan merupakan kelipatan 2003 ?

    \item Sekumpulan orang duduk mengelilingi sebuah meja bundar. Diketahui ada 7 wanita dimana di sebelah kanan setiap wanita tersebut adalah wanita dan ada 12 wanita yang di sebelah kanan setiap wanita tersebut adalah pria. Diketahui pula bahwa 3 dari 4 pria di sebelah kanannya adalah wanita. Berapa orang yang duduk mengelilingi meja tersebut?

    \item (OSK 2015) Masing-masing kotak pada papan catur berukuran $3 \times 3$ dilabeli dengan satu angka yaitu 1, 2, atau 3. Banyaknya penomoran yang mungkin sehingga jumlah angka pada masing-masing baris dan masing-masing kolom habis dibagi 3 adalah \ldots

    \item (OSK 2023) Banyaknya bilangan 4 digit yang habis dibagi 3 dan memuat angka 6 adalah \ldots
\end{enumerate}