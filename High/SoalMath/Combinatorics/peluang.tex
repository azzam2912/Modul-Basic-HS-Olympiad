\subsection{Latihan Soal Peluang}
\begin{enumerate}
        \item (OSK 2015) Suatu dadu ditos enam kali. Probabilitas jumlah mata dadu yang muncul 9 adalah \ldots

        \item (OSK 2012) Suatu set soal terdiri dari 10 soal pilihan B atau S dan 15 soal pilihan ganda dengan 4 pilihan. Seorang siswa menjawab semua soal dengan menebak jawaban secara acak. Tentukan probabilitas ia menjawab dengan benar hanya 2 soal.

        \item (OSK 2013) Suatu partikel bergerak pada bidang Cartesius dari titik $(0, 0)$. Setiap langkah bergerak satu satuan searah sumbu $X$ positif dengan probabilitas $0.6$ atau searah sumbu $Y$ positif dengan probabilitas $0.4$. Setelah sepuluh langkah, probabilitas partikel tersebut sampai pada titik $(6,4)$ dengan melalui titik $(3,4)$ adalah \ldots
        
        \item (OSK 2012) Misalkan terdapat 5 kartu dimana setiap kartu diberi nomor yang berbeda yaitu 2, 3, 4, 5, 6. Kartu-kartu tersebut kemudian dijajarkan dari kiri ke kanan secara acak sehingga berbentuk barisan. Berapa probabilitas bahwa semua kartu yang dijajarkan dari kiri ke kanan dan ditempatkan pada tempat ke- $i$ akan lebih besar atau sama dengan $i$ untuk setiap $i$ dengan $1 \le i \le 5$ ?
        
        \item (OSK 2013) Suatu dadu ditos enam kali. Banyak cara memperoleh jumlah mata yang muncul 28 dengan tepat satu dadu muncul angka 6 adalah \dots
        
        \item (OSK 2013) Sepuluh kartu ditulis dengan angka satu sampai sepuluh (setiap kartu hanya terdapat satu angka dan tidak ada dua kartu yang memiliki angka yang sama). Kartu - kartu tersebut dimasukkan kedalam kotak dan diambil satu secara acak. Kemudian sebuah dadu dilempar. Probabilitas dari hasil kali angka pada kartu dan angka pada dadu menghasilkan bilangan kuadrat adalah \dots
        
        \item (OSK 2018) Diberikan satu koin yang tidak seimbang. Bila koin tersebut ditos satu kali, peluang muncul angka adalah $\frac{1}{4}$. Jika ditos $n$ kali, peluang muncul tepat dua angka sama dengan peluang muncul tepat tiga angka. Nilai $n$ adalah \dots
        
        \item (OSK 2017) Pada suatu kotak ada sekumpulan bola berwarna merah dan hitam yang secara keseluruhannya kurang dari 1000 bola. Misalkan diambil dua bola. Peluang terambilnya dua bola merah adalah $p$ dan peluang terambilnya dua bola hitam adalah $q$ dengan $p-q =\frac{23}{37}$. Selisih terbesar yang mungkin dari banyaknya bola merah dan hitam adalah \dots
\end{enumerate}