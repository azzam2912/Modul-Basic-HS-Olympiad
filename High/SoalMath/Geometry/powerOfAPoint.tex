\subsection{Latihan Soal Power of A Point}
\begin{enumerate}
    \item (\textbf{Soal Legend: OSK 2011,2012,2013,2018}) Diberikan segitiga $ABC$ dan lingkaran $\Gamma$ yang berdiameter $AB$ . Lingkaran $\Gamma$ memotong sisi $AC$ dan $BC$ berturut-turut di titik $D$ dan $E$. Jika $AD = \frac13 AC, BE =\frac14 BC$ dan $AB = 30$, maka luas segitiga $ABC$ adalah \dots

    \item (OSN SL 2010) Pada segitiga $ABC$, misalkan $D$ adalah titik tengah $BC$, dan $BE$, $CF$ adalah garis tinggi. Buktikan bahwa $DE$ dan $DF$ keduanya adalah garis singgung lingkaran luar $\triangle AEF$. %(OSN SL 2010) 

    %AIME 2016 I and AIME 2016 I
    \item (AIME 2016 I) Misalkan $\triangle ABC$ adalah segitiga lancip dengan lingkaran $\omega,$ dan misalkan $H$ adalah titik potong dari garis tinggi $\triangle ABC.$ Garis singgung lingkaran luar $\triangle HBC$ di $H$ memotong $\omega$ pada titik $X$ dan $Y$ dengan $HA=3,HX=2,$ dan $HY=6.$ Carilah luas dari $\triangle ABC$.

    \item (AIME 2016 I) Lingkaran $\omega_1$ dan $\omega_2$ bertemu di titik $X$ dan $Y$. Garis $\ell$ menyinggung lingkaran $\omega_1$ dan $\omega_2$ di $A$ dan $B$, berturut-turut, dengan garis $AB$ lebih dekat ke titik $X$ daripada $Y$. Lingkaran $\omega$ yang melewati $A$ dan $B$, memotong $\omega_1$ lagi di $D \neq A$ dan memotong $\omega_2$ lagi di $C \neq B$. Ketiga titik $C$, $Y$, $D$ segaris dengan $XC = 67$, $XY = 47$, dan $XD = 37$. Carilah panjang $AB$.
\end{enumerate}