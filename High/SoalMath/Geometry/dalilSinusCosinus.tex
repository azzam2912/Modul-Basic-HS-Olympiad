\subsection{Latihan Soal Dalil Sinus dan Cosinus}
\begin{enumerate}
    \item (OSK 2016) Pada segitiga $ABC$, titik $M$ terletak pada $BC$ sehingga $AB=7, AM=3, BM=5$, dan $MC=6$. Panjang $AC$ adalah \dots

    \item (OSK 2013) Misalkan $P$ adalah titik interior dalam daerah segitiga $ABC$ sehingga besar $\angle PAB = 10^\circ, \angle PBA = 20^\circ, \angle PCA = 30^\circ, \angle PAC=40^\circ$. Besar $\angle ABC = \dots$
    
    \item (LMNAS SMP 33 Penyisihan) Diberikan segitiga tumpul $ABC$ dengan $AB = BC$. Titik $D$ berada di dalam segitiga tersebut sedemikian sehingga $AD = BD$, $\angle ADB = 140^\circ$, dan $\angle ADC = 150^\circ$. Besar sudut $ACD$ dalam satuan $^\circ$ (derajat) adalah \dots
    
    \item (Modifikasi OSK 2017) Pada sebuah lingkaran dengan pusat $O$, talibusur $AB$ berjarak 5 dari titik $O$ dan talibusur $AC$ berjarak $5\sqrt{2}$ dari titik $O$ dengan titik $A$ terletak di busur $BC$ yang lebih kecil ($A$ diantara $B$ dan $C$) Jika panjang jari-jari lingkaran 10, maka $BC^2=\dots$
\end{enumerate}