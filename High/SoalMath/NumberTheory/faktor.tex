\subsection{Latihan Soal Fungsi Faktor Bilangan}
\begin{enumerate}
    \item (OSK 2015) Banyaknya faktor bulat positif dari 2015 adalah \ldots

    \item (OSP 2009) Misalkan $n$ bilangan asli terkecil yang mempunyai tepat 2009 faktor dan $n$ merupakan kelipatan 2009. Faktor prima terkecil dari $n$ adalah \dots
    
    \item Misalkan $n$ bilangan asli dimana $2n$ mempunyai 28 faktor positif dan $3n$ mempunyai 30 faktor positif. Banyaknya faktor positif yang dimilik $6n$ adalah \dots
    
    \item (OSK 2011) Ada berapa faktor positif dari $2^73^55^37^2$ yang merupakan kelipatan 10?
    
    \item (AIME 1995) Tentukan banyaknya faktor positif dari $n^2$ yang kurang dari $n$ tetapi tidak membagi $n$ jika $n={2^{31}}3^{19}.$
    
    \item (AIME 2000) Tentukan bilangan asli terkecil yang memiliki 12 faktor positif genap dan $6$ faktor positif ganjil.

    \item Berapa banyak bilangan di himpunan $\{1,2,\dots,200\}$ yang relatif prima dengan 100?

    \item (OSK 2023) Misalkan $n=2^a3^b$ dengan $a$ dan $b$ bilangan asli. Jika hasil kali semua faktor positif dari $n$ adalah $12^{90}$, maka nilai $ab = \ldots$ 

    \item (OSK 2016) Banyaknya bilangan asli $n$ yang memenuhi sifat hasil jumlah $n$ dan suatu pembagi positif $n$  yang kurang dari sama dengan 2016 adalah \dots

    \item (OSK 2017) Misalkan $s(n)$ menyatakan faktor prima $t(n)$ menyatakan faktor prima terkecil dari $n$. Banyaknya bilangan asli $n \in \{1,2,\dots,100\}$ sehingga $t(n)+1=s(n)$ adalah \dots
\end{enumerate}