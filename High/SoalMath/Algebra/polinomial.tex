\subsection{Latihan Soal Polinomial}
\begin{enumerate}

\item Jika $P(x)$ dibagi $x^2-x$ dan $x^2+x$ berturut-turut akan bersisa $5x+1$ dan $3x+1$, maka bila $P(x)$ dibagi $x^2-1$ sisanya adalah \dots

\item (OSP 2006) Jika $(x-1)^2$ membagi $ax^4+bx^3+1$, maka $ab=\dots$

\item Diketahui suatu polinomial $P(x)$ memenuhi $P(k)=\dfrac{k}{k+1}$ untuk $k=1,2,3,\dots,2020$. Jika $P(0)=1$, nilai $P(2022)=\dots$

\item (OSK 2010) Polinom $P(x)=x^3-x^2+x-2$ mempunyai tiga pembuat nol yaitu $a,b,$ dan $c$. Nilai dari $a^3+b^3+c^3$ adalah \dots

\item (OSP 2010) Persamaan kuadrat $x^2-px-2p=0$ mempunyai dua akar real $a$ dan $b$. Jika $a^3+b^3=16$, maka hasil jumlah semua nilai $p$ yang memenuhi adalah \dots 

\item (OSP 2010) Diberikan polinomial $P(x)=x^4+ax^3+bx^2+cx+d$ dengan $a,b,c,$ dan $d$ konstanta. Jika $P(1)=10$, $P(2)=20$, dan  $P(3)=30$, maka nilai
$$\dfrac{P(12)+P(-8)}{10}=\dots$$

\item (OSK 2015) Diketahui $a$, $b$, $c$ akar-akar dari persamaan $x^3 - 5x^2 - 9x + 10 = 0$. Jika suku banyak $P(x) = Ax^3 + Bx^2 + Cx - 2015$ memenuhi $P(a) = b + c$, $P(b) = a + c$ dan $P(c) = a + b$ maka nilai dari $A + B + C$ adalah \ldots

\end{enumerate}