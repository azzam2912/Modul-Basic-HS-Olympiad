\section{Teori Bilangan}
Pada dasarnya aljabar tetapi di ranah bilangan bulat (atau rasional).
\subsection{Latihan Soal Paritas}
\begin{enumerate}
\item (OSK 2012) Banyaknya bilangan bulat $n$ yang memenuhi $$(n-1)(n-3)(n-5)\dots(n-2013)=n(n+2)(n+4)\dots (n+2012)$$ adalah \dots
\end{enumerate}
\subsection{Latihan Soal Paritas}
\begin{enumerate}
\item (OSK 2012) Banyaknya bilangan bulat $n$ yang memenuhi $$(n-1)(n-3)(n-5)\dots(n-2013)=n(n+2)(n+4)\dots (n+2012)$$ adalah \dots
\end{enumerate}
\subsection{Latihan Soal Keterbagian}
\begin{enumerate}    
    \item Carilah semua bilangan bulat $n$ sehingga $\dfrac{2n+6}{n-1}$ adalah bilangan bulat.
    
    \item (OSK 2002) Bilangan asli $n$ terbesar sehingga $8^n \mid 44^{44}$ adalah \dots
    
    \item Berapa banyak pasangan bilangan bulat positif $(a,b)$ yang memenuhi $\dfrac{1}{a}+\dfrac{1}{b}=\dfrac{1}{6}$.
    
    \item Jika $a$ dan $b$ adalah bilangan bulat sedemikian sehingga $a^2-b^2=2017$, maka nilai dari $a^2+b^2$ adalah \dots
    
    \item (AIME 1986) Tentukan bilangan asli $n$ terbesar sehingga $n+10 \mid n^3+100$.

    \item (OSK 2015) Bilangan bulat $x$ jika dikalikan 11 terletak di antara 1500 dan 2000. Jika $x$ dikalikan 7 terletak di antara 970 dan 1275. Jika $x$ dikalikan 5 terletak di antara 690 dan 900. Banyaknya bilangan $x$ sedemikian yang habis dibagi 3 sekaligus habis dibagi 5 ada sebanyak \ldots

    \item (OSK 2023) Banyaknya bilangan 4 digit yang habis dibagi 3 dan memuat angka 6 adalah \ldots
    
    \item (OSN SMP 2003) Buktikan bahwa $(n-1)n(n^3+1)$ selalu habis dibagi 6 untuk semua bilangan asli $n$.
    
\end{enumerate}
\subsection{Latihan Soal Keterbagian}
\begin{enumerate}    
    \item Carilah semua bilangan bulat $n$ sehingga $\dfrac{2n+6}{n-1}$ adalah bilangan bulat.
    
    \item (OSK 2002) Bilangan asli $n$ terbesar sehingga $8^n \mid 44^{44}$ adalah \dots
    
    \item Berapa banyak pasangan bilangan bulat positif $(a,b)$ yang memenuhi $\dfrac{1}{a}+\dfrac{1}{b}=\dfrac{1}{6}$.
    
    \item Jika $a$ dan $b$ adalah bilangan bulat sedemikian sehingga $a^2-b^2=2017$, maka nilai dari $a^2+b^2$ adalah \dots
    
    \item (AIME 1986) Tentukan bilangan asli $n$ terbesar sehingga $n+10 \mid n^3+100$.

    \item (OSK 2015) Bilangan bulat $x$ jika dikalikan 11 terletak di antara 1500 dan 2000. Jika $x$ dikalikan 7 terletak di antara 970 dan 1275. Jika $x$ dikalikan 5 terletak di antara 690 dan 900. Banyaknya bilangan $x$ sedemikian yang habis dibagi 3 sekaligus habis dibagi 5 ada sebanyak \ldots

    \item (OSK 2023) Banyaknya bilangan 4 digit yang habis dibagi 3 dan memuat angka 6 adalah \ldots
    
    \item (OSN SMP 2003) Buktikan bahwa $(n-1)n(n^3+1)$ selalu habis dibagi 6 untuk semua bilangan asli $n$.
    
\end{enumerate}
\subsection{Latihan Soal Aritmatika Modular}
\begin{enumerate}    
    \item (AIME 1986) Tentukan bilangan asli $n$ terbesar sehingga $n+10 \mid n^3+100$.
    
    \item Tentukan digit satuan dari $7^{7^7}$.
    
    \item Jika $S=1!+2!+3!+\dots+2021!$, tentukan sisa $S$ saat dibagi 6.
    
    \item (OSK 2009) Sisa saat $10^{999999999}$ saat dibagi oleh 7 adalah \dots
    
    \item (OSK 2011) Bilangan asli terkecil $n>2011$ yang bersisa 1 jika dibagi $2,3,4,5,6,7,8,9,10$ adalah \dots.

    \item (OSK 2015) Bilangan $x$ adalah bilangan bulat positif terkecil yang membuat
    \[31^n + x \cdot 96^n\]
    merupakan kelipatan 2015 untuk setiap bilangan asli $n$. Nilai $x$ adalah \ldots

    \item (OSK 2023) Sisa pembagian bilangan $5^{2022}+11^{2022}$ oleh $64$ adalah \ldots

    \item (OSK 2022) Untuk setiap bilangan asli $n$, misalkan $S(n)$ menyatakan hasil penjumlahan semua digit dari $n$. Diberikan barisan $\{a_n\}$ dengan $a_1 = 5$ dan $a_n = (S(a_{n-1}))^2 - 1$ untuk $n \geq 2$. Sisa pembagian $a_1 + a_2 + \cdots + a_{2022}$ oleh $21$ adalah \ldots
    
    \item (OSK 2021) Diketahui dua digit terakhir dari $a^{777}$ adalah $77$, maka dua digit terakhir dari $a$ adalah \ldots

    \item (OSK 2020) Misalkan $n \geq 2$ adalah bilangan asli sedemikian sehingga untuk setiap bilangan asli $a$, $b$ dengan $a + b = n$ berlaku $a^2 + b^2$ merupakan bilangan prima. Hasil penjumlahan semua bilangan asli $n$ semacam itu adalah \ldots

    \item (OSK 2019) Sisa pembagian $1111^{2019}$ oleh $11111$ adalah \ldots

    \item (OSK 2017) Bilangan prima terbesar yang dapat dinyatakan dalam bentuk $a^4+b^4+13$ untuk suatu bilangan-bilangan prima $a$ dan $b$ adalah \ldots

    \item (OSK 2016) Palindrom adalah bilangan yang sama dibaca dari depan atau dari belakang. Sebagai contoh 12321 dan 32223 merupakan palindrom. Palindrom 5 digit terbesar yang habis dibagi 303 adalah \ldots
\end{enumerate}
\subsection{Latihan Soal Aritmatika Modular}
\begin{enumerate}    
    \item (AIME 1986) Tentukan bilangan asli $n$ terbesar sehingga $n+10 \mid n^3+100$.
    
    \item Tentukan digit satuan dari $7^{7^7}$.
    
    \item Jika $S=1!+2!+3!+\dots+2021!$, tentukan sisa $S$ saat dibagi 6.
    
    \item (OSK 2009) Sisa saat $10^{999999999}$ saat dibagi oleh 7 adalah \dots
    
    \item (OSK 2011) Bilangan asli terkecil $n>2011$ yang bersisa 1 jika dibagi $2,3,4,5,6,7,8,9,10$ adalah \dots.

    \item (OSK 2015) Bilangan $x$ adalah bilangan bulat positif terkecil yang membuat
    \[31^n + x \cdot 96^n\]
    merupakan kelipatan 2015 untuk setiap bilangan asli $n$. Nilai $x$ adalah \ldots

    \item (OSK 2023) Sisa pembagian bilangan $5^{2022}+11^{2022}$ oleh $64$ adalah \ldots

    \item (OSK 2022) Untuk setiap bilangan asli $n$, misalkan $S(n)$ menyatakan hasil penjumlahan semua digit dari $n$. Diberikan barisan $\{a_n\}$ dengan $a_1 = 5$ dan $a_n = (S(a_{n-1}))^2 - 1$ untuk $n \geq 2$. Sisa pembagian $a_1 + a_2 + \cdots + a_{2022}$ oleh $21$ adalah \ldots
    
    \item (OSK 2021) Diketahui dua digit terakhir dari $a^{777}$ adalah $77$, maka dua digit terakhir dari $a$ adalah \ldots

    \item (OSK 2020) Misalkan $n \geq 2$ adalah bilangan asli sedemikian sehingga untuk setiap bilangan asli $a$, $b$ dengan $a + b = n$ berlaku $a^2 + b^2$ merupakan bilangan prima. Hasil penjumlahan semua bilangan asli $n$ semacam itu adalah \ldots

    \item (OSK 2019) Sisa pembagian $1111^{2019}$ oleh $11111$ adalah \ldots

    \item (OSK 2017) Bilangan prima terbesar yang dapat dinyatakan dalam bentuk $a^4+b^4+13$ untuk suatu bilangan-bilangan prima $a$ dan $b$ adalah \ldots

    \item (OSK 2016) Palindrom adalah bilangan yang sama dibaca dari depan atau dari belakang. Sebagai contoh 12321 dan 32223 merupakan palindrom. Palindrom 5 digit terbesar yang habis dibagi 303 adalah \ldots
\end{enumerate}
\subsection{Latihan Soal Uji Habis Dibagi}
\begin{enumerate}
    \item Show that a number is divisible by 9 if and only if the sum of its digits is divisible by 9. How about divisibility by 11?
    
    \item (OSK 2010) Nilai $n$ terkecil sehingga $\underbrace{20102010\dots2010}_\text{$n$ buah 2010}$ habis dibagi 99 adalah \dots
    
    \item Jika dihitung maka didapat $17! = 3a56874280b6000$. Tentukan nilai digit $a$ dan $b$.
\end{enumerate}
\subsection{Latihan Soal Uji Habis Dibagi}
\begin{enumerate}
    \item Show that a number is divisible by 9 if and only if the sum of its digits is divisible by 9. How about divisibility by 11?
    
    \item (OSK 2010) Nilai $n$ terkecil sehingga $\underbrace{20102010\dots2010}_\text{$n$ buah 2010}$ habis dibagi 99 adalah \dots
    
    \item Jika dihitung maka didapat $17! = 3a56874280b6000$. Tentukan nilai digit $a$ dan $b$.
\end{enumerate}
\subsection{Latihan Soal Trik Bilangan Prima dan Modulo}
\begin{enumerate}
        \item (OSK 2013) Diketahui $x_1,x_2$ adalah dua bilangan bulat berbeda yang merupakan akar-akar dari persamaan kuadrat $x^2+px+q+1=0$. Jika $p$ dan $p^2+q^2$ adalah bilangan-bilangan prima, maka nilai terbesar yang mungkin dari $x_1^{2013}+x_2^{2013}$ adalah \dots
        
        \item (OSK 2014) Diberikan tiga bilangan bulat positif berurutan. Jika bilangan pertama tetap, bilangan kedua ditambah 10 dan bilangan ketiga ditambah bilangan prima, maka ketiga bilangan ini membentuk deret ukur. Bilangan ketiga dari bilangan bulat berurutan adalah \dots
        
        \item (OSK 2014) Semua pasangan bilangan prima $(p,q)$ yang memenuhi persamaan
        $$(7p-q)^2=2(p-1)q^2$$
        adalah \dots
        
        \item (OSK 2014) Semua bilangan bulat $n$ sehingga $n^4-51n^2+225$ merupakan bilangan prima adalah \dots

        \item (OSK 2015) Banyaknya bilangan asli $n \leq 2015$ yang dapat dinyatakan dalam bentuk $n = a + b$ dengan $a$, $b$ bilangan asli yang memenuhi $a - b$ bilangan prima dan $ab$ bilangan kuadrat sempurna adalah \ldots

        
        \item (OSK 2023) Jika bilangan asli $x$ dan $y$ memenuhi persamaan
        $$x(x-y)=5y-6,$$
        maka $x+y=\ldots$

        \item (OSK 2020) Misalkan $n \geq 2$ adalah bilangan asli sedemikian sehingga untuk setiap bilangan asli $a$, $b$ dengan $a + b = n$ berlaku $a^2 + b^2$ merupakan bilangan prima. Hasil penjumlahan semua bilangan asli $n$ semacam itu adalah \ldots
\end{enumerate}
\subsection{Latihan Soal Trik Bilangan Prima dan Modulo}
\begin{enumerate}
        \item (OSK 2013) Diketahui $x_1,x_2$ adalah dua bilangan bulat berbeda yang merupakan akar-akar dari persamaan kuadrat $x^2+px+q+1=0$. Jika $p$ dan $p^2+q^2$ adalah bilangan-bilangan prima, maka nilai terbesar yang mungkin dari $x_1^{2013}+x_2^{2013}$ adalah \dots
        
        \item (OSK 2014) Diberikan tiga bilangan bulat positif berurutan. Jika bilangan pertama tetap, bilangan kedua ditambah 10 dan bilangan ketiga ditambah bilangan prima, maka ketiga bilangan ini membentuk deret ukur. Bilangan ketiga dari bilangan bulat berurutan adalah \dots
        
        \item (OSK 2014) Semua pasangan bilangan prima $(p,q)$ yang memenuhi persamaan
        $$(7p-q)^2=2(p-1)q^2$$
        adalah \dots
        
        \item (OSK 2014) Semua bilangan bulat $n$ sehingga $n^4-51n^2+225$ merupakan bilangan prima adalah \dots

        \item (OSK 2015) Banyaknya bilangan asli $n \leq 2015$ yang dapat dinyatakan dalam bentuk $n = a + b$ dengan $a$, $b$ bilangan asli yang memenuhi $a - b$ bilangan prima dan $ab$ bilangan kuadrat sempurna adalah \ldots

        
        \item (OSK 2023) Jika bilangan asli $x$ dan $y$ memenuhi persamaan
        $$x(x-y)=5y-6,$$
        maka $x+y=\ldots$

        \item (OSK 2020) Misalkan $n \geq 2$ adalah bilangan asli sedemikian sehingga untuk setiap bilangan asli $a$, $b$ dengan $a + b = n$ berlaku $a^2 + b^2$ merupakan bilangan prima. Hasil penjumlahan semua bilangan asli $n$ semacam itu adalah \ldots
\end{enumerate}
\subsection{Latihan Soal FPB dan KPK}
\begin{enumerate}
    \item (OSK 2011) Bilangan asli terkecil lebih dari 2011 yang bersisa 1 jika dibagi 2,3,4,5,6,7,8,9,10 adalah \dots
    
    \item (OSK 2009) Nilai dari $\sum_{k=1}^{2009} FPB(k,7)$ adalah \dots
    
    \item Banyaknya anggota himpunan himpunan 
    $$S = \{gcd(n^3+1,n^2+3n+9 \mid n \in Z\}$$
    adalah \dots
    
    \item (AIME 1998) Ada berapa banyak bilangan bulat positif $k$ sehingga $lcm(6^6,8^8,k)=12^{12}$?
    
    \item Jika $a,b$ adalah bilangan asli dan $c$ adalah bilangan bulat, buktikan bahwa $gcd(a,b)=gcd(a,b+ac).$

    \item (OSK 2022) Diberikan bilangan asli $m$ dan $n$. Jika $\text{FPB}(m, n) = 7$ dan $\text{FPB}(2m, 3n) = 42$, nilai dari $\text{FPB}(21m, 14n)$ adalah . . . .
\end{enumerate}
\subsection{Latihan Soal FPB dan KPK}
\begin{enumerate}
    \item (OSK 2011) Bilangan asli terkecil lebih dari 2011 yang bersisa 1 jika dibagi 2,3,4,5,6,7,8,9,10 adalah \dots
    
    \item (OSK 2009) Nilai dari $\sum_{k=1}^{2009} FPB(k,7)$ adalah \dots
    
    \item Banyaknya anggota himpunan himpunan 
    $$S = \{gcd(n^3+1,n^2+3n+9 \mid n \in Z\}$$
    adalah \dots
    
    \item (AIME 1998) Ada berapa banyak bilangan bulat positif $k$ sehingga $lcm(6^6,8^8,k)=12^{12}$?
    
    \item Jika $a,b$ adalah bilangan asli dan $c$ adalah bilangan bulat, buktikan bahwa $gcd(a,b)=gcd(a,b+ac).$

    \item (OSK 2022) Diberikan bilangan asli $m$ dan $n$. Jika $\text{FPB}(m, n) = 7$ dan $\text{FPB}(2m, 3n) = 42$, nilai dari $\text{FPB}(21m, 14n)$ adalah . . . .
\end{enumerate}
\subsection{Fungsi yang Melibatkan Faktor Bilangan}
Misalkan $a$ dapat difaktorisasi prima seperti sebelumnya yaitu $a=p_1^{a_1}\cdot p_2^{a_2}\cdot \ldots \cdot p_n^{a_n}$.
\subsubsection{Banyaknya Faktor Positif}
Fungsi $d(a)$ didefinisikan sebagai banyaknya faktor atau pembagi positif dari $a$ dengan
$$d(a) = (a_1+1)(a_2+1)\cdot \ldots \cdot (a_n+1).$$

Contoh: Banyaknya faktor positif dari $12= 2^2 \cdot 3^1$ adalah $d(12)=(2+1)(1+1)=6$ dengan pembagi positifnya adalah $1,2,3,4,6,12$ (ada 6 faktor positif.)
\subsubsection{Jumlah Faktor Positif}
Fungsi $\sigma (a)$ didefinisikan sebagai banyaknya faktor atau pembagi positif dari $a$ dengan
$$\sigma (a) = (p_1^0+p_1^1+p_1^2+\dots+p_1^{a_1})(p_2^0+p_2^1+p_2^2+\dots+p_2^{a_2})\cdot \ldots \cdot (p_n^0+p_n^1+p_n^2+\dots+p_n^{a_n}).$$

Contoh: Jumlah faktor atau pembagi positif dari $12= 2^2 \cdot 3^1$ adalah $d(12)=(2^0+2^1+2^2)(3^0+3^1)=28$ yang setara dengan penjumlahan secara manualnya, yaitu $2^03^0+2^03^1+2^13^0+2^13^1+2^23^0+2^23^1=28.$

\subsubsection{Banyaknya Bilangan Relatif Prima}
\begin{remark*}
         Bilangan $b$ dikatakan relatif prima dengan $a$ jika dan hanya jika $FPB(a,b)=1$.
    \end{remark*}
 Definisikan fungsi \textbf{Euler Totient Phi $\phi(n)$ sebagai banyaknya bilangan bulat positif $b$ yang kurang dari sama dengan $n$ dimana $n$ relatif prima dengan $b$}. Rumus eksplisit untuk menghitung fungsi ini adalah
$$\phi(n) = n\left(1-\dfrac{1}{p_1}\right)\left(1-\dfrac{1}{p_2}\right)\dots\left(1-\dfrac{1}{p_n}\right).$$

Contoh: $\phi(4)=4(1-\frac{1}{2})=2$ karena ada 2 bilangan yang realtif prima dengan 4, yaitu 1 dan 3.

Catatan: Untuk semua bilangan prima $p$, nilai $\phi(p) = p-1$. (Silakan dibuktikan sendiri :D)


\subsection{Latihan Soal Fungsi Faktor Bilangan}
\begin{enumerate}
    \item (OSK 2015) Banyaknya faktor bulat positif dari 2015 adalah \ldots

    \item (OSP 2009) Misalkan $n$ bilangan asli terkecil yang mempunyai tepat 2009 faktor dan $n$ merupakan kelipatan 2009. Faktor prima terkecil dari $n$ adalah \dots
    
    \item Misalkan $n$ bilangan asli dimana $2n$ mempunyai 28 faktor positif dan $3n$ mempunyai 30 faktor positif. Banyaknya faktor positif yang dimilik $6n$ adalah \dots
    
    \item (OSK 2011) Ada berapa faktor positif dari $2^73^55^37^2$ yang merupakan kelipatan 10?
    
    \item (AIME 1995) Tentukan banyaknya faktor positif dari $n^2$ yang kurang dari $n$ tetapi tidak membagi $n$ jika $n={2^{31}}3^{19}.$
    
    \item (AIME 2000) Tentukan bilangan asli terkecil yang memiliki 12 faktor positif genap dan $6$ faktor positif ganjil.

    \item Berapa banyak bilangan di himpunan $\{1,2,\dots,200\}$ yang relatif prima dengan 100?

    \item (OSK 2023) Misalkan $n=2^a3^b$ dengan $a$ dan $b$ bilangan asli. Jika hasil kali semua faktor positif dari $n$ adalah $12^{90}$, maka nilai $ab = \ldots$ 

    \item (OSK 2016) Banyaknya bilangan asli $n$ yang memenuhi sifat hasil jumlah $n$ dan suatu pembagi positif $n$  yang kurang dari sama dengan 2016 adalah \dots

    \item (OSK 2017) Misalkan $s(n)$ menyatakan faktor prima $t(n)$ menyatakan faktor prima terkecil dari $n$. Banyaknya bilangan asli $n \in \{1,2,\dots,100\}$ sehingga $t(n)+1=s(n)$ adalah \dots
\end{enumerate}
\subsection{Euler's Theorem on Modulo}
Untuk bilangan asli $a$ dan $n$ yang saling relatif prima kita punya
$$a^{\phi(n)} \equiv 1 \mod n.$$

\subsection{Fermat's Little Theorem}
Teorema ini merupakan kasus khusus dari Euler's Theorem saat $n$ prima sehingga $\phi(n)=n-1$. Untuk bilangan prima $p$, kita punya
$$a^{p-1} \equiv 1 \mod p.$$

\subsection{Fermat's Little Theorem}
Teorema ini merupakan kasus khusus dari Euler's Theorem saat $n$ prima sehingga $\phi(n)=n-1$. Untuk bilangan prima $p$, kita punya
$$a^{p-1} \equiv 1 \mod p.$$
\subsection{Wilson's Theorem}
Untuk suatu bilangan asli $p$, kita punya $p$ adalah bilangan prima jika dan hanya jika
$$(p-1)! \equiv -1 \mod p.$$


\subsection{Latihan Soal Euler's Theorem dan Fermat's Little Theorem}
\begin{enumerate}
    \item Berapakah sisa pembagian $43^{43^{43}}$ oleh 100?
    
    \item Jika $10^{999999999}$ dibagi oleh 7, maka sisanya adalah \dots
    
    \item Tentukan sisa saat $2^{70}+3^{70}$ saat dibagi 13.
\end{enumerate}
\subsection{Inverse Modulo}
Misalkan bilangan bulat $a$, $x$ dan bilangan bulat positif $m$. Kita sebut $x$ adalah inverse dari $a \mod m$ jika dan hanya jika $gcd(a,m)=1$ dan $ax \equiv 1 \mod m$.
\subsection{Basis Bilangan}
Basis bilangan adalah sistem bilangan yang menyatakan banyaknya digit atau kombinasi dari digit-digit yang menyatakan sebuah bilangan. Secara matematis, bilangan $a$ dalam basis $n > 0$ yaitu $(a)_n$ mempunyai bentuk (yang setara dengan nilai basis 10):
$$(c_kc_{k-1}\dotsc_1c_0)_n = c_{k}n^k + c_{k-1}n^{k-1}+\dots+c_1n^{1}+c_0n^{0}$$

Secara umum bahkan kita telah memakai sistem basis tersebut untuk basis 10. Misalkan 123 dapat dinyatakan sebagai $123 = 1\cdot 10^2 + 2\cdot 10^1 + 1\cdot 10^0$

Lalu, berikut merupakan contoh untuk bilangan basis selain 10 misalnya: 
\begin{itemize}
    \item Bilangan basis 2 atau bilangan biner yang digit-digitnya terdiri dari $\{0,1\}$. Misalkan $1001_2$ dalam biner yang setara dengan $9$ atau $1001_2 = 9$ karena $1001_2 = 1\cdot 2^3+0\cdot 2^2+0\cdot 2^1+1\cdot 2^0 = 9$. 
    \item Bilangan basis 3 yang digit-digitnya terdiri dari $\{0,1,2\}$. Misalkan $211_3 = 22$ karena $211_3 = 2\cdot 3^2+ 1\cdot 3^1+ 1\cdot 3^0 = 22$.
    \item Bilangan basis 16 atau heksadesimal yang digit-digitnya terdiri dari $\{0,1,2,\dots,9,A,B,\dots,F\}$. Misalkan $5F_{16} = 95$ karena $5F_{16} = 5 \cdot 16^1 + (15)\cdot 16^0 = 95$.
\end{itemize}

\subsection{Basis Bilangan}
Basis bilangan adalah sistem bilangan yang menyatakan banyaknya digit atau kombinasi dari digit-digit yang menyatakan sebuah bilangan. Secara matematis, bilangan $a$ dalam basis $n > 0$ yaitu $(a)_n$ mempunyai bentuk (yang setara dengan nilai basis 10):
$$(c_kc_{k-1}\dotsc_1c_0)_n = c_{k}n^k + c_{k-1}n^{k-1}+\dots+c_1n^{1}+c_0n^{0}$$

Secara umum bahkan kita telah memakai sistem basis tersebut untuk basis 10. Misalkan 123 dapat dinyatakan sebagai $123 = 1\cdot 10^2 + 2\cdot 10^1 + 1\cdot 10^0$

Lalu, berikut merupakan contoh untuk bilangan basis selain 10 misalnya: 
\begin{itemize}
    \item Bilangan basis 2 atau bilangan biner yang digit-digitnya terdiri dari $\{0,1\}$. Misalkan $1001_2$ dalam biner yang setara dengan $9$ atau $1001_2 = 9$ karena $1001_2 = 1\cdot 2^3+0\cdot 2^2+0\cdot 2^1+1\cdot 2^0 = 9$. 
    \item Bilangan basis 3 yang digit-digitnya terdiri dari $\{0,1,2\}$. Misalkan $211_3 = 22$ karena $211_3 = 2\cdot 3^2+ 1\cdot 3^1+ 1\cdot 3^0 = 22$.
    \item Bilangan basis 16 atau heksadesimal yang digit-digitnya terdiri dari $\{0,1,2,\dots,9,A,B,\dots,F\}$. Misalkan $5F_{16} = 95$ karena $5F_{16} = 5 \cdot 16^1 + (15)\cdot 16^0 = 95$.
\end{itemize}

\subsection{Basis Bilangan}
Basis bilangan adalah sistem bilangan yang menyatakan banyaknya digit atau kombinasi dari digit-digit yang menyatakan sebuah bilangan. Secara matematis, bilangan $a$ dalam basis $n > 0$ yaitu $(a)_n$ mempunyai bentuk (yang setara dengan nilai basis 10):
$$(c_kc_{k-1}\dotsc_1c_0)_n = c_{k}n^k + c_{k-1}n^{k-1}+\dots+c_1n^{1}+c_0n^{0}$$

Secara umum bahkan kita telah memakai sistem basis tersebut untuk basis 10. Misalkan 123 dapat dinyatakan sebagai $123 = 1\cdot 10^2 + 2\cdot 10^1 + 1\cdot 10^0$

Lalu, berikut merupakan contoh untuk bilangan basis selain 10 misalnya: 
\begin{itemize}
    \item Bilangan basis 2 atau bilangan biner yang digit-digitnya terdiri dari $\{0,1\}$. Misalkan $1001_2$ dalam biner yang setara dengan $9$ atau $1001_2 = 9$ karena $1001_2 = 1\cdot 2^3+0\cdot 2^2+0\cdot 2^1+1\cdot 2^0 = 9$. 
    \item Bilangan basis 3 yang digit-digitnya terdiri dari $\{0,1,2\}$. Misalkan $211_3 = 22$ karena $211_3 = 2\cdot 3^2+ 1\cdot 3^1+ 1\cdot 3^0 = 22$.
    \item Bilangan basis 16 atau heksadesimal yang digit-digitnya terdiri dari $\{0,1,2,\dots,9,A,B,\dots,F\}$. Misalkan $5F_{16} = 95$ karena $5F_{16} = 5 \cdot 16^1 + (15)\cdot 16^0 = 95$.
\end{itemize}

\input{Soal/NumberTheory/basisBilangan}
\subsection{Basis Bilangan}
Basis bilangan adalah sistem bilangan yang menyatakan banyaknya digit atau kombinasi dari digit-digit yang menyatakan sebuah bilangan. Secara matematis, bilangan $a$ dalam basis $n > 0$ yaitu $(a)_n$ mempunyai bentuk (yang setara dengan nilai basis 10):
$$(c_kc_{k-1}\dotsc_1c_0)_n = c_{k}n^k + c_{k-1}n^{k-1}+\dots+c_1n^{1}+c_0n^{0}$$

Secara umum bahkan kita telah memakai sistem basis tersebut untuk basis 10. Misalkan 123 dapat dinyatakan sebagai $123 = 1\cdot 10^2 + 2\cdot 10^1 + 1\cdot 10^0$

Lalu, berikut merupakan contoh untuk bilangan basis selain 10 misalnya: 
\begin{itemize}
    \item Bilangan basis 2 atau bilangan biner yang digit-digitnya terdiri dari $\{0,1\}$. Misalkan $1001_2$ dalam biner yang setara dengan $9$ atau $1001_2 = 9$ karena $1001_2 = 1\cdot 2^3+0\cdot 2^2+0\cdot 2^1+1\cdot 2^0 = 9$. 
    \item Bilangan basis 3 yang digit-digitnya terdiri dari $\{0,1,2\}$. Misalkan $211_3 = 22$ karena $211_3 = 2\cdot 3^2+ 1\cdot 3^1+ 1\cdot 3^0 = 22$.
    \item Bilangan basis 16 atau heksadesimal yang digit-digitnya terdiri dari $\{0,1,2,\dots,9,A,B,\dots,F\}$. Misalkan $5F_{16} = 95$ karena $5F_{16} = 5 \cdot 16^1 + (15)\cdot 16^0 = 95$.
\end{itemize}

\subsection{Basis Bilangan}
Basis bilangan adalah sistem bilangan yang menyatakan banyaknya digit atau kombinasi dari digit-digit yang menyatakan sebuah bilangan. Secara matematis, bilangan $a$ dalam basis $n > 0$ yaitu $(a)_n$ mempunyai bentuk (yang setara dengan nilai basis 10):
$$(c_kc_{k-1}\dotsc_1c_0)_n = c_{k}n^k + c_{k-1}n^{k-1}+\dots+c_1n^{1}+c_0n^{0}$$

Secara umum bahkan kita telah memakai sistem basis tersebut untuk basis 10. Misalkan 123 dapat dinyatakan sebagai $123 = 1\cdot 10^2 + 2\cdot 10^1 + 1\cdot 10^0$

Lalu, berikut merupakan contoh untuk bilangan basis selain 10 misalnya: 
\begin{itemize}
    \item Bilangan basis 2 atau bilangan biner yang digit-digitnya terdiri dari $\{0,1\}$. Misalkan $1001_2$ dalam biner yang setara dengan $9$ atau $1001_2 = 9$ karena $1001_2 = 1\cdot 2^3+0\cdot 2^2+0\cdot 2^1+1\cdot 2^0 = 9$. 
    \item Bilangan basis 3 yang digit-digitnya terdiri dari $\{0,1,2\}$. Misalkan $211_3 = 22$ karena $211_3 = 2\cdot 3^2+ 1\cdot 3^1+ 1\cdot 3^0 = 22$.
    \item Bilangan basis 16 atau heksadesimal yang digit-digitnya terdiri dari $\{0,1,2,\dots,9,A,B,\dots,F\}$. Misalkan $5F_{16} = 95$ karena $5F_{16} = 5 \cdot 16^1 + (15)\cdot 16^0 = 95$.
\end{itemize}

\subsection{Basis Bilangan}
Basis bilangan adalah sistem bilangan yang menyatakan banyaknya digit atau kombinasi dari digit-digit yang menyatakan sebuah bilangan. Secara matematis, bilangan $a$ dalam basis $n > 0$ yaitu $(a)_n$ mempunyai bentuk (yang setara dengan nilai basis 10):
$$(c_kc_{k-1}\dotsc_1c_0)_n = c_{k}n^k + c_{k-1}n^{k-1}+\dots+c_1n^{1}+c_0n^{0}$$

Secara umum bahkan kita telah memakai sistem basis tersebut untuk basis 10. Misalkan 123 dapat dinyatakan sebagai $123 = 1\cdot 10^2 + 2\cdot 10^1 + 1\cdot 10^0$

Lalu, berikut merupakan contoh untuk bilangan basis selain 10 misalnya: 
\begin{itemize}
    \item Bilangan basis 2 atau bilangan biner yang digit-digitnya terdiri dari $\{0,1\}$. Misalkan $1001_2$ dalam biner yang setara dengan $9$ atau $1001_2 = 9$ karena $1001_2 = 1\cdot 2^3+0\cdot 2^2+0\cdot 2^1+1\cdot 2^0 = 9$. 
    \item Bilangan basis 3 yang digit-digitnya terdiri dari $\{0,1,2\}$. Misalkan $211_3 = 22$ karena $211_3 = 2\cdot 3^2+ 1\cdot 3^1+ 1\cdot 3^0 = 22$.
    \item Bilangan basis 16 atau heksadesimal yang digit-digitnya terdiri dari $\{0,1,2,\dots,9,A,B,\dots,F\}$. Misalkan $5F_{16} = 95$ karena $5F_{16} = 5 \cdot 16^1 + (15)\cdot 16^0 = 95$.
\end{itemize}

\input{Soal/NumberTheory/basisBilangan}