\subsection{Latihan Soal Aritmatika Modular}
\begin{enumerate}    
    \item (AIME 1986) Tentukan bilangan asli $n$ terbesar sehingga $n+10 \mid n^3+100$.
    
    \item Tentukan digit satuan dari $7^{7^7}$.
    
    \item Jika $S=1!+2!+3!+\dots+2021!$, tentukan sisa $S$ saat dibagi 6.
    
    \item (OSK 2009) Sisa saat $10^{999999999}$ saat dibagi oleh 7 adalah \dots
    
    \item (OSK 2011) Bilangan asli terkecil $n>2011$ yang bersisa 1 jika dibagi $2,3,4,5,6,7,8,9,10$ adalah \dots.

    \item (OSK 2015) Bilangan $x$ adalah bilangan bulat positif terkecil yang membuat
    \[31^n + x \cdot 96^n\]
    merupakan kelipatan 2015 untuk setiap bilangan asli $n$. Nilai $x$ adalah \ldots
\end{enumerate}