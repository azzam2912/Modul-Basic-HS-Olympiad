\subsection{Latihan Soal Pemfaktoran dan Manipulasi Aljabar}
\begin{enumerate}
    \item  Nilai dari $\sqrt{5050^2-4950^2}$ adalah \dots

    \item (OSP 2008) Jika $0 < b < a$ dan $a^2+b^2=6ab$, maka nilai $\dfrac{a+b}{a-b}=\dots$
    
    \item Jika $x > 0$ dan $x + \dfrac{1}{x} =  5$, maka nilai $x^3+\dfrac{1}{x^3}$ adalah \dots
    
    \item (OSK 2017) Diketahui $x-y=10$ dan $xy=10$. Nilai $x^4+y^4$ adalah \dots
    
    \item (OSK 2018) Diketahui $x$ dan $y$ bilangan prima dengan $x < y$, dan $x^3+y^3+2018=30y^2-300y+3018$. Nilai $x$ yang memenuhi adalah \dots
    
    \item Jika $a+b+c=0$ untuk suatu bilangan riil $a,b,c$, buktikan bahwa $a^3+b^3+c^3=3abc$.
    
    \item Jika $x=2021^3-2019^3$, maka nilai $\sqrt{\dfrac{x-2}{6}}$ adalah \dots

    \item (AIME 1987)
    Tentukan nilai sederhana dari $\dfrac{(10^4+324)(22^4+324)(34^4+324)(46^4+324)(58^4+324)}{(4^4+324)(16^4+324)(28^4+324)(40^4+324)(52^4+324)}$

    \item (OSK 2017) Jika $\dfrac{(a-b)(c-d)}{(b-c)(d-a)}=-\dfrac{4}{7}$, maka nilai dari $\dfrac{(a-c)(b-d)}{(a-b)(c-d)}$ adalah \dots

    \item (OSK 2019) Diketahui $a+2b=1$, $b+2c=2$, dan $b \neq 0$. Jika $a+nb+2018c = 2019$ maka nilai $n$ adalah \dots

    \item (OSK 2019) Misalkan $a = 2\sqrt{2} - \sqrt{8-4\sqrt{2}}$ dan $b = 2\sqrt{2} + \sqrt{8-4\sqrt{2}}$. Jika $\dfrac{a}{b}+\dfrac{b}{a}=x+y\sqrt{2}$ dengan $x,y$ bulat, maka nilai $x+y$ adalah \dots

    \item (OSK 2015) Diketahui bilangan real positif $a$ dan $b$ memenuhi persamaan
    \begin{align*}
        a^4+a^2b^2+b^4=6 \text{ dan } a^2+ab+b^2=4
    \end{align*}
    Nilai dari $a + b$ adalah \ldots

    \item (OSK 2022) Diketahui $a,b,c,d$ bilangan real positif yang memenuhi $a>c$, $d>b$, dan 
    $$3a^2+3b^2=3c^2+3d^2=4ac+4bd.$$
    Nilai $\dfrac{12(ab+cd)}{ad+bc}=\dots$ 
\end{enumerate}