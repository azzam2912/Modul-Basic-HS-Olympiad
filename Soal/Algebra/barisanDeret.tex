\subsection{Latihan Soal Barisan dan Deret}
\begin{enumerate}
\item (OSK 2006) Diketahui $a+(a+1)+(a+2)+\dots+50=1139$. Jika $a$ bilangan real positif, maka $a=\dots$

\item (AIME 1984) Barisan $a_1,a_2,\dots,a_{98}$ memenuhi $a_{n+1}=a_n+1$ untuk $n=1,2,\dots,97$ dan mempunyai jumlah sama dengan $137$. Tentukan nilai dari $a_2+a_4+a_6+\dots+a_{98}$.

\item (AIME 2003) Diketahui $0<a<b<c<d$ adalah bilangan bulat dimana $a,b,c$ membentuk barisan aritmatika sedangkan $b,c,d$ membentuk barisan geometri. Jika $d-a=30$ maka tentukan nilai dari $a+b+c+d$.

\item (OSK 2009) Bilangan bulat positif terkecil $n$ dengan $n> 2009$ sehingga $$\sqrt{\dfrac{1^3+2^3+3^3+\dots+n^3}{n}}$$
merupakan bilangan bulat adalah \dots

\item (OSK 2015) Diketahui barisan bilangan real $a_1$, $a_2$, $a_3$, $\dots$, $a_n$, $\dots$ merupakan barisan geometri. Jika $a_1+a_4 = 20$, maka nilai minimal dari
\[a_1 + a_2 + a_3 + a_4 + a_5 + a_6\]
adalah \ldots

\item (OSK 2022) Jika $\sum_{k=1}^{\infty} \frac{2k + B}{3^{k+1}} = 10$, maka $B = \ldots $.

\item Tentukan nilai paling sederhana dari $\sqrt{2\sqrt{2\sqrt{2\sqrt{\dots}}}}$.

\item Tentukan nilai dari $\dfrac{1}{1\cdot2}+\dfrac{1}{2\cdot 3}+\dfrac{1}{3 \cdot 4}+\dots+\dfrac{1}{2020 \cdot 2021}.$

\item  Nilai paling sederhana dari $\left(1-\dfrac{1}{2^2}\right)\cdot\left(1-\dfrac{1}{3^2}\right)\cdot\left(1-\dfrac{1}{4^2}\right)\cdot\dots\cdot\left(1-\dfrac{1}{2021^2}\right)$ adalah \dots

\item Tentukan hasil dari jumlah $\dfrac{1}{\sqrt{1}+\sqrt{2}}+\dfrac{1}{\sqrt{2}+\sqrt{3}}+\dots+\dfrac{1}{\sqrt{99}+\sqrt{100}}.$

\item Nilai $x$ yang memenuhi persamaan
$$\sqrt{x\sqrt{x\sqrt{x\sqrt{\dots}}}}=\sqrt{4x+\sqrt{4x+\sqrt{4x+\sqrt{\dots}}}}$$
adalah \dots

\item Hitunglah nilai paling sederhana dari
$$6-\dfrac{5}{3+\dfrac{4}{3+\dfrac{4}{3+\dfrac{4}{3+\dfrac{4}{\dots}}}}}$$
\end{enumerate}