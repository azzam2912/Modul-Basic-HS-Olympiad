\subsection{Latihan Soal Lingkaran}
\begin{enumerate}
    \item Pada segiempat $WXYZ$ dengan diagonal yang saling tegak lurus diketahui bahwa $\angle WZX = 30^\circ, \angle XWY = 40^\circ,$ and $\angle WYZ = 50^\circ$. Hitunglah besar $\angle X$ dan $\angle Z$.
    \begin{center}
        \includegraphics[width=0.25\textwidth]{Soal/Geometry/evanquad.PNG}
    \end{center}
		
    \item (OSK 2013) Diberikan segitiga lancip $ABC$ dengan $O$ sebagai pusat lingkaran luarnya. Misalkan $M$ dan $N$ berturut - turut pertengahan $OA$ dan $BC$. Jika $\angle ABC = 4\angle OMN$ dan $\angle ACB = 6\angle OMN$, maka besarnya $\angle OMN$ sama dengan \dots

    \item 	Pada gambar di bawah, diketahui titik A $\ne$ B pada lingkaran berdiameter $MN$ dan berpusat di $C$. $P$ adalah titik pada segmen $CN$ dimana $\angle CAP = \angle CBP = 10 ^\circ$. Jika $\angle ACM = 40^\circ$, maka $\angle BCN = \dots^\circ$	
    \begin{center}
         \includegraphics[width=0.25\textwidth]{Soal/Geometry/soalLingkaran1.PNG}
    \end{center}

    \item (OSK 2011,2012,2013,2018) Diberikan segitiga $ABC$ dan lingkaran $\Gamma$ yang berdiameter $AB$. Lingkaran $\Gamma$ memotong sisi $AC$ dan $BC$ berturut-turut di titik $D$ dan $E$. Jika $AD = \frac13 AC, BE =\frac14 BC$ dan $AB = 30$, maka luas segitiga $ABC$ adalah \dots

    \item (OSK 2015) Diberikan segitiga $ABC$ dengan sudut $\angle ABC = 90^\circ$. Lingkaran $L_1$ dengan $AB$ sebagai diameter sedangkan lingkaran $L_2$ dengan $BC$ sebagai diameternya. Kedua lingkaran $L_1$ dan $L_2$ berpotongan di $B$ dan $P$. Jika $AB = 5$, $BC = 12$ dan $BP = x$ maka nilai dari $\frac{240}{x}$ adalah \ldots
\end{enumerate}