\subsection{Latihan Soal Permutasi}
\begin{enumerate} 
    \item Sembilan orang siswa akan duduk pada 5 kursi sejajar. Ada berapa cara susunan mereka ? 
    
    \item Denny akan membentuk bilangan genap 3 angka yang angka-angkanya diambil dari 2, 3, 4, 5, 6, 7, 8. Berapa banyak bilangan yang dapat dibentuk jika : 
    \begin{enumerate}
        \item angka-angkanya boleh berulang 
        \item angka-angkanya tidak boleh berulang
    \end{enumerate}
    
    \item (OSP 2003) Empat pasang suami istri menonton pagelaran orkestra. Tempat duduk mereka harus dipisah antara kelompok suami dan kelompok istri. Untuk masing-masing kelompok disediakan 4 buah tempat duduk bersebelahan dalam satu barisan. Ada berapa banyak cara memberikan tempat duduk kepada mereka ?

    \item Di suatu ruangan terdapat 12 kursi yang disusun menjadi 3 baris. Di baris pertama, terdapat 3 kursi. Di baris kedua, terdapat 4 kursi. Di baris ketiga, terdapat 5 kursi. Jika kursi akan diduduki oleh 12 siswa termasuk Aska dan Budi. Misal banyaknya cara untuk 12 siswa menempati tempat duduk jika Aska dan Budi ada di baris pertama adalah $A$. Nilai dari $\frac{A}{8!}$ adalah \ldots

\end{enumerate}