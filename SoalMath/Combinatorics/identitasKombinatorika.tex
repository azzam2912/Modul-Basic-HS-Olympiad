\subsection{Latihan Soal Identitas Kombinatorika}
\begin{enumerate}
    \item (AIME II 2000) Given that
            $\frac 1{2!17!}+\frac 1{3!16!}+\frac 1{4!15!}+\frac 1{5!14!}+\frac 1{6!13!}+\frac 1{7!12!}+\frac 1{8!11!}+\frac 1{9!10!}=\frac N{1!18!}$
            find the greatest integer that is less than $\frac N{100}$. 

        \item (AIME 1986) The polynomial $1-x+x^2-x^3+\cdots+x^{16}-x^{17}$ may be written in the form $a_0+a_1y+a_2y^2+\cdots +a_{16}y^{16}+a_{17}y^{17}$, where $y=x+1$ and the $a_i$'s are constants. Find the value of $a_2$.

        \item (AMC 10A 2016) For some particular value of $N$, when $(a+b+c+d+1)^N$ is expanded and like terms are combined, the resulting expression contains exactly $1001$ terms that include all four variables $a, b,c,$ and $d$, each to some positive power. What is $N$?
\end{enumerate}