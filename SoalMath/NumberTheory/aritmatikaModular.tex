\subsection{Latihan Soal Aritmatika Modular}
\begin{enumerate}    
    \item (AIME 1986) Tentukan bilangan asli $n$ terbesar sehingga $n+10 \mid n^3+100$.
    
    \item Tentukan digit satuan dari $7^{7^7}$.
    
    \item Jika $S=1!+2!+3!+\dots+2021!$, tentukan sisa $S$ saat dibagi 6.
    
    \item (OSK 2009) Sisa saat $10^{999999999}$ saat dibagi oleh 7 adalah \dots
    
    \item (OSK 2011) Bilangan asli terkecil $n>2011$ yang bersisa 1 jika dibagi $2,3,4,5,6,7,8,9,10$ adalah \dots.

    \item (OSK 2015) Bilangan $x$ adalah bilangan bulat positif terkecil yang membuat
    \[31^n + x \cdot 96^n\]
    merupakan kelipatan 2015 untuk setiap bilangan asli $n$. Nilai $x$ adalah \ldots

    \item (OSK 2023) Sisa pembagian bilangan $5^{2022}+11^{2022}$ oleh $64$ adalah \ldots

    \item (OSK 2022) Untuk setiap bilangan asli $n$, misalkan $S(n)$ menyatakan hasil penjumlahan semua digit dari $n$. Diberikan barisan $\{a_n\}$ dengan $a_1 = 5$ dan $a_n = (S(a_{n-1}))^2 - 1$ untuk $n \geq 2$. Sisa pembagian $a_1 + a_2 + \cdots + a_{2022}$ oleh $21$ adalah \ldots
    
    \item (OSK 2021) Diketahui dua digit terakhir dari $a^{777}$ adalah $77$, maka dua digit terakhir dari $a$ adalah \ldots

    \item (OSK 2020) Misalkan $n \geq 2$ adalah bilangan asli sedemikian sehingga untuk setiap bilangan asli $a$, $b$ dengan $a + b = n$ berlaku $a^2 + b^2$ merupakan bilangan prima. Hasil penjumlahan semua bilangan asli $n$ semacam itu adalah \ldots

    \item (OSK 2019) Sisa pembagian $1111^{2019}$ oleh $11111$ adalah \ldots

    \item (OSK 2017) Bilangan prima terbesar yang dapat dinyatakan dalam bentuk $a^4+b^4+13$ untuk suatu bilangan-bilangan prima $a$ dan $b$ adalah \ldots

    \item (OSK 2016) Palindrom adalah bilangan yang sama dibaca dari depan atau dari belakang. Sebagai contoh 12321 dan 32223 merupakan palindrom. Palindrom 5 digit terbesar yang habis dibagi 303 adalah \ldots
\end{enumerate}