\subsection{Latihan Soal Keterbagian}
\begin{enumerate}    
    \item Carilah semua bilangan bulat $n$ sehingga $\dfrac{2n+6}{n-1}$ adalah bilangan bulat.
    
    \item (OSK 2002) Bilangan asli $n$ terbesar sehingga $8^n \mid 44^{44}$ adalah \dots
    
    \item Berapa banyak pasangan bilangan bulat positif $(a,b)$ yang memenuhi $\dfrac{1}{a}+\dfrac{1}{b}=\dfrac{1}{6}$.
    
    \item Jika $a$ dan $b$ adalah bilangan bulat sedemikian sehingga $a^2-b^2=2017$, maka nilai dari $a^2+b^2$ adalah \dots
    
    \item (AIME 1986) Tentukan bilangan asli $n$ terbesar sehingga $n+10 \mid n^3+100$.

    \item (OSK 2015) Bilangan bulat $x$ jika dikalikan 11 terletak di antara 1500 dan 2000. Jika $x$ dikalikan 7 terletak di antara 970 dan 1275. Jika $x$ dikalikan 5 terletak di antara 690 dan 900. Banyaknya bilangan $x$ sedemikian yang habis dibagi 3 sekaligus habis dibagi 5 ada sebanyak \ldots

    \item (OSK 2023) Banyaknya bilangan 4 digit yang habis dibagi 3 dan memuat angka 6 adalah \ldots
    
    \item (OSN SMP 2003) Buktikan bahwa $(n-1)n(n^3+1)$ selalu habis dibagi 6 untuk semua bilangan asli $n$.
    
\end{enumerate}