\subsection{Latihan Soal Segitiga }
\begin{enumerate}    
    \item Garis berat $AD$ pada segitiga $ABC$ memotong garis berat $CF$ di titik $P$, serta perpanjangan $BP$ memotong $AC$ di $E$. Jika diketahui segitiga $ABC$ lancip dan $AB=6$, maka panjang $DE$ adalah \dots

    \item (OSK 2011,2012,2013,2018) Diberikan segitiga $ABC$ dan lingkaran $\Gamma$ yang berdiameter $AB$. Lingkaran $\Gamma$ memotong sisi $AC$ dan $BC$ berturut-turut di titik $D$ dan $E$. Jika $AD = \frac13 AC, BE =\frac14 BC$ dan $AB = 30$, maka luas segitiga $ABC$ adalah \dots
		
    \item Diberikan segitiga $ABC$ dengan $D$ titik tengah $AC$, $E$ titik tengah $BD$, dan $H$ merupakan pencerminan $A$ terhadap $E$. Jika $F$ merupakan perpotongan antara $AH$ dengan $BC$, maka nilai $\dfrac{AF}{FH}$ sama dengan \dots
		 
    \item Diberikan segitiga $ABC$ dengan panjang sisi $BC = 20$, $CA = 24$, dan $AB=12$. Titik $D$ pada segmen $BC$ dengan $BD = 5$. Lingkaran luar dari segitiga $ABD$ memotong $CA$ di $E$. Hitunglah nilai $2 \times DE$.

    \item (OSK 2015) Diberikan trapesium $ABCD$ dengan $AB$ sejajar $DC$ dan $AB = 84$ serta $DC = 25$. Jika trapesium $ABCD$ memiliki lingkaran dalam yang menyinggung keempat sisinya, keliling trapesium $ABCD$ adalah \ldots

    \item (OSK 2022) Diberikan segitiga siku-siku $ABC$. Jika luas dari segitiga $ABC$ adalah 112. Misalkan $R$ adalah panjang jari-jari lingkaran luar segitiga $ABC$ dan $r$ adalah panjang jari-jari lingkaran dalam segitiga $ABC$. Diketahui juga $R + r = 16$. Panjang sisi miring dari segitiga $ABC$ adalah \ldots
\end{enumerate}