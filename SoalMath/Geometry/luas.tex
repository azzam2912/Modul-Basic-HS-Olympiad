\subsection{Latihan Soal Luas}
\begin{enumerate}
\item (OSK 2006) Pada segitiga $ABC$, titik $F$ membagi sisi $AC$ dalam perbandingan $1 : 2$. Misalkan $G$ titik tengah $BF$ dan $E$ titik perpotongan antara sisi $BC$ dengan $AG$. Maka titik $E$ membagi sisi $BC$ dalam perbandingan \dots

\item (OSK 2009) Diberikan persegi $ABCD$ dengan panjang sisi 10. Misalkan $E$ pada $AB$ dan $F$ pada $BC$ dengan $AE = FB = 5$. Misalkan $P$ adalah titik potong $DE$ dan $AF$. Luas $DCFP$ adalah \dots

\item Pada segitiga $ABC$, titik $D$ dan $E$ berturut-turut berada di segmen $AB$ dan $AC$. Misalkan $CD$ dan $BE$ berpotongan di titik $F$. Jika $[DBF]=3$, $[BFC]=6$, dan $[CFE]=4$, berapakah luas $ADFE$?

\item (OSK 2016) Pada segitiga ABC, titik $X, Y$ dan $Z$ berturut-turut terletak pada sinar $BA, CB$ dan $AC$ sehingga $BX = 2BA, CY = 2CB$ dan $AZ = 2AC$. Jika luas segitiga $ABC$ adalah 1, maka luas segitiga $XYZ$ adalah \dots

\item (OSK 2016) Segitiga $ABC$ merupakan segitiga sama kaki dengan panjang $AB = AC = 10 $. Titik $D$ terletak pada garis $AB$ sejauh $7 $ dari $A$ dan $E$ titik pada garis $AC$ yang terletak sejauh $4 $ dari $A$. Dari $A$ ditarik garis tinggi dan memotong $BC$ di $F$. Jika bilangan rasional $\frac{a}{b}$ menyatakan perbandingan luas segi empat $ADFE$ terhadap luas segitiga $ABC$ dalam bentuk yang paling sederhana, maka nilai $a + b$ adalah \dots

\item (OSK 2014) Diberikan segitiga $ABC$ yang sisi-sisinya tidak sama panjang sehingga panjang garis berat $AN$ dan $BP$ berturut-turut 3 dan 6. Jika luas segitiga $ABC$ adalah $3\sqrt{15}$, maka panjang garis berat ketiga $CM$ adalah \dots

\item (OSK 2012) Diketahui $\triangle ABC$ sama kaki dengan panjang $AB = AC = 3$, $BC = 2$, titik $D$ pada sisi $AC$ dengan panjang $AD = 1$. Tentukan luas $\triangle ABD$.

\item (OSK 2012) Diberikan segitiga $ABC$ dengan keliling 3, dan jumlah kuadrat sisi-sisinya sama dengan 5. Jika jari-jari lingkaran luarnya sama dengan 1, maka jumlah ketiga garis tinggi dari segitiga $ABC$ tersebut adalah \dots

\item (OSN 2011 SMP/MTs) Bangun datar $ABCD$ adalah trapesium dengan $AB$ sejajar $CD$ dan $AB < CD$. Titik $E$ dan $F$ terletak ada $CD$ sehingga $AD$ sejajar $BE$ dan $AF$ sejajar $BC$. Titik $H$  adalah perpotongan $AF$ dengan $BE$ dan titik $G$ adalah  perpotongan $AC$ dengan $BE$. Jika panjang $AB$ adalah 4 cm dan panjang $CD$ adalah 10 cm hitunglah perbandingan luas  segitiga $AGH$ dengan luas trapesium $ABCD$. 
\end{enumerate}