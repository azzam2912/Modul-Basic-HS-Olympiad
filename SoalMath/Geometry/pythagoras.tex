\subsection{Latihan Soal Pythagoras}
\begin{enumerate}
    \item (OSK SMP 2016) Diketahui $ABCD$ dan $CEGH$ adalah dua persegipanjang kongruen dengan panjang $17$ cm, dan lebar $8$ cm. Titik $E$ berada di sisi $AB$ dan $D$ berada di sisi $GH$. Titik $F$ adalah titik potong sisi $AD$ dan $EG$. Luas segiempat $EFDC$ adalah .... $cm^2$.

    \item Misalkan $ABC$ adalah segitiga lancip. Titik $D$, $E$, dan $F$ terletak pada sisi $BC$, $CA$, dan $AB$, berturut-turut, sedemikian sehingga $AD$, $BE$, dan $CF$ adalah garis tinggi segitiga $ABC$. Titik $H$ adalah titik tinggi segitiga $ABC$. Jika $DE = 8$, $DF = 15$, dan $EF = 17$, tentukan panjang $AH$.

    \item (OSK 2015) Diberikan segitiga $ABC$ dengan sudut $\angle ABC = 90^\circ$. Lingkaran $L_1$ dengan $AB$ sebagai diameter sedangkan lingkaran $L_2$ dengan $BC$ sebagai diameternya. Kedua lingkaran $L_1$ dan $L_2$ berpotongan di $B$ dan $P$. Jika $AB = 5$, $BC = 12$ dan $BP = x$ maka nilai dari $\frac{240}{x}$ adalah \ldots

    \item (OSK 2022) Diberikan segitiga $ABC$ siku-siku di B. Titik $D$ berada pada sisi $AB$ dan titik $E$ berada pada sisi $AC$. Diketahui $DE$ sejajar $BC$. Jika $AD = 21$, $DB = 3$, dan $BC = 32$, maka panjang $AE$ adalah \dots

\end{enumerate}