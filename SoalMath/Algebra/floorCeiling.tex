\subsection{Latihan Soal Floor dan Ceiling}
\begin{enumerate}
    \item (Modifikasi JBMO 2021) Carilah seluruh penyelesaian dari persamaan $2\cdot \lfloor{\frac{1}{2x}}\rfloor - 7 = 9(1 - 8x)$.

    \item (OSK 2013) Misalkan $\floor{x}$ menyatakan bilangan bulat terbesar yang lebih kecil atau sama dengan $x$ dan $\ceiling{x}$ menyatakan bilangan bulat terkecil yang lebih besar atau sama dengan $x$. Tentukan semua $x$ yang memenuhi $\floor{x}$ + $\ceiling{x}$ = 5.
    
    \item Banyaknya bilangan asli $n \in \{1,2,3,\dots,1000\}$ sehingga terdapat bilangan real positif $x$ yang memenuhi $x^2+\floor{x}^2=n$ adalah \dots
    
    \item (OSK 2018) Untuk setiap bilangan real $z$, $\lfloor z \rfloor$ menyatakan bilangan bulat terbesar yang lebih kecil dari atau sama dengan $z$. Jika diketahui $\lfloor x \rfloor + \lfloor y \rfloor + y = 43.8$ dan $x + y - \lfloor x \rfloor = 18.4$. Nilai $10(x + y)$ adalah...
    
    \item Let $[x]$ denote the largest integer not exceeding $x$. For example, $[2.1]=2$, $[4]=4$ and $[5.7]=5$. How many positive integers $n$ satisfy the equation $\left[\frac{n}{5}\right]=\frac{n}{6}$.

    \item (OSK 2019) Untuk sebarang bilangan real $x$, simbol $\lfloor x \rfloor$ menyatakan bilangan bulat terbesar yang tidak lebih besar daripada $x$, sedangkan $\lceil x \rceil$ menyatakan bilangan bulat terkecil yang tidak lebih kecil dibanding $x$. Interval $[a, b)$ adalah himpunan semua bilangan real $x$ yang memenuhi
    $$\lfloor 2x \rfloor^2 = \lceil x \rceil + 7$$.
    Nilai $a \cdot b$ adalah ...

    \item (OSK 2021) Jika $a > 1$ suatu bilangan asli sehingga hasil penjumlahan semua bilangan riil $x$ yang memenuhi persamaan
    $$\lfloor x \rfloor^2 - 2ax + a = 0$$
    adalah $p$, maka $a$ adalah ...

    \item Carilah banyaknya bilangan real $x$ yang memenuhi
    $$\floor{x^2}=4x+3.$$

    \item Jika $x$ adalah suatu bilangan real yang memenuhi persamaan berikut
    $$\floor{x}+\floor{2x}+\floor{3x}+\floor{4x}=2024.$$

    Tentukan nilai dari $\floor{6x}$.

    \item Jumlah dari semua solusi bilangan real $x$ dari persamaan
    $$2\floor{x}^2+3\{x\}^2=\frac{7}{4}x\floor{x}$$
    dapat dinyatakan dalam bentuk $\frac{p}{q}$ dimana $p$ dan $q$ merupakan bilangan bulat yang saling relatif prima. Tentukan nilai dari $10p+q$.
\end{enumerate}