\section{Aljabar}
    \subsection{Barisan dan Deret}
    Simpelnya, \textbf{barisan} adalah kumpulan bilangan $a_1,a_2,\dots,a_n$ (beberapa buku atau author memulai dari $a_1$ namun ada juga yang memulai dari $a_0$, kita pakai yang dari $a_1$) yang memenuhi properti atau pola tertentu, sedangkan \textbf{deret} adalah jumlah bilangan-bilangan barisan tadi yaitu $a_1+a_1+a_2+\dots+a_n$ untuk suatu bilangan bulat non-negatif $n$.
    
    Barisan dan deret di atas adalah barisan dan deret terbatas. Bagaimana dengan barisan dan deret tak hingga? Observasi saja nilai $n$ yang sangat besar, atau dapat dikatakan ambil $n \rightarrow \infty.$
    
    \textbf{Sekali lagi barisan (bilangan-bilangan) $\neq$ deret (jumlah).}
    
    Untuk anak SMP (dan SMA) barisan dan deret paling familiar adalah barisan dan deret aritmatika serta geometri. Untuk tantangan, kerjakan latihan soal.
    
    \subsubsection{Notasi Sigma dan Pi}
    Jumlah dari suku-suku pada suatu barisan dinotasikan dengan huruf sigma kapital:
    $$\sum_{i=k}^{j} a_i = a_k+a_{k+1}+\dots+a_j.$$
    Perkalian dari suku-suku pada suatu barisan dinotasikan dengan huruf pi kapital:
    $$\prod_{i=k}^{j} a_i = a_k \cdot a_{k+1}\cdot \ldots \cdot a_j.$$
    
    Ada juga $cyc$ sebagai notasi siklis dan $sym$ sebagai notasi simetris.
    
    Misalkan kita akan mengevaluasi jumlah siklis dan simetris dari $a,b,c$.
    
    Penjumlahan bersifat \textbf{siklis} apabila setiap suku tersebut dapat diperoleh dengan "menggeser" secara siklis atau "memutar" suku lainnya.
    Contoh: $$\sum_{cyc} a = a+b+c$$
         $$\sum_{cyc} ab = ab+ bc + ca$$
    Sedikit mirip dengan penjumlahan siklis, suatu penjumlahan bersifat \textbf{simetris} apabila setiap suku dapat diperoleh dari permutasi $n$ variabel dari suku-suku lainnya pada penjumlahan. Contoh: 
        $$\sum_{sym} a = a+b+c$$
        $$\sum_{sym} ab = ab+ ac + ba + bc + ca + cb$$
    
    
    \subsubsection{Barisan dan Deret Aritmatika}
    Barisan aritmatika in a nutshell: $a,a+b,a+2b,\dots$ dimana setiap \textbf{suku ke-$n$} dari barisan tersebut adalah $$a_n=a+(n-1)b$$ untuk suatu bilangan asli $n$ dan bilangan \textbf{real} $a,b$.
    
    Lalu, \textbf{rumus deret}nya atau jumlah $n$ suku pertama barisan tersebut adalah (coba buktikan rumus ini) $$S_n = a_1+a_2+\dots+a_n=a+(a+b)+(a+2b)+\dots+(a+(n-1)b)=\dfrac{n}{2}(2a+(n-1)b).$$
    
    \subsubsection{Barisan dan Deret Geometri}
     Barisan geometri in a nutshell: $a,ar,ar^2,\dots$ dimana setiap \textbf{suku ke-$n$} dari barisan tersebut adalah $$a_n=ar^{n-1}$$ untuk suatu bilangan asli $n$ dan bilangan \textbf{real} $a$ dan $r \neq 0$.
    
    Lalu, \textbf{rumus deret}nya atau jumlah $n$ suku pertama barisan tersebut adalah (coba buktikan rumus ini) $$S_n = a_1+a_2+\dots+a_n=a+ar+ar^2+\dots+ar^{n-1}=\dfrac{a(1-r^n}{1-r}.$$
    Perhatikan jika $-1 < r < 1$ maka saat $n \rightarrow \infty$ rumus deret tak hingganya menjadi (why?) $$S_\infty =  \dfrac{a}{1-r}.$$
    
    \subsubsection{Rumus Barisan dan Deret Lainnya}
    Sangat dianjurkan untuk mencoba membuktikan rumus-rumus berikut. Untuk bilangan asli $n$, kita punya
    \begin{enumerate}
        \item $1+2+\dots+n = \dfrac{n(n+1)}{2}.$
        \item $1^2+2^2+\dots+n^2 = \dfrac{n(n+1)(2n+1)}{6}.$
        \item $1^3+2^3+\dots+n^3 = \left(1+2+\dots+n\right)^2= \left(\dfrac{n(n+1)}{2}\right)^2.$
    \end{enumerate}
    
    \subsubsection{Prinsip Teleskopik}
    Pernahkah kalian melihat deret yang dapat saling menghilangkan atau bisa "dicoret"? Berikut beberapa bentuk umumnya. Untuk lebih banyak contoh, silakan lihat contoh soal.
    
    \begin{enumerate}
        \item $\sum_{i=1}^{n} (a_{i+1}-a_{i}) = (a_2-a_1)+(a_3-a_2)+\dots+(a_{n+1}-a_{n}) = a_{n+1}-a_1.$
        \item $\prod_{i=1}^{n} \dfrac{a_{i+1}}{a_i} =  \dfrac{a_2}{a_1}\cdot\dfrac{a_3}{a_2}\cdot\dfrac{a_4}{a_3}\cdot\ldots\cdot\dfrac{a_{n+1}}{a_n} = \dfrac{a_{n+1}}{a_1}.$
    \end{enumerate}
    
    \subsection{Fungsi}
    Fungsi $f : A \rightarrow B$ adalah suatu pemetaan dari $A$ ke $B$ dimana $A$ adalah domain dan $B$ adalah kodomain fungsi. Lebh lanjut, fungsi $f$ \textit{well-defined} jika untuk $\forall x \in A$ terdapat tepat satu (jadi ngga boleh ada dua) $y \in B$ sehingga $f(x)=y$. Himpunan seluruh $y$ hasil pemetaan fungsi tersebut dinamakan \textbf{range} fungsi $f$.
    
    Tips-tips mengerjakan soal fungsi adalah (tergantung domainnya)
    \begin{enumerate}
    \item Substitusi $x=0$
    \item Mengganti $x$ dengan $-x$
    \item Mmebentuk suatu bentuk yang simetris (banyak-banyak latihan soal saja biar mengerti).
    \end{enumerate}
    
    \subsubsection{Fungsi Injektif (satu-satu)}
    Definisi: Untuk setiap $x \in A$ ada tepat satu $y \in B$ sehingga $f(x)=y$. 
    Catatan: Range dari fungsi $f$ tidak perlu mencover seluruh kodomain $B$.\\
    Contoh: Dengan $f: \RR \rightarrow \RR$, $f(x)=x$, $f(x)=2x+3$, dll.\\ 
    Contoh yang bukan fungsi injektif: Dengan $f(x)=x^2$ dari real ke real, misalkan $f(x)=4$, berarti $x=2$ dan $x=-2$, padahal biar injektif harusnya $x$ cuma satu aja.
    
    \subsubsection{Fungsi Surjektif (Fungsi Onto)}
    Definisi: Untuk setiap $y \in B$ terdapat $x \in A$ sehingga $f(x)=y$. 
    Catatan: Dapat dikatakan range fungsi tersebut adalah kodomainnya juga (untuk sebagian besar kasus) atau dengan kata lain $f(x)$ "menyentuh" semua nilai yang mungkin di kodomainnya. \\
    Contoh: $f(x)=x$, $f(x)=x^3$, dll.\\
    Contoh yang bukan fungsi surjektif: $f(x)=x^4$ dari real ke real, perhatikan bahwa $f(x)$ harus "mengcover" semua nilai, namun adakah $x$ yang membuat $f(x)=-1$? Tidak ada bukan, berarti bukan fungsi surjektif.
    
    \subsubsection{Fungsi Bijektif}
    Fungsi yang injektif sekaligus surjektif.
    
    \subsection{Komposisi Fungsi}
    Simpelnya, fungsi di dalam fungsi. Untuk fungsi $f:A \rightarrow B$ dan $g:B \rightarrow C$, komposisi fungsinya adalah
    $$(g \circ f)(x) = g(f(x)).$$
    
    \subsubsection{Invers Fungsi}
    Simpelnya kebalikan fungsi. Invers dari fungsi $f: A \rightarrow B$ adalah fungsi $f^{-1} : B \rightarrow A$ dengan didefinisikan sebagai
    $$f^{-1}(f(x))=x.$$
    Dengan kata lain, kita punya $f(x)=y$ jika dan hanya jika $f^{-1}(y)=x$.
    
    \subsection{Polinomial / Suku banyak}
    Suatu polinomial real $P(x)$ yang mempunyai derajat $n$ atau $deg(P) = n$ untuk suatu bilangan bulat non negatif $n$ dinyatakan sebagai
    $$P(x)=a_nx^n+a_{n-1}x^{n-1}+\dots+a_ax+a_0$$
    untuk suatu bilangan real $a_n,a_{n-1},\dots,a_0$ dimana $a_n \neq 0$. Dari teorema fundamental aljabar, setiap polinomial $P(x)$ tersebut memiliki maksimal $n$ akar kompleks dan dapat dinyatakan dalam bentuk
    $$P(x)=a(x-x_1)(x-x_2)\dots(x-x_n)$$
    dimana $a = a_n \neq 0$ dan $x_1,x_2,\dots,x_n$ adalah penyelesaian atau akar-akar dari persamaan $P(x)=0$ atau $a(x-x_1)(x-x_2)\dots(x-x_n)=0$.
    
    Catatan: Polinomial suku-sukunya harus memiliki pangkat positif sehingga $P(x)=x^4+5x+\sqrt{2}$ adalah suatu polinomial, tetapi $P(x)=x^3+5x^2+\dfrac{5}{x^4}+\dfrac{1}{x^8}$ bukan polinomial (karena $\dfrac{1}{x^8}$ dan $\dfrac{5}{x^4}$ bukan suku yang memiliki pangkat positif.
    
    \subsubsection{Remainder Theorem}
    Remainder Theorem = Teorema Sisa.
    Sisa polinomial $P(x)$ saat dibagi $(x-a)$ adalah $P(a)$ (why?).
    Dengan kata lain kita dapat membentuk polinomial $P(x)$ menjadi
    $$P(x)=Q(x)(x-a)+P(a)$$
    untuk suatu polinomial tak nol $Q(x)$ dengan $deg(Q)<deg(P)$.
    
    \subsubsection{Factor Theorem}
    Factor Theorem = Teorema Faktor.
    Hasil bagi polinomial $P(x)$ saat dibagi $(x-a)$ adalah $0$ atau $P(a)=0$ jika dan hanya jika $(x-a)$ adalah faktor dari $P(x)$ (why?).
    Dengan kata lain kita dapat membentuk polinomial $P(x)$ menjadi
    $$P(x)=Q(x)(x-a)$$
    untuk suatu polinomial tak nol $Q(x)$ dengan $deg(Q)<deg(P)$.
    
    \subsubsection{Teorema Vieta}
    \begin{itemize}
        \item $x_1+x_2+\dots+x_n=-\dfrac{a_{n-1}}{a_n}$
        \item $x_1x_2+x_1x_3+\dots+x_{n-1}x_n=\dfrac{a_{n-2}}{a_n}$
        \item \dots
        \item $x_1x_2x_3\ldots x_n = (-1)^{n-1}\dfrac{a_{0}}{a_n}$
    \end{itemize}
    
    Contoh:
    \begin{enumerate}
    \item Untuk $ax^3+bx^2+cx+d=0$ kita punya $x_1+x_2+x_3=-\dfrac{b}{a}$, $x_1x_2+x_1x_3+x_2x_3=\dfrac{c}{a}$, dan $x_1x_2x_3=-\dfrac{d}{a}$.
    \item Untuk $ax^2+bx+c=0$ kita punya $x_1+x_2=-\dfrac{b}{a}$ dan $x_1x_2=\dfrac{c}{a}$.
    \end{enumerate}
    
    \subsubsection{Fungsi Kuadrat}
    Fungsi kuadrat atau polinomial pangkat 2 $P(x)=ax^2+bx+c$ termasuk salah satu polinomial yang memiliki derajat 2 dan mempunyai beberapa properti penting:
    \begin{enumerate}
    \item Diskriminan $D=b^2-4ac$. Hal-hal berikut adalah \textbf{akibat langsung dari rumus abc di bawah}: $D>0$ jika dan hanya jika $P(x)$ mempunyai 2 akar real berbeda,  $D=0$ jika dan hanya jika $P(x)$ mempunyai 2 akar real kembar, dan $D<0$ jika dan hanya jika $P(x)$ mempunyai akar kompleks tidak real.
    \item Rumus abc / penyelesaian umum: $x_{1,2} = \dfrac{-b \pm \sqrt{b^2-4ac}}{2a} = \dfrac{-b+\sqrt{D}}{2a}$. Perhatikan dari rumus ini dapat dibuktikan juga rumus Vieta yang sebelumnya ditunjukkan.
    \end{enumerate}
    
    
\section{Teori Bilangan}
    \subsection{FPB dan KPK dua bilangan}
    Secara matematis FPB (atau gcd - greatest common divisors) dan KPK (atau lcm - least common multiples) dari dua bilangan bulat positif $a$ dan $b$ 
    didefinisikan sebagai:
    $$FPB(a,b) = gcd(a,b) = p_1^{\min\{a_1,b_1\}}\cdot p_2^{\min\{a_2,b_2\}} \cdot \ldots \cdot p_n^{\min\{a_n,b_n\}}$$
    $$KPK(a,b) = lcm(a,b) =p_1^{\max\{a_1,b_1\}}\cdot p_2^{\max\{a_2,b_2\}} \cdot \ldots \cdot p_n^{\max\{a_n,b_n\}}$$
    dimana kedua bilangan tersebut dapat difaktorisasi prima menjadi
    $a=p_1^{a_1}\cdot p_2^{a_2}\cdot \ldots \cdot p_n^{a_n}$ dan $b=p_1^{b_1}\cdot p_2^{b_2} \cdot \ldots \cdot p_n^{b_n}$, dengan $p_1,p_2,\dots,p_n$ adalah bilangan prima berbeda, serta $a_1,a_2,\dots,a_n,b_1,b_2,\dots,b_n$ adalah bilangan bulat non-negatif.
    
    Dari definisi tersebut mudah dibuktikan bahwa
    $$FPB(a,b) \cdot KPK(a,b) = ab.$$
    
    Perlu dicatat, bahwa $a$ dan $b$ tidak boleh bernilai nol. Untuk $a,b$ yang bernilai negatif, didefinisikan $FPB(a,b) = FPB(|a|,|b|)$.
    \subsubsection{Algoritma Euclid}
    Pada dasarnya algoritma ini bertumpu pada sebuah teorema:
    $$FPB(a,b) = FPB(a,b-a)$$
    
    \subsection{Fungsi yang Melibatkan Faktor Bilangan}
    Misalkan $a$ dapat difaktorisasi prima seperti sebelumnya yaitu $a=p_1^{a_1}\cdot p_2^{a_2}\cdot \ldots \cdot p_n^{a_n}$.
    \subsubsection{Banyaknya Faktor Positif}
    Fungsi $d(a)$ didefinisikan sebagai banyaknya faktor atau pembagi positif dari $a$ dengan
    $$d(a) = (a_1+1)(a_2+1)\cdot \ldots \cdot (a_n+1).$$
    
    Contoh: Banyaknya faktor positif dari $12= 2^2 \cdot 3^1$ adalah $d(12)=(2+1)(1+1)=6$ dengan pembagi positifnya adalah $1,2,3,4,6,12$ (ada 6 faktor positif.)
    
    \subsubsection{Jumlah Faktor Positif}
    Fungsi $\sigma (a)$ didefinisikan sebagai banyaknya faktor atau pembagi positif dari $a$ dengan
    $$\sigma (a) = (p_1^0+p_1^1+p_1^2+\dots+p_1^{a_1})(p_2^0+p_2^1+p_2^2+\dots+p_2^{a_2})\cdot \ldots \cdot (p_n^0+p_n^1+p_n^2+\dots+p_n^{a_n}).$$
    
    Contoh: Jumlah faktor atau pembagi positif dari $12= 2^2 \cdot 3^1$ adalah $d(12)=(2^0+2^1+2^2)(3^0+3^1)=28$ yang setara dengan penjumlahan secara manualnya, yaitu $2^03^0+2^03^1+2^13^0+2^13^1+2^23^0+2^23^1=28.$
    
    \subsubsection{Banyaknya Bilangan Relatif Prima}
    \begin{remark*}
    	     Bilangan $b$ dikatakan relatif prima dengan $a$ jika dan hanya jika $FPB(a,b)=1$.
    	\end{remark*}
     Definisikan fungsi Euler Totient Phi $\phi(b)$ sebagai banyaknya bilangan bulat positif $n$ yang kurang dari sama dengan $b$ dimana $n$ relatif prima dengan $b$. Rumus eksplisit untuk menghitung fungsi ini adalah
    $$\phi(n) = a\left(1-\dfrac{1}{p_1}\right)\left(1-\dfrac{1}{p_2}\right)\dots\left(1-\dfrac{1}{p_n}\right).$$
    
    Contoh: $\phi(4)=4(1-\frac{1}{2})=2$ karena ada 2 bilangan yang realtif prima dengan 4, yaitu 1 dan 3.
    
    Catatan: Untuk semua bilangan prima $p$, nilai $\phi(p) = p-1$. (Silakan dibuktikan sendiri :D)
    
    \subsection{Euler's Theorem on Modulo}
    Untuk bilangan asli $a$ dan $n$ yang saling relatif prima kita punya
    $$a^{\phi(n)} \equiv 1 \mod n.$$
    
    \subsection{Fermat's Little Theorem}
    Teorema ini merupakan kasus khusus dari Euler's Theorem saat $n$ prima sehingga $\phi(n)=n-1$. Untuk bilangan prima $p$, kita punya
    $$a^{p-1} \equiv 1 \mod p.$$
    
    \subsection{Wilson's theorem}
    Untuk suatu bilangan asli $p$, kita punya $p$ adalah bilangan prima jika dan hanya jika
    $$(p-1)! \equiv -1 \mod p.$$
    
\section{Kombinatorika}
    \subsection{Binomial Newton}
    $(a+b)^n = {n \choose 0} a^nb^0 + {n \choose 1} a^{n-1}b^1+ \dots +{n \choose n}a^0b^n$
    
    
    \subsection{Pigeon Hole Principle (PHP)}
    Teorema yang dalam Bahasa Indonesia ini disebut dengan Teorema Sangkar Burung Merpati secara matematis berbunyi:
    Jika ada $kn+1$ merpati dan $n$ sangkar, maka setidaknya ada satu sagnkar yang berisi $k+1$ burung merpati.
    
    Versi lebih simpelnya adalah: jika ada $n+1$ objek yang akan dibagi ke dalam $n$ buah kotak, maka setidaknya ada 1 kotak yang berisi 2 objek.
    
    Contoh: \begin{itemize}
        \item Di dalam ruangan berisi 3 orang, pasti terdapat setidaknya 2 orang berjenis kelamin sama.
        \item Jika ada 367 orang di suatu sekolah, maka setidaknya ada dua orang diantara mereka yang tanggal lahirnya persis sama.
    \end{itemize}
    
    \subsection{Relasi Rekurensi}
    Sering disebut dengan rekursif. Intinya adalah sebuah persamaan yang melibatkan barisan $a_1, a_2, \dots , a_n$ dimana untuk mendapatkan nilai $a_k$ membutuhkan suku-suku sebelumnya $a_{k-1}, a_{k-2}, \dots,$ atau $a_1$. 
    
    Contoh paling terkenal dari persamaan rekursif adalah bilangan Fibonacci $0,1,1,2,3,5,8,13,21,\dots$ yang secara matematis didefinisikan sebagai berikut.
    \begin{align*}
        F_0 &= 0, F_1 = 1\\
        F_n &= F_{n-1}+F_{n-2} \text{ untuk } n \ge 2
    \end{align*}
    
    atau yang lebih terkenal di ranah \textit{Computer Science} adalah permasalahan \textit{Tower of Hanoi} dengan persamaan rekursifnya didefinisikan sebagai berikut.
    \begin{align*}
        T_1 &= 1 \\
        T_n &= 2T_{n-1}+1
    \end{align*}
    
    Untuk menyelesaikan soal relasi rekurensi, butuh manipulasi aljabar yang mumpuni sehingga tidak ada pendekatan selain menggunakan persamaan karakteristik atau fungsi pembangkit (tidak dibahas disini) yang dijamin berhasil.
    
    \subsubsection{Persamaan Karakteristik untuk Relasi Rekurensi Linear}
    Persamaan karakteristik berikut berlaku untuk persamaan rekursif yang linear. Persamaan karakteristik berikut berguna untuk mengubah relasi rekurensi menjadi iteratif, atau persamaan berbentuk implisit. (Jadi, untuk persamaan yang bukan linear, sebagai contoh $a_n = a_{n-1}^2 + a_{n-2}$ tidak bisa dijamin selesai dengan persamaan karakteristik yang disajikan berikut).
    Untuk persamaan rekursif
    $$a_n = c_1a_{n-1}+c_2a_{n-2}+\dots+c_da_{n-d}$$
    mempunyai persamaan karakteristik
    $$x^d-c_1x^{d-1}-c_2x^{d-2}-\dots-c_dx^0=0$$
    
    Sebagai contoh, rumus rekursif dari barisan Fibonacci di atas dapat diselesaikan menjadi 
    $$x^{n}-x^{n-1}-x^{n-2}=0 \implies x^2-x-1=0$$
    yang mempunyai dua akar, yaitu $x_1 = \dfrac{1+\sqrt{5}}{2}=\phi$ dan $x_2 = \dfrac{1-\sqrt{5}}{2}=1-\phi$ dimana $\phi$ adalah \textit{Golden Ratio}. Sadari bahwa setiap suku di barisan Fibonacci tersebut berbentuk $F_n = c_1x_1^n + c_2x_2^n$ (buktikan). Dengan substitusi $x_1$ dan $x_2$ serta pemilihan suku dari barisan Fibonacci (misal suku pertama dan kedua) maka akan ditemukan nilai $c_1$ dan $c_2$ sehingga pada akhirnya kita punya
    $$F_n=\dfrac{\phi^n-(1-\phi)^n}{\sqrt{5}}$$
    
    

\section{Geometri}
    \subsection{Segi-n}
    Pada segi-$n$, jumlah sudut totalnya adalah $(n-2)\times 180^\circ$. Hal ini dikarenakan kita dapat membuat $n-2$ segitiga di dalam sembarang segi-$n$ dimana jumlah total sudut dalam setiap segitiga adalah $180^\circ$.\\
    
    Dari fakta tersebut, berarti besar setiap sudut segi-$n$ beraturan adalah $\dfrac{(n-2)}{n}180^\circ$.
    
    \subsection{Luas}
    Misalkan $[A_1A_2A_3\dots A_n]$ menotasikan luas bangun $A_1A_2A_3\dots A_n$. 
    
    Misalkan $r$ adalah panjang jari-jari lingkaran dalam $\triangle ABC$, $s=\frac{1}{2}(a+b+c)$ adalah setengah keliling $\triangle ABC$ dimana $a=BC, b=CA, c=AB$ serta $AT$ adalah garis tinggi $\triangle ABC$, maka
    
    $$[ABC]  = \dfrac{1}{2}BC \cdot AT = \dfrac{1}{2}AB\cdot AC \cdot \sin \angle A = rs$$
    
    \subsubsection{Rumus Heron}
    $$[ABC]  = \sqrt{s(s-a)(s-b)(s-c)}$$
    
    \subsection{Perbandingan Luas}
    Jika pada segitiga $ABC$, titik $D,E,F$ berturut-turut berada di segmen $BC$,$CA$,$AB$, maka 
    \begin{enumerate}
        \item $\dfrac{BD}{DC} = \dfrac{[ABD]}{[ADC]}.$
        \item $\dfrac{[AEF]}{[ABC]}=\dfrac{AE \cdot AC}{AF \cdot AB}.$
    \end{enumerate}
    
    \subsection{Dalil Ceva}
    Jika pada segitiga $ABC$, titik $D,E,F$ berturut-turut berada di segmen $BC$,$CA$,$AB$, maka 
    $AD,BE,CF$ konkuren atau berpotongan di satu titik jika dan hanya jika $$\dfrac{AF}{FB} \cdot \dfrac{BD}{DC} \cdot \dfrac{CE}{EA} = 1.$$
    
    \subsection{Dalil Menelaus}
    Jika pada segitiga $ABC$, titik $D,E$ berturut-turut berada di segmen $BC$,$CA$,serta $F$ pada garis $AB$, maka $D,E,F$ segaris jika dan hanya jika
    $$\dfrac{AF}{FB} \cdot \dfrac{BD}{DC} \cdot \dfrac{CE}{EA} = 1.$$
    
    \subsection{Teorema Garis Bagi}
    Misalkan garis bagi sudut $\angle A$ (bisa garis bagi dalam atau garis bagi luar) memotong garis $BC$ di $J$, maka 
    $$\dfrac{BJ}{JC} = \dfrac{AB}{AC}.$$
    
\section{Latihan Soal}

\subsection{Aljabar}
\begin{enumerate}
    \item (OSK 2006) Diketahui $a+(a+1)+(a+2)+\dots+50=1139$. Jika $a$ bilangan real positif, maka $a=\dots$
    
    \item (AIME 1984) Barisan $a_1,a_2,\dots,a_{98}$ memenuhi $a_{n+1}=a_n+1$ untuk $n=1,2,\dots,97$ dan mempunyai jumlah sama dengan $137$. Tentukan nilai dari $a_2+a_4+a_6+\dots+a_{98}$.
    
    \item (AIME 2003) Diketahui $0<a<b<c<d$ adalah bilangan bulat dimana $a,b,c$ membentuk barisan aritmatika sedangkan $b,c,d$ membentuk barisan geometri. Jika $d-a=30$ maka tentukan nilai dari $a+b+c+d$.
    
    \item (OSK 2009) Bilangan bulat positif terkecil $n$ dengan $n> 2009$ sehingga $$\sqrt{\dfrac{1^3+2^3+3^3+\dots+n^3}{n}}$$
    merupakan bilangan bulat adalah \dots
    
    \item Tentukan nilai paling sederhana dari $\sqrt{2\sqrt{2\sqrt{2\sqrt{\dots}}}}$.
    
    \item Tentukan nilai dari $\dfrac{1}{1\cdot2}+\dfrac{1}{2\cdot 3}+\dfrac{1}{3 \cdot 4}+\dots+\dfrac{1}{2020 \cdot 2021}.$
    
    \item  Nilai paling sederhana dari $\left(1-\dfrac{1}{2^2}\right)\cdot\left(1-\dfrac{1}{3^2}\right)\cdot\left(1-\dfrac{1}{4^2}\right)\cdot\dots\cdot\left(1-\dfrac{1}{2021^2}\right)$ adalah \dots
    
    \item Tentukan hasil dari jumlah $\dfrac{1}{\sqrt{1}+\sqrt{2}}+\dfrac{1}{\sqrt{2}+\sqrt{3}}+\dots+\dfrac{1}{\sqrt{99}+\sqrt{100}}.$
    
    \item Nilai $x$ yang memenuhi persamaan
    $$\sqrt{x\sqrt{x\sqrt{x\sqrt{\dots}}}}=\sqrt{4x+\sqrt{4x+\sqrt{4x+\sqrt{\dots}}}}$$
    adalah \dots
    
    \item Hitunglah nilai paling sederhana dari
    $$6-\dfrac{5}{3+\dfrac{4}{3+\dfrac{4}{3+\dfrac{4}{3+\dfrac{4}{\dots}}}}}$$
    
    \item Jika $f:\RR-\{0\} \rightarrow \RR$ adalah fungsi yang memenuhi $f(x)+2f\left(\frac{1}{x}\right)=3x$ untuk setiap bilangan real $x \neq 0$, maka tentukan nilai $f(2021).$
    
    \item (OSP 2004) Misalkan $f$ sebuah fungsi yang memenuhi $f(x)f(y)-f(xy)=x+y,$ untuk setiap bilangan bulat $x$ dan $y$. Berapakah nilai $f(2004)$?
    
    \item (OSK 2011) Misalkan $f$ suatu fungsi yang memenuhi $f(xy) = \dfrac{f(x)}{y}$ untuk semua bilangan real positif $x$ dan $y$. Jika $f(100)=3$ maka $f(10)$ adalah \dots
    
    \item (OSP 2009) Suatu fungsi $f:\ZZ \rightarrow \QQ$ mempunyai sifat $f(x+1)=\dfrac{1+f(x)}{1-f(x)}$ untuk setiap $x \in \ZZ$. Jika $f(2)=2$, maka nilai fungsi $f(2009)$ adalah \dots
    
    \item Jika $P(x)$ dibagi $x^2-x$ dan $x^2+x$ berturut-turut akan bersisa $5x+1$ dan $3x+1$, maka bila $P(x)$ dibagi $x^2-1$ sisanya adalah \dots
    
    \item (OSP 2006) Jika $(x-1)^2$ membagi $ax^4+bx^3+1$, maka $ab=\dots$
    
    \item Diketahui suatu polinomial $P(x)$ memenuhi $P(k)=\dfrac{k}{k+1}$ untuk $k=1,2,3,\dots,2020$. Jika $P(0)=1$, nilai $P(2022)=\dots$
    
    \item (OSK 2010) Polinom $P(x)=x^3-x^2+x-2$ mempunyai tiga pembuat nol yaitu $a,b,$ dan $c$. Nilai dari $a^3+b^3+c^3$ adalah \dots
    
    \item (OSP 2010) Persamaan kuadrat $x^2-px-2p=0$ mempunyai dua akar real $\alpha$ dan $\beta$. Jika $a^3+b^3=16$, maka hasil jumlah semua nilai $p$ yang memenuhi adalah \dots 
    
    \item Diberikan polinomial $P(x)=x^4+ax^3+bx^2+cx+d$ dengan $a,b,c,$ dan $d$ konstanta. Jika $P(1)=10$, $P(2)=20$, dan  $P(3)=30$, maka nilai
    $$\dfrac{P(12)+P(-8)}{10}=\dots$$
\end{enumerate}

\subsection{Teori Bilangan}
    \begin{enumerate}
        \item (OSK 2011) Bilangan asli terkecil lebih dari 2011 yang bersisa 1 jika dibagi 2,3,4,5,6,7,8,9,10 adalah \dots
        
        \item (OSK 2009) Nilai dari $\sum_{k=1}^{2009} FPB(k,7)$ adalah \dots
        
        \item Banyaknya anggota himpunan himpunan 
        $$S = \{gcd(n^3+1,n^2+3n+9 \mid n \in Z\}$$
        adalah \dots
        
        \item (AIME 1998) Ada berapa banyak bilangan bulat positif $k$ sehingga $lcm(6^6,8^8,k)=12^{12}$?
        
        \item Jika $a,b$ adalah bilangan asli dan $c$ adalah bilangan bulat, buktikan bahwa $gcd(a,b)=gcd(a,b+ac).$
        
        \item (OSP 2009) Misalkan $n$ bilangan asli terkecil yang mempunyai tepat 2009 faktor dan $n$ merupakan kelipatan 2009. Faktor prima terkecil dari $n$ adalah \dots
        
        \item Misalkan $n$ bilangan asli dimana $2n$ mempunyai 28 faktor positif dan $3n$ mempunyai 30 faktor positif. Banyaknya faktor positif yang dimilik $6n$ adalah \dots
        
        \item (OSK 2011) Ada berapa faktor positif dari $2^73^55^37^2$ yang merupakan kelipatan 10?
        
        \item (AIME 1995) Tentukan banyaknya faktor positif dari $n^2$ yang kurang dari $n$ tetapi tidak membagi $n$ jika $n^{2^{31}}3^{19}.$
        
        \item (AIME 2000) Tentukan bilangan asli terkecil yang memiliki 12 faktor positif genap dan $6$ faktor positif ganjil.
        
        \item Berapakah sisa pembagian $43^{43^{43}}$ oleh 100?
        
        \item Jika $10^{999999999}$ dibagi oleh 7, maka sisanya adalah \dots
        
        \item Tentukan sisa saat $2^{70}+3^{70}$ saat dibagi 13.
        
        \item Berapa banyak bilangan di himpunan $\{1,2,\dots,200\}$ yang relatif prima dengan 100?
        
        \item Carilah nilai $FPB(19!+19,20!+19).$

 \end{enumerate}
 
\subsection{Kombinatorika}
\begin{enumerate}
    \item Carilah koefisien $x^4$ dari penjabaran $(x+1)^9$
    
    \item (OSK 2013) Koefisien $x^{2013}$ pada ekspansi
    $$(1+x)^{4026}+x(1+x)^{4025}+x^2(1+x)^{4024}+\dots x^{2013}(1+x)^{2013}$$
    adalah \dots
    
    \item Jika $S=(\sqrt{71}+1)^{71}-(\sqrt{71}-1)^{71}$ adalah bilangan bulat, carilah digit terakhir dari $S$
    \item Berapa banyak orang minimum yang harus hadir di suatu pesta sehingga dipastikan terdapat 3 orang yang lahir di bulan yang sama di pesta itu?
    
    \item Misalkan Naruko memilih $k$ buah bilangan dari himpunan $\{1,2,3,\dots,2016\}$ secara acak. Berapakah nilai $k$ terkecil sehingga Naruko pasti bisa mendapatkan setidaknya sepasang bilangan (dari $k$ bilangan itu) yang jika dijumlahkan hasilnya 2017?
    
    \item Suatu malam di rumah WonYoung terjadi pemadaman listrik. Karena WonYoung sangat malas, ia hanya ingin tidur dengan membawa banyak kaus kaki (hobi yang aneh :/). Ia mengambil kaus kaki dari lemari di ruangan yang sangat gelap. Lemari itu berisi 100 buah kaus kaki merah, 80 kaus kaki hijau, 60 kaus kaki biru, dan 40 kaus kaki hitam. WonYoung mengambil banyak kaus kaki tapi tidak bisa tahu warnanya. Berapa banyak kaus kaki paling sedikit yang perlu diambil sehingga dijamin terdapat setidaknya 10 pasang kaus kaki (dengan setiap pasang kaus kaki harus berwarna sama) ?
    
    \item (OSK 2011) Di lemari hanya ada 2 macam kaos kaki yaitu kaos kaki berwarna hitam dan putih. 
Ali, Budi dan Candra berangkat di malam hari saat mati lampu dan mereka mengambil kaos kaki 
secara acak di dalam lemari dalam kegelapan. Berapa kaos kaki minimal harus mereka ambil untuk 
memastikan bahwa akan ada tiga pasang kaos kaki yang bisa mereka pakai ? (Sepasang kaos kaki harus 
memiliki warna yang sama).
    
    \item Tandai satu buah kartu dengan angka 1, dua buah kartu dengan angka 2, tiga buah kartu dengan 
angka satu hingga lima puluh buah kartu dengan angka 50. Semua kartu tersebut dimasukkan ke 
dalam kotak. Berapa buah kartu minimal yang harus diambil agar dapat dipastikan terdapat sekurang-kurangnya 10 buah kartu dengan tanda angka yang sama 

    \item (OSK 2016)
	Anak laki-laki dan anak perempuan yang berjumlah 48 orang duduk melingkar secara acak.
Banyaknya minimum anak perempuan sehingga pasti ada enam anak perempuan yang duduk
berdekatan tanpa diselingi anak laki-laki adalah \dots
    
\end{enumerate}

\subsection{Geometri}
\begin{enumerate}
    \item (OSK 2006) Pada segitiga $ABC$, titik $F$ membagi sisi $AC$ dalam perbandingan $1 : 2$. 
Misalkan $G$ titik tengah $BF$ dan $E$ titik perpotongan antara sisi $BC$ dengan $AG$. Maka titik $E$ membagi sisi $BC$
dalam perbandingan \dots

\item (OSK 2009) Diberikan persegi $ABCD$ dengan panjang sisi 10. Misalkan $E$ pada $AB$ dan $F$
pada $BC$ dengan $AE = FB = 5$. Misalkan $P$ adalah titik potong $DE$ dan $AF$. Luas $DCFP$ adalah \dots

\item (OSN 2011 SMP/MTs) Bangun datar $ABCD$ adalah
trapesium dengan $AB$ sejajar $CD$ dan $AB < CD$. Titik $E$ dan $F$ terletak
pada $CD$ sehingga $AD$ sejajar $BE$ dan $AF$ sejajar $BC$. Titik $H$ 
adalah perpotongan $AF$ dengan $BE$ dan titik $G$ adalah 
perpotongan $AC$ dengan $BE$. Jika panjang $AB$ adalah 4 cm
dan panjang $CD$ adalah 10 cm hitunglah perbandingan luas 
segitiga $AGH$ dengan luas trapesium $ABCD$. 

\item Pada segitiga $ABC$, titik $D$ dan $E$ berturut-turut berada di segmen $AB$ dan $AC$. Misalkan $CD$ dan $BE$ berpotongan di titik $F$. Jika $[DBF]=3$, $[BFC]=6$, dan $[CFE]=4$, berapakah luas $ADFE$?

\item (OSK 2016) Pada segitiga ABC, titik $X, Y$ dan $Z$ berturut-turut terletak pada sinar $BA, CB$ dan $AC$
	sehingga $BX = 2BA, CY = 2CB$ dan $AZ = 2AC$. Jika luas segitiga $ABC$ adalah 1, maka luas
	segitiga $XYZ$ adalah \dots
	
\item (OSK 2016) Segitiga $ABC$ merupakan segitiga sama kaki dengan panjang $AB = AC = 10 $.
	Titik $D$ terletak pada garis $AB$ sejauh $7 $ dari $A$ dan $E$ titik pada garis $AC$ yang
	terletak sejauh $4 $ dari $A$. Dari $A$ ditarik garis tinggi dan memotong $BC$ di $F$.
	Jika bilangan rasional $\frac{a}{b}$ menyatakan perbandingan luas segi empat $ADFE$ terhadap
	luas segitiga $ABC$ dalam bentuk yang paling sederhana, maka nilai $a + b$ adalah \dots
	
	\item (OSK 2014) Diberikan segitiga $ABC$ dengan $AB = 360, BC = 240,$ dan $AC = 180$. Garis
	bagi dalam dan garis bagi luar dari $\angle CAB$ memotong $BC$ dan perpanjangan $BC$
	berturut-turut di $P$ dan $Q$. Jari-jari lingkaran yang melalui titik-titik $A, P,$ dan $Q$
	adalah \dots
	
	\item (OSK 2012) Diketahui $\triangle ABC$ sama kaki dengan panjang $AB = AC = 3$, $BC = 2$, titik $D$ pada sisi $AC$ dengan panjang $AD = 1$. Tentukan luas $\triangle ABD$.
\end{enumerate}