\section{Logika Dasar}
\begin{enumerate}
    \item $A \land B$ dibaca $A$ dan $B$.
    \item $A \lor B$ dibaca $A$ atau $B$.
    \item $A \equiv B$ dibaca $A$ ekuivalen $B$ (untuk logika).
    \item $\exists x$ dibaca ada $x$ atau terdapat $x$.
    \item $\forall x$ dibaca untuk semua $x$.
    \item $A \implies B$ dibaca
    \begin{enumerate}
        \item $A$ hanya jika $B$,
        \item $B$ jika $A$,
        \item $A$ mengimplikasikan $B$,
        \item $A$ menyebabkan $B$,
        \item jika $A$ maka $B$.
    \end{enumerate}
    \item $A \Longleftarrow B$ dibaca $A$ jika $B$ (kebalikannya $\implies$).
    \item $A \iff B$ dibaca $A$ jika dan hanya jika $B$. Definisinya adalah $A \iff B \equiv (A \implies B) \land (A \Longleftarrow B)$.
\end{enumerate}
Apa bedanya $A \implies B$ dan $A \iff B$? Kalau $A \implies B$ berarti agar pernyataan benar haruslah $B$ benar, $A$ bisa salah atau benar. Kalau $A \iff B$, agar pernyataan benar, haruslah $A$ dan $B$ sama-sama benar atau sama-sama salah. Contohnya:
\begin{itemize}
    \item Jika sekarang hujan, maka saya tidak pergi. (Baik sekarang hujan ataupun tidak hujan, bisa saja saya tidak pergi, jadi tidak pengaruh).
    \item Saya laki-laki jika dan hanya jika saya bukan perempuan.
        \item Saya tidak bernafas selamanya jika dan hanya jika saya tidak hidup.
\end{itemize}