\subsection{Segitiga}
Pada segitiga $ABC$ seperti gambar berikut:
\begin{center}
\begin{tikzpicture}
    % Define the coordinates of the vertices
    \coordinate (A) at (0,0);
    \coordinate (B) at (8,0);
    \coordinate (C) at (1.5,4);
    
    % Draw the triangle
    \draw (A) -- (B) -- (C) -- cycle;
    
    % circumcircle
    \tkzCircumCenter(A,B,C)\tkzGetPoint{O}
    \tkzDrawPoint(O)
    \tkzDrawCircle(O,A)
    
    % incircle
    \tkzDefCircle[in](A,B,C)\tkzGetPoint{I}\tkzGetLength{rIN}
    \tkzDrawPoint(I)
    \tkzDrawCircle[R](I,\rIN pt)
    
    % angle bisector
    \tkzDrawBisector[blue](B,A,C)\tkzGetPoint{E}
    \tkzDrawCircle[R](E,1 pt)
    
    %altitude
    \tkzDefPointBy[projection=onto B--C](A) \tkzGetPoint{D}
    \tkzDefPointBy[projection=onto B--A](C) \tkzGetPoint{C1}
    \tkzInterLL(A,D)(C,C1) \tkzGetPoint{H}
    \tkzDrawPoints(H) \tkzLabelPoints[below right](H)
    \tkzDrawSegment[red](A,D)
    \tkzDrawSegment[red](C,C1)

    %garis sumbu
    \tkzDefPointBy[projection=onto B--C](O) \tkzGetPoint{M}
    \tkzDefPointBy[homothety=center O ratio 3.2](M) \tkzGetPoint{M1}
    \tkzDefPointBy[homothety=center O ratio -3.2](M) \tkzGetPoint{M2}
    \tkzDrawSegment(M1,M2)
    \tkzDefPointBy[projection=onto B--A](O) \tkzGetPoint{M3}

    %centroid
    \tkzDrawSegment[green](A,M)
    \tkzDrawSegment[green](C,M3)
    \tkzInterLL(A,M)(C,M3) \tkzGetPoint{G}
    \tkzDrawPoints(G) \tkzLabelPoints[below right](G)
    
    % Label the vertices
    \node[below left] at (A) {$A$};
    \node[below right] at (B) {$B$};
    \node[above] at (C) {$C$};
    \node[below right] at (I) {$I$};
    \node[right] at (O) {$O$};
    \node[above right] at (E) {$E$};
    \node[above] at (D) {$D$};
    \node[right] at (M) {$M$};
\end{tikzpicture}
\end{center}
\begin{enumerate}
    \item Garis bagi $AE$ yaitu garis yang membagi dua sudut $A$ sama besar sehingga $\angle BAE = \angle EAC$. 
    \item Garis berat $AM$ dengan $M$ adalah titik tengah $BC$.
    \item Garis tinggi $AD$ adalah garis yang tegak lurus dengan $BC$. $D$ biasa disebut dengan proyeksi $A$ ke $BC$.
    \item Garis $OM$ adalah salah satu garis sumbu segitiga $ABC$, yaitu garis yang melewati titik tengah sisi segitiga dan tegak lurus dengan sisi itu.
    \item Pertemuan atau perpotongan ketiga garis tinggi segitiga $ABC$ adalah titik tinggi, dalam gambar ini adalah $H$ (orthocenter).
    \item Pertemuan atau perpotongan ketiga garis bagi segitiga $ABC$ adalah titik bagi atau titik pusat lingkaran dalam (incircle) segitiga $ABC$ dalam gambar ini adalah $I$ (incenter).
    \item Pertemuan atau perpotongan ketiga garis berat segitiga $ABC$ adalah titik berat (centroid).
    \item Pertemuan atau perpotongan ketiga garis sumbu segitiga $ABC$ adalah titik pusat lingkaran luar (circumcircle) segitiga $ABC$ yang dalam gambar ini adalah $O$ (circumcenter).
    \item Berlaku \textbf{ketaksamaan segitiga} yaitu $AB+BC>CA$, $BC+CA>AB$, dan $CA+AB>BC$. Selain itu juga berlaku $|AB-BC|<CA$, $|BC-CA|<AB$, dan $|CA-AB|<BC$.
\end{enumerate}
\subsection{Kesebangunan Segitiga}
\begin{center}
\begin{tikzpicture}
  % First triangle
  \coordinate (A) at (0,0);
  \coordinate (B) at (3,0);
  \coordinate (C) at (2,2);
  \draw (A) -- (B) -- (C) -- cycle;

  % Second triangle
  \coordinate (D) at (6,0);
  \coordinate (E) at (12,0);
  \coordinate (F) at (8,4);
  \draw (D) -- (E) -- (F) -- cycle;

  % Labeling the vertices
  \node[below] at (A) {$A$};
  \node[below] at (B) {$B$};
  \node[above] at (C) {$C$};
  \node[below] at (D) {$D$};
  \node[below] at (E) {$E$};
  \node[above] at (F) {$F$};

  \draw pic[draw=green!30,fill=green!30,angle radius=0.5cm] {angle=A--C--B};
  \draw pic[draw=green!30,fill=green!30,angle radius=0.5cm] {angle=D--F--E};
  \draw pic[draw=red!30,fill=red!30,angle radius=0.5cm] {angle=B--A--C};
  \draw pic[draw=red!30,fill=red!30,angle radius=0.5cm] {angle=F--E--D};
  \draw pic[draw=blue!30,fill=blue!30,angle radius=0.5cm] {angle=C--B--A};
  \draw pic[draw=blue!30,fill=blue!30,angle radius=0.5cm] {angle=E--D--F};
\end{tikzpicture}
\end{center}



Segitiga $ABC$ dan $DEF$ sebangun atau $ABC \sim DEF$ jika dan hanya jika minimal salah satu syarat ini terpenuhi:
\begin{enumerate}
    \item $\angle ABC = \angle DEF$ dan $\angle BAC = \angle EDF$.
    \item $\dfrac{AB}{DE} = \dfrac{BC}{EF} = \dfrac{CA}{FD}$.
    \item $\dfrac{AB}{DE} = \dfrac{BC}{EF}$ dan $\angle ABC = \angle DEF$ (sudut yang diapit dua sisi yang diperbandingkan nilainya sama)
\end{enumerate}

\subsection{Kekongruenan Segitiga}
\begin{center}
\begin{tikzpicture}
  % First triangle
  \coordinate (A) at (0,0);
  \coordinate (B) at (3,0);
  \coordinate (C) at (2,2);
  \draw (A) -- (B) -- (C) -- cycle;

  % Second triangle
  \coordinate (D) at (6,0);
  \coordinate (E) at (9,0);
  \coordinate (F) at (7,2);
  \draw (D) -- (E) -- (F) -- cycle;

  % Labeling the vertices
  \node[below] at (A) {$A$};
  \node[below] at (B) {$B$};
  \node[above] at (C) {$C$};
  \node[below] at (D) {$D$};
  \node[below] at (E) {$E$};
  \node[above] at (F) {$F$};

  \draw pic[draw=green!30,fill=green!30,angle radius=0.5cm] {angle=A--C--B};
  \draw pic[draw=green!30,fill=green!30,angle radius=0.5cm] {angle=D--F--E};
  \draw pic[draw=red!30,fill=red!30,angle radius=0.5cm] {angle=B--A--C};
  \draw pic[draw=red!30,fill=red!30,angle radius=0.5cm] {angle=F--E--D};
  \draw pic[draw=blue!30,fill=blue!30,angle radius=0.5cm] {angle=C--B--A};
  \draw pic[draw=blue!30,fill=blue!30,angle radius=0.5cm] {angle=E--D--F};

    \tkzMarkSegment[pos=.5,mark=|](C,B)
    \tkzMarkSegment[pos=.5,mark=|](D,F)
    \tkzMarkSegment[pos=.5,mark=||](C,A)
    \tkzMarkSegment[pos=.5,mark=||](E,F)
    \tkzMarkSegment[pos=.5,mark=|||](B,A)
    \tkzMarkSegment[pos=.5,mark=|||](E,D)
\end{tikzpicture}
\end{center}

Sedangkan $ABC$ dan $DEF$ dikatakan kongruen atau $\triangle ABC \cong \triangle DEF$ jika dan hanya jika $AB=DE, BC=EF, CA=FD$ atau dengan kata lain kedua segitiga tersebut sebangun dan ada salah satu sisi dari kedua segitiga tersebut yang panjangnya sama. Simpelnya kongruen = sama persis.

\subsection{Latihan Soal Segitiga }
\begin{enumerate}    
    \item Garis berat $AD$ pada segitiga $ABC$ memotong garis berat $CF$ di titik $P$, serta perpanjangan $BP$ memotong $AC$ di $E$. Jika diketahui segitiga $ABC$ lancip dan $AB=6$, maka panjang $DE$ adalah \dots

    \item (OSK 2011,2012,2013,2018) Diberikan segitiga $ABC$ dan lingkaran $\Gamma$ yang berdiameter $AB$. Lingkaran $\Gamma$ memotong sisi $AC$ dan $BC$ berturut-turut di titik $D$ dan $E$. Jika $AD = \frac13 AC, BE =\frac14 BC$ dan $AB = 30$, maka luas segitiga $ABC$ adalah \dots
		
    \item Diberikan segitiga $ABC$ dengan $D$ titik tengah $AC$, $E$ titik tengah $BD$, dan $H$ merupakan pencerminan $A$ terhadap $E$. Jika $F$ merupakan perpotongan antara $AH$ dengan $BC$, maka nilai $\dfrac{AF}{FH}$ sama dengan \dots
		 
    \item Diberikan segitiga $ABC$ dengan panjang sisi $BC = 20$, $CA = 24$, dan $AB=12$. Titik $D$ pada segmen $BC$ dengan $BD = 5$. Lingkaran luar dari segitiga $ABD$ memotong $CA$ di $E$. Hitunglah nilai $2 \times DE$.

    \item (OSK 2015) Diberikan trapesium $ABCD$ dengan $AB$ sejajar $DC$ dan $AB = 84$ serta $DC = 25$. Jika trapesium $ABCD$ memiliki lingkaran dalam yang menyinggung keempat sisinya, keliling trapesium $ABCD$ adalah \ldots

    \item (OSK 2022) Diberikan segitiga siku-siku $ABC$. Jika luas dari segitiga $ABC$ adalah 112. Misalkan $R$ adalah panjang jari-jari lingkaran luar segitiga $ABC$ dan $r$ adalah panjang jari-jari lingkaran dalam segitiga $ABC$. Diketahui juga $R + r = 16$. Panjang sisi miring dari segitiga $ABC$ adalah \ldots
\end{enumerate}


\subsection{Latihan Soal Segitiga }
\begin{enumerate}    
    \item Garis berat $AD$ pada segitiga $ABC$ memotong garis berat $CF$ di titik $P$, serta perpanjangan $BP$ memotong $AC$ di $E$. Jika diketahui segitiga $ABC$ lancip dan $AB=6$, maka panjang $DE$ adalah \dots

    \item (OSK 2011,2012,2013,2018) Diberikan segitiga $ABC$ dan lingkaran $\Gamma$ yang berdiameter $AB$. Lingkaran $\Gamma$ memotong sisi $AC$ dan $BC$ berturut-turut di titik $D$ dan $E$. Jika $AD = \frac13 AC, BE =\frac14 BC$ dan $AB = 30$, maka luas segitiga $ABC$ adalah \dots
		
    \item Diberikan segitiga $ABC$ dengan $D$ titik tengah $AC$, $E$ titik tengah $BD$, dan $H$ merupakan pencerminan $A$ terhadap $E$. Jika $F$ merupakan perpotongan antara $AH$ dengan $BC$, maka nilai $\dfrac{AF}{FH}$ sama dengan \dots
		 
    \item Diberikan segitiga $ABC$ dengan panjang sisi $BC = 20$, $CA = 24$, dan $AB=12$. Titik $D$ pada segmen $BC$ dengan $BD = 5$. Lingkaran luar dari segitiga $ABD$ memotong $CA$ di $E$. Hitunglah nilai $2 \times DE$.

    \item (OSK 2015) Diberikan trapesium $ABCD$ dengan $AB$ sejajar $DC$ dan $AB = 84$ serta $DC = 25$. Jika trapesium $ABCD$ memiliki lingkaran dalam yang menyinggung keempat sisinya, keliling trapesium $ABCD$ adalah \ldots

    \item (OSK 2022) Diberikan segitiga siku-siku $ABC$. Jika luas dari segitiga $ABC$ adalah 112. Misalkan $R$ adalah panjang jari-jari lingkaran luar segitiga $ABC$ dan $r$ adalah panjang jari-jari lingkaran dalam segitiga $ABC$. Diketahui juga $R + r = 16$. Panjang sisi miring dari segitiga $ABC$ adalah \ldots
\end{enumerate}