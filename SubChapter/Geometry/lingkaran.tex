\subsection{Lingkaran}

\begin{center}
\begin{tikzpicture}
\coordinate (O) at (0,0);
\coordinate (B) at (0:2cm);
\coordinate (A) at (210:2cm);
\coordinate (D) at (180:2cm);
\coordinate (C) at (130:2cm);
\coordinate (E) at (-20:2cm);

\draw (O) circle (2cm);

\draw (A) -- (B) -- (C) -- (D) -- cycle;
\draw (A) -- (E) -- (C);
\draw[red] (A) -- (C);

\tkzDefPointBy[projection=onto A--C](O) \tkzGetPoint{F}
\draw[orange] (O) -- (F);
\draw[blue] (C) -- (O) -- (A);

\node[right] at (B) {$B$};
\node[below left] at (A) {$A$};
\node[left] at (D) {$D$};
\node[above left] at (C) {$C$};
\node[below right] at (E) {$E$};
\node[right] at (O) {$O$};
\node[above right] at (F) {$F$};
\end{tikzpicture}
\end{center}

    
Misalkan $O$ pusat lingkaran $\Gamma$ dan $A,B,C,D,E$ adalah sembarang titik pada lingkaran $\Gamma$ seperti pada gambar.
\begin{enumerate}
    \item $CO=OA$ adalah jari-jari dengan $\angle ACO = \angle OAC$.
    \item Misalkan titik $F$ adalah titik tengah tali busur $CA$, maka $OF \perp CA$ atau $OF$ tegak lurus dengan $CA$, dengan kata lain, $F$ adalah proyeksi titik $O$ ke $CA$
    \item (Sudut keliling-sudut pusat) Untuk$\angle COA = 2\angle CBA$.
    \item (sudut keliling) $\angle CBA = \angle CEA$.
    \item $ABCD$ adalah segiempat tali busur atau segiempat siklis  atau $A,B,C,D$ terletak di lingkaran (seperti pada gambar) jika dan hanya jika $\angle CBA + \angle ADC = 180^\circ$ atau $\angle ABD = \angle ACD$.
\end{enumerate}