\subsubsection{Rumus Barisan dan Deret Lainnya}
Sangat dianjurkan untuk mencoba membuktikan rumus-rumus berikut. Untuk bilangan asli $n$, kita punya
\begin{enumerate}
    \item $1+2+\dots+n = \dfrac{n(n+1)}{2}.$
    \item $1^2+2^2+\dots+n^2 = \dfrac{n(n+1)(2n+1)}{6}.$
    \item $1^3+2^3+\dots+n^3 = \left(1+2+\dots+n\right)^2= \left(\dfrac{n(n+1)}{2}\right)^2.$
\end{enumerate}

\subsubsection{Prinsip Teleskopik}
Pernahkah kalian melihat deret yang dapat saling menghilangkan atau bisa "dicoret"? Berikut beberapa bentuk umumnya. Untuk lebih banyak contoh, silakan lihat contoh soal.

\begin{enumerate}
    \item $\sum_{i=1}^{n} (a_{i+1}-a_{i}) = (a_2-a_1)+(a_3-a_2)+\dots+(a_{n+1}-a_{n}) = a_{n+1}-a_1.$
    \item $\prod_{i=1}^{n} \dfrac{a_{i+1}}{a_i} =  \dfrac{a_2}{a_1}\cdot\dfrac{a_3}{a_2}\cdot\dfrac{a_4}{a_3}\cdot\ldots\cdot\dfrac{a_{n+1}}{a_n} = \dfrac{a_{n+1}}{a_1}.$
\end{enumerate}