\subsubsection{Cauchy-Schwarz}
Untuk bilangan real $a_1,a_2,\dots,a_n$ dan $b_1,b_2,\dots,b_n$ berlaku
$$(a_1^2+a_2^2+\dots+a_n^2)(b_1^2+b_2^2+\dots+b_n^2) \ge (a_1b_1+a_2b_2+\dots+a_nb_n)^2$$

dengan kesamaan terjadi jika dan hanya jika $\dfrac{a_1}{b_1}=\dfrac{a_2}{b_2}=\dots =\dfrac{a_n}{b_n}$.

\subsubsection{Ketaksamaan Bernoulli}
Untuk $x > -1$ berlaku $(1+x)^n \ge 1+nx$.

\subsection{Ketaksamaan Kuadrat}
Untuk $x \in \RR$, berlaku kuadrat sempurnanya selalu nonnegatif atau $x^2 \ge 0$. Hal ini menyebabkan fungsi kuadrat $f(x)=ax^2+bx+c$ untuk $a \neq 0$ selalu mempunyai nilai minimum atau maksimum saat $x = -\dfrac{b}{2a}$. (why?)