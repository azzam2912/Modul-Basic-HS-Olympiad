\subsubsection{QM-AM-GM-HM}
Untuk bilangan real positif $a_1,a_2,\dots,a_n$ dengan $n\ge 2$,definisikan
\begin{align*}
    QM \text{ (Quadratic Mean) } &= \sqrt{\dfrac{a_1^2+a_2^2+\dots+a_n^2}{n}}\\
    AM \text{ (Arithmetic Mean) } &= \dfrac{a_1+a_2+\dots+a_n}{n}\\
    GM \text{ (Geometric Mean) } &=
    \sqrt[n]{a_1a_2\dots a_n}\\
    HM \text{ (Harmonic Mean) } &=
    \dfrac{n}{\dfrac{1}{a_1}+\dfrac{1}{a_2}+\dots+\dfrac{1}{a_n}}
\end{align*}

Maka berlaku $QM \ge AM \ge GM \ge HM$ atau 
$$\sqrt{\dfrac{a_1^2+a_2^2+\dots+a_n^2}{n}} \ge  \dfrac{a_1+a_2+\dots+a_n}{n}\ge
    \sqrt[n]{a_1a_2\dots a_n} \ge
    \dfrac{n}{\dfrac{1}{a_1}+\dfrac{1}{a_2}+\dots+\dfrac{1}{a_n}}$$
dengan kesamaan terjadi jika dan hanya jika $a_1=a_2=\dots =a_n$.