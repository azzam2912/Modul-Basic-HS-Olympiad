\subsection{Polinomial / Suku banyak}
Suatu polinomial real $P(x)$ yang mempunyai derajat $n$ atau $deg(P) = n$ untuk suatu bilangan bulat non negatif $n$ dinyatakan sebagai
$$P(x)=a_nx^n+a_{n-1}x^{n-1}+\dots+a_ax+a_0$$
untuk suatu bilangan real $a_n,a_{n-1},\dots,a_0$ dimana $a_n \neq 0$. Dari teorema fundamental aljabar, setiap polinomial $P(x)$ tersebut memiliki maksimal $n$ akar kompleks dan dapat dinyatakan dalam bentuk
$$P(x)=a(x-x_1)(x-x_2)\dots(x-x_n)$$
dimana $a = a_n \neq 0$ dan $x_1,x_2,\dots,x_n$ adalah penyelesaian atau akar-akar dari persamaan $P(x)=0$ atau $a(x-x_1)(x-x_2)\dots(x-x_n)=0$.

Catatan: Polinomial suku-sukunya harus memiliki pangkat positif sehingga $P(x)=x^4+5x+\sqrt{2}$ adalah suatu polinomial, tetapi $P(x)=x^3+5x^2+\dfrac{5}{x^4}+\dfrac{1}{x^8}$ bukan polinomial (karena $\dfrac{1}{x^8}$ dan $\dfrac{5}{x^4}$ bukan suku yang memiliki pangkat positif.

\subsubsection{Remainder Theorem}
Remainder Theorem = Teorema Sisa.
Sisa polinomial $P(x)$ saat dibagi $(x-a)$ adalah $P(a)$ (why?).
Dengan kata lain kita dapat membentuk polinomial $P(x)$ menjadi
$$P(x)=Q(x)(x-a)+P(a)$$
untuk suatu polinomial tak nol $Q(x)$ dengan $deg(Q)<deg(P)$.
\subsubsection{Factor Theorem}
Factor Theorem = Teorema Faktor.
Hasil bagi polinomial $P(x)$ saat dibagi $(x-a)$ adalah $0$ atau $P(a)=0$ jika dan hanya jika $(x-a)$ adalah faktor dari $P(x)$ (why?).
Dengan kata lain kita dapat membentuk polinomial $P(x)$ menjadi
$$P(x)=Q(x)(x-a)$$
untuk suatu polinomial tak nol $Q(x)$ dengan $deg(Q)<deg(P)$.
\subsubsection{Teorema Vieta}
\begin{itemize}
    \item $x_1+x_2+\dots+x_n=-\dfrac{a_{n-1}}{a_n}$
    \item $x_1x_2+x_1x_3+\dots+x_{n-1}x_n=\dfrac{a_{n-2}}{a_n}$
    \item \dots
    \item $x_1x_2x_3\ldots x_n = (-1)^{n-1}\dfrac{a_{0}}{a_n}$
\end{itemize}

Contoh:
\begin{enumerate}
\item Untuk $ax^3+bx^2+cx+d=0$ kita punya $x_1+x_2+x_3=-\dfrac{b}{a}$, $x_1x_2+x_1x_3+x_2x_3=\dfrac{c}{a}$, dan $x_1x_2x_3=-\dfrac{d}{a}$.
\item Untuk $ax^2+bx+c=0$ kita punya $x_1+x_2=-\dfrac{b}{a}$ dan $x_1x_2=\dfrac{c}{a}$.
\end{enumerate}
\subsubsection{Fungsi Kuadrat}
Fungsi kuadrat $P(x)=ax^2+bx+c$ termasuk salah satu polinomial yang memiliki derajat 2 dan mempunyai beberapa properti penting:
\begin{enumerate}
\item Diskriminan $D=b^2-4ac$. Hal-hal berikut adalah \textbf{akibat langsung dari rumus abc di bawah}: $D>0$ jika dan hanya jika $P(x)$ mempunyai 2 akar real berbeda,  $D=0$ jika dan hanya jika $P(x)$ mempunyai 2 akar real kembar, dan $D<0$ jika dan hanya jika $P(x)$ mempunyai akar kompleks tidak real.
\item Rumus abc / penyelesaian umum: $x_{1,2} = \dfrac{-b \pm \sqrt{b^2-4ac}}{2a} = \dfrac{-b+\sqrt{D}}{2a}$. Perhatikan dari rumus ini dapat dibuktikan juga rumus Vieta yang sebelumnya ditunjukkan.
\end{enumerate}

\subsection{Latihan Soal Polinomial}
\begin{enumerate}

\item Jika $P(x)$ dibagi $x^2-x$ dan $x^2+x$ berturut-turut akan bersisa $5x+1$ dan $3x+1$, maka bila $P(x)$ dibagi $x^2-1$ sisanya adalah \dots

\item (OSP 2006) Jika $(x-1)^2$ membagi $ax^4+bx^3+1$, maka $ab=\dots$

\item Diketahui suatu polinomial $P(x)$ memenuhi $P(k)=\dfrac{k}{k+1}$ untuk $k=1,2,3,\dots,2020$. Jika $P(0)=1$, nilai $P(2022)=\dots$

\item (OSK 2010) Polinom $P(x)=x^3-x^2+x-2$ mempunyai tiga pembuat nol yaitu $a,b,$ dan $c$. Nilai dari $a^3+b^3+c^3$ adalah \dots

\item (OSP 2010) Persamaan kuadrat $x^2-px-2p=0$ mempunyai dua akar real $a$ dan $b$. Jika $a^3+b^3=16$, maka hasil jumlah semua nilai $p$ yang memenuhi adalah \dots 

\item (OSP 2010) Diberikan polinomial $P(x)=x^4+ax^3+bx^2+cx+d$ dengan $a,b,c,$ dan $d$ konstanta. Jika $P(1)=10$, $P(2)=20$, dan  $P(3)=30$, maka nilai
$$\dfrac{P(12)+P(-8)}{10}=\dots$$

\item (OSK 2015) Diketahui $a$, $b$, $c$ akar-akar dari persamaan $x^3 - 5x^2 - 9x + 10 = 0$. Jika suku banyak $P(x) = Ax^3 + Bx^2 + Cx - 2015$ memenuhi $P(a) = b + c$, $P(b) = a + c$ dan $P(c) = a + b$ maka nilai dari $A + B + C$ adalah \ldots

\end{enumerate}