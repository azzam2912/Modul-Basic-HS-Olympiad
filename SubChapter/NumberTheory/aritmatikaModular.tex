\subsection{Aritmatika Modular}
    Untuk suatu bilangan asli $m$ dan bilangan bulat $a,b,c$ dan $d$, notasikan $m\mid a-b \iff a \equiv b \mod m$ (dibaca $a$ kongruen $b$ modulo $m$). Simpelnya $a \equiv b \mod m$ adalah $a$ dibagi $m$ bersisa $b$. Contohnya $5 \equiv 2 \mod 3$. $13 \equiv 3 \mod 5$. $10 \equiv -2 \mod 12$.
    \begin{enumerate}
        \item $a \equiv a \mod m$.
        \item $a \equiv 0 \mod m \iff m\mid a$.
        \item $a \equiv b \mod m \iff b \equiv a \mod m$.
        \item $a \equiv b \mod m \text{ dan } b \equiv c \mod m \implies a \equiv c \mod m$.
        \item Jika $a \equiv b \mod m$ dan $d\mid m$ maka $a \equiv b \mod d$.
        \item Untuk semua bilangan asli $k$, $a \equiv b \mod m \iff a^k \equiv b^k \mod m$.
        \item $a \equiv b \mod m \text{ dan } c \equiv d \mod m \implies a+c \equiv b+d \mod m$.
        \item $a \equiv b \mod m \text{ dan } c \equiv d \mod m \implies a-c \equiv b-d \mod m$.
        \item $a \equiv b \mod m \text{ dan } c \equiv d \mod m \implies ac \equiv bd \mod m$.
        \item $\forall k\in \ZZ^+, (am+b)^k \equiv b^k \mod m$.
        \item Jika $ca \equiv cb \mod m$ dengan $FPB(c,m)=1$, maka $a \equiv b \mod m$.
    \end{enumerate}
    
    Catatan: Penggunaan sifat nomor 8 dapat dimodifikasi sehingga menjadi konsep \textbf{Chinese Remainder Theorem}.