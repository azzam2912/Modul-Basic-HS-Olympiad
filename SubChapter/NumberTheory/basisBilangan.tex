\subsection{Basis Bilangan}
Basis bilangan adalah sistem bilangan yang menyatakan banyaknya digit atau kombinasi dari digit-digit yang menyatakan sebuah bilangan. Secara matematis, bilangan $a$ dalam basis $n > 0$ yaitu $(a)_n$ mempunyai bentuk (yang setara dengan nilai basis 10):
$$(c_kc_{k-1}\dotsc_1c_0)_n = c_{k}n^k + c_{k-1}n^{k-1}+\dots+c_1n^{1}+c_0n^{0}$$

Secara umum bahkan kita telah memakai sistem basis tersebut untuk basis 10. Misalkan 123 dapat dinyatakan sebagai $123 = 1\cdot 10^2 + 2\cdot 10^1 + 1\cdot 10^0$

Lalu, berikut merupakan contoh untuk bilangan basis selain 10 misalnya: 
\begin{itemize}
    \item Bilangan basis 2 atau bilangan biner yang digit-digitnya terdiri dari $\{0,1\}$. Misalkan $1001_2$ dalam biner yang setara dengan $9$ atau $1001_2 = 9$ karena $1001_2 = 1\cdot 2^3+0\cdot 2^2+0\cdot 2^1+1\cdot 2^0 = 9$. 
    \item Bilangan basis 3 yang digit-digitnya terdiri dari $\{0,1,2\}$. Misalkan $211_3 = 22$ karena $211_3 = 2\cdot 3^2+ 1\cdot 3^1+ 1\cdot 3^0 = 22$.
    \item Bilangan basis 16 atau heksadesimal yang digit-digitnya terdiri dari $\{0,1,2,\dots,9,A,B,\dots,F\}$. Misalkan $5F_{16} = 95$ karena $5F_{16} = 5 \cdot 16^1 + (15)\cdot 16^0 = 95$.
\end{itemize}

\subsection{Basis Bilangan}
Basis bilangan adalah sistem bilangan yang menyatakan banyaknya digit atau kombinasi dari digit-digit yang menyatakan sebuah bilangan. Secara matematis, bilangan $a$ dalam basis $n > 0$ yaitu $(a)_n$ mempunyai bentuk (yang setara dengan nilai basis 10):
$$(c_kc_{k-1}\dotsc_1c_0)_n = c_{k}n^k + c_{k-1}n^{k-1}+\dots+c_1n^{1}+c_0n^{0}$$

Secara umum bahkan kita telah memakai sistem basis tersebut untuk basis 10. Misalkan 123 dapat dinyatakan sebagai $123 = 1\cdot 10^2 + 2\cdot 10^1 + 1\cdot 10^0$

Lalu, berikut merupakan contoh untuk bilangan basis selain 10 misalnya: 
\begin{itemize}
    \item Bilangan basis 2 atau bilangan biner yang digit-digitnya terdiri dari $\{0,1\}$. Misalkan $1001_2$ dalam biner yang setara dengan $9$ atau $1001_2 = 9$ karena $1001_2 = 1\cdot 2^3+0\cdot 2^2+0\cdot 2^1+1\cdot 2^0 = 9$. 
    \item Bilangan basis 3 yang digit-digitnya terdiri dari $\{0,1,2\}$. Misalkan $211_3 = 22$ karena $211_3 = 2\cdot 3^2+ 1\cdot 3^1+ 1\cdot 3^0 = 22$.
    \item Bilangan basis 16 atau heksadesimal yang digit-digitnya terdiri dari $\{0,1,2,\dots,9,A,B,\dots,F\}$. Misalkan $5F_{16} = 95$ karena $5F_{16} = 5 \cdot 16^1 + (15)\cdot 16^0 = 95$.
\end{itemize}

\subsection{Basis Bilangan}
Basis bilangan adalah sistem bilangan yang menyatakan banyaknya digit atau kombinasi dari digit-digit yang menyatakan sebuah bilangan. Secara matematis, bilangan $a$ dalam basis $n > 0$ yaitu $(a)_n$ mempunyai bentuk (yang setara dengan nilai basis 10):
$$(c_kc_{k-1}\dotsc_1c_0)_n = c_{k}n^k + c_{k-1}n^{k-1}+\dots+c_1n^{1}+c_0n^{0}$$

Secara umum bahkan kita telah memakai sistem basis tersebut untuk basis 10. Misalkan 123 dapat dinyatakan sebagai $123 = 1\cdot 10^2 + 2\cdot 10^1 + 1\cdot 10^0$

Lalu, berikut merupakan contoh untuk bilangan basis selain 10 misalnya: 
\begin{itemize}
    \item Bilangan basis 2 atau bilangan biner yang digit-digitnya terdiri dari $\{0,1\}$. Misalkan $1001_2$ dalam biner yang setara dengan $9$ atau $1001_2 = 9$ karena $1001_2 = 1\cdot 2^3+0\cdot 2^2+0\cdot 2^1+1\cdot 2^0 = 9$. 
    \item Bilangan basis 3 yang digit-digitnya terdiri dari $\{0,1,2\}$. Misalkan $211_3 = 22$ karena $211_3 = 2\cdot 3^2+ 1\cdot 3^1+ 1\cdot 3^0 = 22$.
    \item Bilangan basis 16 atau heksadesimal yang digit-digitnya terdiri dari $\{0,1,2,\dots,9,A,B,\dots,F\}$. Misalkan $5F_{16} = 95$ karena $5F_{16} = 5 \cdot 16^1 + (15)\cdot 16^0 = 95$.
\end{itemize}

\subsection{Basis Bilangan}
Basis bilangan adalah sistem bilangan yang menyatakan banyaknya digit atau kombinasi dari digit-digit yang menyatakan sebuah bilangan. Secara matematis, bilangan $a$ dalam basis $n > 0$ yaitu $(a)_n$ mempunyai bentuk (yang setara dengan nilai basis 10):
$$(c_kc_{k-1}\dotsc_1c_0)_n = c_{k}n^k + c_{k-1}n^{k-1}+\dots+c_1n^{1}+c_0n^{0}$$

Secara umum bahkan kita telah memakai sistem basis tersebut untuk basis 10. Misalkan 123 dapat dinyatakan sebagai $123 = 1\cdot 10^2 + 2\cdot 10^1 + 1\cdot 10^0$

Lalu, berikut merupakan contoh untuk bilangan basis selain 10 misalnya: 
\begin{itemize}
    \item Bilangan basis 2 atau bilangan biner yang digit-digitnya terdiri dari $\{0,1\}$. Misalkan $1001_2$ dalam biner yang setara dengan $9$ atau $1001_2 = 9$ karena $1001_2 = 1\cdot 2^3+0\cdot 2^2+0\cdot 2^1+1\cdot 2^0 = 9$. 
    \item Bilangan basis 3 yang digit-digitnya terdiri dari $\{0,1,2\}$. Misalkan $211_3 = 22$ karena $211_3 = 2\cdot 3^2+ 1\cdot 3^1+ 1\cdot 3^0 = 22$.
    \item Bilangan basis 16 atau heksadesimal yang digit-digitnya terdiri dari $\{0,1,2,\dots,9,A,B,\dots,F\}$. Misalkan $5F_{16} = 95$ karena $5F_{16} = 5 \cdot 16^1 + (15)\cdot 16^0 = 95$.
\end{itemize}

\input{Soal/NumberTheory/basisBilangan}