\subsection{Keterbagian}
    Untuk bilangan bulat $a \neq 0$ serta bilangan bulat $b,c,x$ dan $y$, notasikan $a \mid b$ sebagai $a$ membagi $b$. Lalu, $a$ dan $b$ relatif prima atau $a$ dan $b$ koprima (coprime) jika dan hanya jika $FPB(a,b)=1$.
    \begin{enumerate}
        \item Kita dapat menyatakan semua bilangan bulat $c = pq+r$ untuk suatu bilangan bulat $q$ dimana $0 \le r < q$. Jadi, saat $c$ dibagi $p$, maka hasil baginya adalah $q$ dan sisa baginya adalah $r$.
        \item Terdapat suatu bilangan bulat $x$ dimana $a \mid b \iff b=ax$.
        \item $a \mid a$.
        \item $a \mid 0$.
        \item $1 \mid a$.
        \item $a \mid b \implies a \mid bc$.
        \item Untuk $a,b \neq 0$ maka $ab \mid c \implies a \mid c \text{ dan } b \mid c$.
        \item $a \mid b \text{ dan } b \mid c \implies a \mid c$.
        \item $a \mid b \text{ dan } a \mid c \implies a \mid bx + cy$.
        \item $a \mid b \text{ dan } a \mid c \implies a \mid b+c$.
        \item $a \mid b \text{ dan } a \mid c \implies a \mid b-c$.
        \item Untuk $x \neq 0$ maka $a \mid b \iff xa \mid xb$.
        \item $a \mid b$ dan $b \neq 0$ maka $|a| \le |b|$.
        \item $a \mid bc$ dan $FPB(a,b)=1$ maka $a\mid c$.
    \end{enumerate}


    