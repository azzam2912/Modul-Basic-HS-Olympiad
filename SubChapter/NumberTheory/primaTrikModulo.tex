\subsection{Bilangan Prima Serta Trik-trik Modulo Umum}
\begin{enumerate}
    \item Bilangan prima adalah bilangan asli yang hanya dapat dibagi dirinya sendiri dan angka 1. 
    \item Bilangan bukan prima dan bukan 1 disebut bilangan komposit.
    \item 1 bukan bilangan prima dan bukan pula bilangan komposit. 
\end{enumerate}

Untuk bilangan prima $p$ dan bilangan bulat $n$.
\begin{enumerate}
    \item Bilangan prima genap hanya ada satu buah, yaitu 2.
    \item Dari definisi bilangan prima $p$, karena $p$ tak terbagi oleh $2$ dan $5$, maka tak ada bilangan prima yang berakhiran $0$.
    \item Untuk sembarang bilangan prima $p$ berlaku $p \mid n$ atau $gcd(p,n)=1$.
    \item $p \mid n^2$ jika dan hanya jika $p \mid n$.
    \item $p \mid ab \iff p \mid a \text{ atau } p \mid b$.
    \item Untuk $p > 3$, kita punya bentuk $p = 6k \pm 1$ untuk suatu bilangan asli $k$.
    \item (Sieve of Erastosthenes) 
    Faktor prima terkecil $t$ dari bilangan komposit $n$ selalu $t \le \sqrt{n}$.
\end{enumerate}
    
Lalu, beberapa trik-trik modulo umum:
\begin{enumerate}
    \item Pada sistem persamaan bulat, tinjau modulo 3, 4, 5, 7, atau modulo 11 nya.
    \item Untuk bilangan bulat $n$ selalu terjadi $n^2 \equiv 1 \mod 4$, $n^2 \equiv 1 \mod 3$. Peninjauan terhadap modulo lain juga bisa, namun tidak terlalu umum.
\end{enumerate}

\subsection{Latihan Soal Trik Bilangan Prima dan Modulo}
\begin{enumerate}
        \item (OSK 2013) Diketahui $x_1,x_2$ adalah dua bilangan bulat berbeda yang merupakan akar-akar dari persamaan kuadrat $x^2+px+q+1=0$. Jika $p$ dan $p^2+q^2$ adalah bilangan-bilangan prima, maka nilai terbesar yang mungkin dari $x_1^{2013}+x_2^{2013}$ adalah \dots
        
        \item (OSK 2014) Diberikan tiga bilangan bulat positif berurutan. Jika bilangan pertama tetap, bilangan kedua ditambah 10 dan bilangan ketiga ditambah bilangan prima, maka ketiga bilangan ini membentuk deret ukur. Bilangan ketiga dari bilangan bulat berurutan adalah \dots
        
        \item (OSK 2014) Semua pasangan bilangan prima $(p,q)$ yang memenuhi persamaan
        $$(7p-q)^2=2(p-1)q^2$$
        adalah \dots
        
        \item (OSK 2014) Semua bilangan bulat $n$ sehingga $n^4-51n^2+225$ merupakan bilangan prima adalah \dots

        \item (OSK 2015) Banyaknya bilangan asli $n \leq 2015$ yang dapat dinyatakan dalam bentuk $n = a + b$ dengan $a$, $b$ bilangan asli yang memenuhi $a - b$ bilangan prima dan $ab$ bilangan kuadrat sempurna adalah \ldots

        
        \item (OSK 2023) Jika bilangan asli $x$ dan $y$ memenuhi persamaan
        $$x(x-y)=5y-6,$$
        maka $x+y=\ldots$

        \item (OSK 2020) Misalkan $n \geq 2$ adalah bilangan asli sedemikian sehingga untuk setiap bilangan asli $a$, $b$ dengan $a + b = n$ berlaku $a^2 + b^2$ merupakan bilangan prima. Hasil penjumlahan semua bilangan asli $n$ semacam itu adalah \ldots
\end{enumerate}