\subsubsection{Banyaknya Bilangan Relatif Prima}
\begin{remark*}
         Bilangan $b$ dikatakan relatif prima dengan $a$ jika dan hanya jika $FPB(a,b)=1$.
    \end{remark*}
 Definisikan fungsi \textbf{Euler Totient Phi $\phi(n)$ sebagai banyaknya bilangan bulat positif $b$ yang kurang dari sama dengan $n$ dimana $n$ relatif prima dengan $b$}. Rumus eksplisit untuk menghitung fungsi ini adalah
$$\phi(n) = n\left(1-\dfrac{1}{p_1}\right)\left(1-\dfrac{1}{p_2}\right)\dots\left(1-\dfrac{1}{p_n}\right).$$

Contoh: $\phi(4)=4(1-\frac{1}{2})=2$ karena ada 2 bilangan yang realtif prima dengan 4, yaitu 1 dan 3.

Catatan: Untuk semua bilangan prima $p$, nilai $\phi(p) = p-1$. (Silakan dibuktikan sendiri :D)