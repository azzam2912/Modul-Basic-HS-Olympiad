\subsection{FPB dan KPK dua bilangan}
Secara matematis FPB (gcd - greatest common divisors) dan KPK (lcm - least common multiples) dari dua bilangan bulat positif $a$ dan $b$ 
didefinisikan sebagai:
$$FPB(a,b) = gcd(a,b) = p_1^{\min\{a_1,b_1\}}\cdot p_2^{\min\{a_2,b_2\}} \cdot \ldots \cdot p_n^{\min\{a_n,b_n\}}$$
$$KPK(a,b) = lcm(a,b) =p_1^{\max\{a_1,b_1\}}\cdot p_2^{\max\{a_2,b_2\}} \cdot \ldots \cdot p_n^{\max\{a_n,b_n\}}$$
dimana kedua bilangan tersebut dapat difaktorisasi prima menjadi
$a=p_1^{a_1}\cdot p_2^{a_2}\cdot \ldots \cdot p_n^{a_n}$ dan $b=p_1^{b_1}\cdot p_2^{b_2} \cdot \ldots \cdot p_n^{b_n}$, dengan $p_1,p_2,\dots,p_n$ adalah bilangan prima berbeda, serta $a_1,a_2,\dots,a_n,b_1,b_2,\dots,b_n$ adalah bilangan bulat non-negatif.

Dari definisi tersebut mudah dibuktikan bahwa
$$FPB(a,b) \cdot KPK(a,b) = ab.$$

Perlu dicatat, bahwa $a$ dan $b$ tidak boleh bernilai nol. Untuk $a,b$ yang bernilai negatif, didefinisikan $FPB(a,b) = FPB(|a|,|b|)$.

\subsubsection{Algoritma Euclid}
Pada dasarnya algoritma ini bertumpu pada sebuah teorema:
$$FPB(a,b) = FPB(a,b-a) = FPB(a-b,b)=FPB(a, b \mod a)$$

\subsection{Latihan Soal FPB dan KPK}
\begin{enumerate}
    \item (OSK 2011) Bilangan asli terkecil lebih dari 2011 yang bersisa 1 jika dibagi 2,3,4,5,6,7,8,9,10 adalah \dots
    
    \item (OSK 2009) Nilai dari $\sum_{k=1}^{2009} FPB(k,7)$ adalah \dots
    
    \item Banyaknya anggota himpunan himpunan 
    $$S = \{gcd(n^3+1,n^2+3n+9 \mid n \in Z\}$$
    adalah \dots
    
    \item (AIME 1998) Ada berapa banyak bilangan bulat positif $k$ sehingga $lcm(6^6,8^8,k)=12^{12}$?
    
    \item Jika $a,b$ adalah bilangan asli dan $c$ adalah bilangan bulat, buktikan bahwa $gcd(a,b)=gcd(a,b+ac).$
\end{enumerate}