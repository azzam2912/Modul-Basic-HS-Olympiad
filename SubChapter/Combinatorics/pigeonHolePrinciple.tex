\subsection{Pigeon Hole Principle (PHP)}
Teorema yang dalam Bahasa Indonesia ini disebut dengan Teorema Sangkar Burung Merpati secara matematis berbunyi:
Jika ada $kn+1$ merpati dan $n$ sangkar, maka setidaknya ada satu sangkar yang berisi $k+1$ burung merpati.

Versi lebih simpelnya adalah: jika ada $n+1$ objek yang akan dibagi ke dalam $n$ buah kotak, maka setidaknya ada 1 kotak yang berisi 2 objek.

Contoh: \begin{itemize}
    \item Di dalam ruangan berisi 3 orang, pasti terdapat setidaknya 2 orang berjenis kelamin sama.
    \item Jika ada 367 orang di suatu sekolah, maka setidaknya ada dua orang diantara mereka yang tanggal lahirnya persis sama.
\end{itemize}

\subsection{Latihan Soal PigeonHole Principle}
\begin{enumerate}
\item Berapa banyak orang minimum yang harus hadir di suatu pesta sehingga dipastikan terdapat 3 orang yang lahir di bulan yang sama di pesta itu?

\item Misalkan Naruko memilih $k$ buah bilangan dari himpunan $\{1,2,3,\dots,2016\}$ secara acak. Berapakah nilai $k$ terkecil sehingga Naruko pasti bisa mendapatkan setidaknya sepasang bilangan (dari $k$ bilangan itu) yang jika dijumlahkan hasilnya 2017?

\item Suatu malam di rumah WonYoung terjadi pemadaman listrik. Karena WonYoung sangat malas, ia hanya ingin tidur dengan membawa banyak kaus kaki (hobi yang aneh :/). Ia mengambil kaus kaki dari lemari di ruangan yang sangat gelap. Lemari itu berisi 100 buah kaus kaki merah, 80 kaus kaki hijau, 60 kaus kaki biru, dan 40 kaus kaki hitam. WonYoung mengambil banyak kaus kaki tapi tidak bisa tahu warnanya. Berapa banyak kaus kaki paling sedikit yang perlu diambil sehingga dijamin terdapat setidaknya 10 pasang kaus kaki (dengan setiap pasang kaus kaki harus berwarna sama) ?

\item (OSK 2011) Di lemari hanya ada 2 macam kaos kaki yaitu kaos kaki berwarna hitam dan putih. Ali, Budi dan Candra berangkat di malam hari saat mati lampu dan mereka mengambil kaos kaki secara acak di dalam lemari dalam kegelapan. Berapa kaos kaki minimal harus mereka ambil untuk memastikan bahwa akan ada tiga pasang kaos kaki yang bisa mereka pakai ? (Sepasang kaos kaki harus memiliki warna yang sama).
    
\item Tandai satu buah kartu dengan angka 1, dua buah kartu dengan angka 2, tiga buah kartu dengan angka satu hingga lima puluh buah kartu dengan angka 50. Semua kartu tersebut dimasukkan ke dalam kotak. Berapa buah kartu minimal yang harus diambil agar dapat dipastikan terdapat sekurang-kurangnya 10 buah kartu dengan tanda angka yang sama 

\item (OSK 2016) Anak laki-laki dan anak perempuan yang berjumlah 48 orang duduk melingkar secara acak. Banyaknya minimum anak perempuan sehingga pasti ada enam anak perempuan yang duduk berdekatan tanpa diselingi anak laki-laki adalah \dots
\end{enumerate}