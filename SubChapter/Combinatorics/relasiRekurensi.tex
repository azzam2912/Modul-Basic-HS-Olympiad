\subsection{Relasi Rekurensi}
Sering disebut dengan rekursif. Intinya adalah sebuah persamaan yang melibatkan barisan $a_1, a_2, \dots , a_n$ dimana untuk mendapatkan nilai $a_k$ membutuhkan suku-suku sebelumnya $a_{k-1}, a_{k-2}, \dots,$ atau $a_1$. 

Contoh paling terkenal dari persamaan rekursif adalah bilangan Fibonacci $0,1,1,2,3,5,8,13,21,\dots$ yang secara matematis didefinisikan sebagai berikut.
\begin{align*}
    F_0 &= 0, F_1 = 1\\
    F_n &= F_{n-1}+F_{n-2} \text{ untuk } n \ge 2
\end{align*}

atau yang lebih terkenal di ranah \textit{Computer Science} adalah permasalahan \textit{Tower of Hanoi} dengan persamaan rekursifnya didefinisikan sebagai berikut.
\begin{align*}
    T_1 &= 1 \\
    T_n &= 2T_{n-1}+1
\end{align*}

Untuk menyelesaikan soal relasi rekurensi, butuh manipulasi aljabar yang mumpuni sehingga tidak ada pendekatan eksplisit selain menggunakan persamaan karakteristik atau fungsi pembangkit (tidak dibahas disini) yang dijamin berhasil.

\subsubsection{Persamaan Karakteristik untuk Relasi Rekurensi Linear}
Persamaan karakteristik berikut berlaku untuk persamaan rekursif yang linear. Persamaan karakteristik berikut berguna untuk mengubah relasi rekurensi menjadi iteratif, atau persamaan berbentuk implisit. (Jadi, untuk persamaan yang bukan linear, sebagai contoh $a_n = a_{n-1}^2 + a_{n-2}$ tidak bisa dijamin selesai dengan persamaan karakteristik yang disajikan berikut).
Untuk persamaan rekursif
$$a_n = c_1a_{n-1}+c_2a_{n-2}+\dots+c_da_{n-d}$$
mempunyai persamaan karakteristik
$$x^d-c_1x^{d-1}-c_2x^{d-2}-\dots-c_dx^0=0$$

Sebagai contoh, rumus rekursif dari barisan Fibonacci di atas dapat diselesaikan menjadi 
$$x^{n}-x^{n-1}-x^{n-2}=0 \implies x^2-x-1=0$$
yang mempunyai dua akar, yaitu $x_1 = \dfrac{1+\sqrt{5}}{2}=\phi$ dan $x_2 = \dfrac{1-\sqrt{5}}{2}=1-\phi$ dimana $\phi$ adalah \textit{Golden Ratio}. Sadari bahwa setiap suku di barisan Fibonacci tersebut berbentuk $F_n = c_1x_1^n + c_2x_2^n$ (buktikan). Dengan substitusi $x_1$ dan $x_2$ serta pemilihan suku dari barisan Fibonacci (misal suku pertama dan kedua) maka akan ditemukan nilai $c_1$ dan $c_2$ sehingga pada akhirnya kita punya
$$F_n=\dfrac{\phi^n-(1-\phi)^n}{\sqrt{5}}$$

\subsection{Latihan Soal Relasi Rekurensi}
\begin{enumerate}
    \item (British) Isaac is planning a nine-day holiday. Every day he will go surfing, or water skiing, or he will rest. On any given day he does just one of these three things. He never does different water-sports on consecutive days. How many schedules are possible for the holiday?

    \item (AIME I 2006) A collection of 8 cubes consists of one cube with edge-length $k$ for each integer $k, 1 \le k \le 8.$ A tower is to be built using all 8 cubes according to the rules:
    \begin{itemize}
        \item Any cube may be the bottom cube in the tower.
        \item The cube immediately on top of a cube with edge-length $k$ must have edge-length at most $k+2.$
    \end{itemize}
    Let $T$ be the number of different towers than can be constructed. What is the remainder when $T$ is divided by 1000?

    \item (AIME I 2006) For each even positive integer $x$, let $g(x)$ denote the greatest power of 2 that divides $x.$ For example, $g(20)=4$ and $g(16)=16.$ For each positive integer $n,$ let $S_n=\sum_{k=1}^{2^{n-1}}g(2k).$ Find the greatest integer $n$ less than 1000 such that $S_n$ is a perfect square.

    \item (AIME I 2001) A mail carrier delivers mail to the nineteen houses on the east side of Elm Street. The carrier notices that no two adjacent houses ever get mail on the same day, but that there are never more than two houses in a row that get no mail on the same day. How many different patterns of mail delivery are possible?
\end{enumerate}